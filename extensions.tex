\section{Extensions of the Theory}
\label{sec:extensions}

In this section we take a look in different extensions that we could add to
our theory without having to go through the sanity theorem all over again.
Remark that without these extensions our type theory would remain empty as there
are no base types.

The way we proceed is simple, if we add elements of syntax and rules, we don't
have to redo the proof entirely, we only have to deal with the new cases it
brings in the induction as well as reproving all inversion lemmata
(this explains why we are so careful in not using inversion inside the sanity
proof without explicitely refering to it as an external lemma).

\subsection{Empty Type}
\label{sec:empty-type}

We extend the syntax with the following:
%
\begin{align*}
  \text{Type $\A$, $\B$, $\C$}
    \bnf   {}& \ldots                   && \text{previous types}\\
    \bnfor {}& \Empty                   && \text{empty type} \\
  \\
  \text{Term $\uu$, $\vv$, $\ww$}
    \bnf   {}& \ldots                   && \text{previous terms} \\
    \bnfor {}& \exfalso{\A} \uu         && \text{ex falso quod libet}
\end{align*}

We now lay the new rules.

\newcommand{\rlTyEmpty}{\referTo{ty-empty}{rul:ty-empty}}
\newcommand{\showTyEmpty}{%
  \infer[\rulename{ty-empty}] % TyEmpty
  {\isctx{\G}}
  {\istype{\G}{\Empty}}
}

\newcommand{\rlTermExfalso}{\referTo{term-exfalso}{rul:term-exfalso}}
\newcommand{\showTermExfalso}{%
\infer[\rulename{term-exfalso}] % TermExfalso
  {\istype{\G}{\A} \\
   \isterm{\G}{\uu}{\Empty}
  }
  {\isterm{\G}{\exfalso{\A} \uu}{\A}}
}

\newcommand{\rlEqTySubstEmpty}{\referTo{eq-ty-subst-empty}{rul:eq-ty-subst-empty}}
\newcommand{\showEqTySubstEmpty}{%
\infer[\rulename{eq-ty-subst-empty}] % EqTySubstEmpty
  {\issubst{\sbs}{\G}{\D}}
  {\eqtype{\G}{\subst{\Empty}{\sbs}}{\Empty}}
}

\newcommand{\rlEqSubstExfalso}{\referTo{eq-subst-exfalso}{rul:eq-subst-exfalso}}
\newcommand{\showEqSubstExfalso}{%
\infer[\rulename{eq-subst-exfalso}] % EqSubstExfalso
  {\issubst{\sbs}{\G}{\D} \\
   \istype{\D}{\A} \\
   \isterm{\D}{\uu}{\Empty}
  }
  {\eqterm{\G}
    {\subst{(\exfalso{\A} \uu)}{\sbs}}
    {\exfalso{\subst{\A}{\sbs}} \subst{\uu}{\sbs}}
    {\subst{\A}{\sbs}}
  }
}

\newcommand{\rlConfExfalso}{\referTo{cong-exfalso}{rul:cong-exfalso}}
\newcommand{\showConfExfalso}{%
\infer[\rulename{cong-exfalso}] % ConfExfalso
  {\eqtype{\G}{\A}{\B} \\
   \eqterm{\G}{\uu}{\vv}{\Empty}
  }
  {\eqterm{\G}
    {\exfalso{\A} \uu}
    {\exfalso{\B} \vv}
    {\A}
  }
}

\begin{mathpar}
  {\label{rul:ty-empty} \showTyEmpty}

  {\label{rul:term-exfalso} \showTermExfalso}

  {\label{rul:eq-ty-subst-empty} \showEqTySubstEmpty}

  {\label{rul:eq-subst-exfalso} \showEqSubstExfalso}

  {\label{rul:cong-exfalso} \showConfExfalso}
\end{mathpar}

First notice that the inversion lemma~\ref{pbm:id-inversion} still holds with
the same proof as none of the rules we introduce can yield
$\istype{\G}{\Id{\A}{\uu}{\vv}}$.
Now let's extend the proof of sanity~\ref{pbm:sanity}.

\subsubsection*{Rule {\rlTyEmpty}}

Consider a derivation ending with
%
\begin{equation*}
  \showTyEmpty
\end{equation*}
%
We have $\isctx{\G}$ as a premise.


\subsubsection*{Rule {\rlTermExfalso}}

Consider a derivation ending with
%
\begin{equation*}
  \showTermExfalso
\end{equation*}
%
We have $\isctx{\G}$ by induction hypothesis and $\istype{\G}{\A}$ as a premise.


\subsubsection*{Rule {\rlEqTySubstEmpty}}

Consider a derivation ending with
%
\begin{equation*}
  \showEqTySubstEmpty
\end{equation*}
%
We have $\isctx{\G}$ and $\isctx{\D}$ by induction hypothesis.
Using {\rlTyEmpty} we derive $\istype{\G}{\Empty}$ and $\istype{\D}{\Empty}$,
then using {\rlTySubst} we conclude $\istype{\G}{\subst{\Empty}{\sbs}}$.


\subsubsection*{Rule {\rlEqSubstExfalso}}

Consider a derivation ending with
%
\begin{equation*}
  \showEqSubstExfalso
\end{equation*}
%
By induction hypothesis on the left premise we get $\isctx{\G}$.
Using {\rlTySubst} we get $\istype{\G}{\subst{\A}{\sbs}}$.
Using {\rlTermExfalso} we derive $\isterm{\D}{\exfalso{\A} \uu}{\A}$
and then with {\rlTermSubst} we conclude
$\isterm{\G}{\subst{(\exfalso{\A} \uu)}{\sbs}}{\subst{\A}{\sbs}}$.
Using {\rlTermSubst} and then {\rlTermExfalso} yields
$\isterm{\G}{\exfalso{\subst{\A}{\sbs}} \subst{\uu}{\sbs}}{\subst{\A}{\sbs}}$.


\subsubsection*{Rule {\rlConfExfalso}}

Consider a derivation ending with
%
\begin{equation*}
  \showConfExfalso
\end{equation*}
%
By induction hypothesis on the left premise $\isctx{\G}$ and $\istype{\G}{\A}$
and $\istype{\G}{\B}$.
Induction hypothesis on the right premise yields $\isterm{\G}{\uu}{\Empty}$
and $\isterm{\G}{\vv}{\Empty}$.
Applications of {\rlTermExfalso} give $\isterm{\G}{\exfalso{\A} \uu}{\A}$
and $\isterm{\G}{\exfalso{\B} \vv}{\B}$.
We conclude $\isterm{\G}{\exfalso{\B} \vv}{\A}$ with {\rlEqTySym} and
{\rlTermTyConv}.


\subsection{Unit Type}
\label{sec:unit-type}

\begin{align*}
  \text{Type $\A$, $\B$, $\C$}
    \bnf   {}& \ldots                   && \text{previous types}\\
    \bnfor {}& \Unit                    && \text{unit type} \\
  \\
  \text{Term $\uu$, $\vv$, $\ww$}
    \bnf   {}& \ldots                   && \text{previous terms} \\
    \bnfor {}& \unit                    && \text{the unit inhabitant} \\
    \bnfor {}& \ulet{\A}{\uu} \vv       && \text{the unit eliminator}
\end{align*}

\newcommand{\rlTyUnit}{\referTo{ty-unit}{rul:ty-unit}}
\newcommand{\showTyUnit}{%
  \infer[\rulename{ty-unit}] % TyUnit
  {\isctx{\G}}
  {\istype{\G}{\Unit}}
}

\newcommand{\rlTermUnit}{\referTo{term-unit}{rul:term-unit}}
\newcommand{\showTermUnit}{%
  \infer[\rulename{term-unit}] % TermUnit
  {\isctx{\G}}
  {\isterm{\G}{\unit}{\Unit}}
}

\newcommand{\rlTermLet}{\referTo{term-let}{rul:term-let}}
\newcommand{\showTermLet}{%
  \infer[\rulename{term-let}] % TermLet
  {\istype{\ctxextend{\G}{\Unit}}{\A} \\
   \isterm{\G}{\uu}{\Unit} \\
   \isterm{\G}{\vv}{\subst{\A}{\sbzero{\G}{\Unit}{\unit}}}}
  {\isterm{\G}{\ulet{\A}{\uu} \vv}{\subst{\A}{\sbzero{\G}{\Unit}{\uu}}}}
}

\newcommand{\rlEqTySubstUnit}{\referTo{eq-ty-subst-unit}{rul:eq-ty-subst-unit}}
\newcommand{\showEqTySubstUnit}{%
  \infer[\rulename{eq-ty-subst-unit}] % EqTySubstUnit
  {\issubst{\sbs}{\G}{\D}}
  {\eqtype{\G}{\subst{\Unit}{\sbs}}{\Unit}}
}

\newcommand{\rlEqSubstUnit}{\referTo{eq-subst-unit}{rul:eq-subst-unit}}
\newcommand{\showEqSubstUnit}{%
  \infer[\rulename{eq-subst-unit}] % EqSubstUnit
  {\issubst{\sbs}{\G}{\D}}
  {\eqterm{\G}{\subst{\unit}{\sbs}}{\unit}{\Unit}}
}

\newcommand{\rlEqSubstLet}{\referTo{eq-subst-let}{rul:eq-subst-let}}
\newcommand{\showEqSubstLet}{%
  \infer[\rulename{eq-subst-let}] % EqSubstLet
  {\issubst{\sbs}{\G}{\D} \\
   \istype{\ctxextend{\D}{\Unit}}{\A} \\
   \isterm{\D}{\uu}{\Unit} \\
   \isterm{\D}{\vv}{\subst{\A}{\sbzero{\D}{\Unit}{\unit}}}}
  {\eqterm{\G}
    {\subst{(\ulet{\A}{\uu} \vv)}{\sbs}}
    {\ulet
      {\subst{\A}{\sbshift{\G}{\Unit}{\sbs}}}
      {\subst{\uu}{\sbs}}
      \subst{\vv}{\sbs}
    }
    {\subst{\subst{\A}{\sbzero{\D}{\Unit}{\uu}}}}{\sbs}}
}

\newcommand{\rlLetBeta}{\referTo{let-beta}{rul:let-beta}}
\newcommand{\showLetBeta}{%
  \infer[\rulename{let-beta}] % LetBeta
  {\istype{\ctxextend{\G}{\Unit}}{\A} \\
   \isterm{\G}{\vv}{\subst{\A}{\sbzero{\G}{\Unit}{\unit}}}}
  {\eqterm{\G}
    {\ulet{\A}{\unit} \vv}
    {\vv}
    {\subst{\A}{\sbzero{\G}{\Unit}{\unit}}}}
}

\newcommand{\rlCongLet}{\referTo{cong-let}{rul:cong-let}}
\newcommand{\showCongLet}{%
  \infer[\rulename{cong-let}] % CongLet
  {\eqtype{\ctxextend{\G}{\Unit}}{\A}{\B} \\
   \eqterm{\G}{\uu_1}{\uu_2}{\Unit} \\
   \eqterm{\G}{\vv_1}{\vv_2}{\subst{\A}{\sbzero{\G}{\Unit}{\unit}}}}
  {\eqterm{\G}
    {\ulet{\A}{\uu_1} \vv_1}
    {\ulet{\B}{\uu_2} \vv_2}
    {\subst{\A}{\sbzero{\G}{\Unit}{\uu_1}}}}
}

\begin{mathpar}
  {\label{rul:ty-unit} \showTyUnit}

  {\label{rul:term-unit} \showTermUnit}

  {\label{rul:term-let} \showTermLet}

  {\label{rul:eq-ty-subst-unit} \showEqTySubstUnit}

  {\label{rul:eq-subst-unit} \showEqSubstUnit}

  {\label{rul:eq-subst-let} \showEqSubstLet}

  {\label{rul:let-beta} \showLetBeta}

  {\label{rul:cong-let} \showCongLet}
\end{mathpar}

\subsubsection*{Rule {\rlTyUnit}}

Consider a derivation ending with
%
\begin{equation*}
  \showTyUnit
\end{equation*}
%
\meta{TODO}

\subsubsection*{Rule {\rlTermUnit}}

Consider a derivation ending with
%
\begin{equation*}
  \showTermUnit
\end{equation*}
%
\meta{TODO}

\subsubsection*{Rule {\rlTermLet}}

Consider a derivation ending with
%
\begin{equation*}
  \showTermLet
\end{equation*}
%
\meta{TODO}

\subsubsection*{Rule {\rlEqTySubstUnit}}

Consider a derivation ending with
%
\begin{equation*}
  \showEqTySubstUnit
\end{equation*}
%
\meta{TODO}

\subsubsection*{Rule {\rlEqSubstUnit}}

Consider a derivation ending with
%
\begin{equation*}
  \showEqSubstUnit
\end{equation*}
%
\meta{TODO}

\subsubsection*{Rule {\rlEqSubstLet}}

Consider a derivation ending with
%
\begin{equation*}
  \showEqSubstLet
\end{equation*}
%
\meta{TODO}

\subsubsection*{Rule {\rlLetBeta}}

Consider a derivation ending with
%
\begin{equation*}
  \showLetBeta
\end{equation*}
%
\meta{TODO}

\subsubsection*{Rule {\rlCongLet}}

Consider a derivation ending with
%
\begin{equation*}
  \showCongLet
\end{equation*}
%
\meta{TODO}

\subsection{Boolean Type}
\meta{TODO}

\subsection{J eliminator}
\meta{TODO}

\subsection{W Types}
\meta{TODO?}
