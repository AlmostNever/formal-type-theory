\subsection{Empty Type}
\label{sec:empty-type}

We extend the syntax as follows:
%
\begin{align*}
  \text{Type $\A$, $\B$, $\C$}
    \bnf   {}& \ldots                   && \text{previous types}\\
    \bnfor {}& \Empty                   && \text{empty type} \\
  \\
  \text{Term $\uu$, $\vv$, $\ww$}
    \bnf   {}& \ldots                   && \text{previous terms} \\
    \bnfor {}& \exfalso{\A} \uu         && \text{ex falso quod libet}
\end{align*}

We now lay the new rules.

\newcommand{\rlTyEmpty}{\referTo{ty-empty}{rul:ty-empty}}
\newcommand{\showTyEmpty}{%
  \infer[\rulename{ty-empty}] % TyEmpty
  {\isctx{\G}}
  {\istype{\G}{\Empty}}
}

\newcommand{\rlTermExfalso}{\referTo{term-exfalso}{rul:term-exfalso}}
\newcommand{\showTermExfalso}{%
  \infer[\rulename{term-exfalso}] % TermExfalso
  {\istype{\G}{\A} \\
   \isterm{\G}{\uu}{\Empty}
  }
  {\isterm{\G}{\exfalso{\A} \uu}{\A}}
}

\newcommand{\rlEqTySubstEmpty}{\referTo{eq-ty-subst-empty}{rul:eq-ty-subst-empty}}
\newcommand{\showEqTySubstEmpty}{%
  \infer[\rulename{eq-ty-subst-empty}] % EqTySubstEmpty
  {\issubst{\sbs}{\G}{\D}}
  {\eqtype{\G}{\subst{\Empty}{\sbs}}{\Empty}}
}

\newcommand{\rlEqSubstExfalso}{\referTo{eq-subst-exfalso}{rul:eq-subst-exfalso}}
\newcommand{\showEqSubstExfalso}{%
  \infer[\rulename{eq-subst-exfalso}] % EqSubstExfalso
  {\issubst{\sbs}{\G}{\D} \\
   \istype{\D}{\A} \\
   \isterm{\D}{\uu}{\Empty}
  }
  {\eqterm{\G}
    {\subst{(\exfalso{\A} \uu)}{\sbs}}
    {\exfalso{\subst{\A}{\sbs}} \subst{\uu}{\sbs}}
    {\subst{\A}{\sbs}}
  }
}

\newcommand{\rlCongExfalso}{\referTo{cong-exfalso}{rul:cong-exfalso}}
\newcommand{\showCongExfalso}{%
  \infer[\rulename{cong-exfalso}] % CongExfalso
  {\eqtype{\G}{\A}{\B} \\
   \eqterm{\G}{\uu}{\vv}{\Empty}
  }
  {\eqterm{\G}
    {\exfalso{\A} \uu}
    {\exfalso{\B} \vv}
    {\A}
  }
}

\newcommand{\rlEqTyExfalso}{\referTo{eq-ty-exfalso}{rul:eq-ty-exfalso}}
\newcommand{\showEqTyExfalso}{%
  \infer[\rulename{eq-ty-exfalso}] % EqTyExfalso
  {\istype{\G}{\A} \\
    \istype{\G}{\B} \\
    \isterm{\G}{\uu}{\Empty}
  }
  {\eqtype{\G}{\A}{\B}
  }
}

\newcommand{\rlEqTermExfalso}{\referTo{eq-term-exfalso}{rul:eq-term-exfalso}}
\newcommand{\showEqTermExfalso}{%
  \infer[\rulename{eq-term-exfalso}] % EqTermExfalso
  {\istype{\G}{\A} \\
    \isterm{\G}{\uu}{\A} \\
    \isterm{\G}{\vv}{\A} \\
    \isterm{\G}{\ww}{\Empty}
  }
  {\eqterm{\G}{\uu}{\vv}{\A}
  }
}

\begin{mathpar}
  {\label{rul:ty-empty} \showTyEmpty}

  {\label{rul:term-exfalso} \showTermExfalso}

  {\label{rul:eq-ty-subst-empty} \showEqTySubstEmpty}

  {\label{rul:eq-subst-exfalso} \showEqSubstExfalso}

  {\label{rul:cong-exfalso} \showCongExfalso}

  {\label{rul:eq-ty-exfalso} \showEqTyExfalso}

  {\label{rul:eq-term-exfalso} \showEqTermExfalso}
\end{mathpar}

First notice that the inversion lemma~\ref{pbm:id-inversion} still holds with
the same proof as none of the rules we introduce can yield
$\istype{\G}{\Id{\A}{\uu}{\vv}}$.
Now let's extend the proof of sanity~\ref{pbm:sanity}.

\subsubsection*{Rule {\rlTyEmpty}}

Consider a derivation ending with
%
\begin{equation*}
  \showTyEmpty
\end{equation*}
%
We have $\isctx{\G}$ as a premise.


\subsubsection*{Rule {\rlTermExfalso}}

Consider a derivation ending with
%
\begin{equation*}
  \showTermExfalso
\end{equation*}
%
We have $\isctx{\G}$ by induction hypothesis and $\istype{\G}{\A}$ as a premise.


\subsubsection*{Rule {\rlEqTySubstEmpty}}

Consider a derivation ending with
%
\begin{equation*}
  \showEqTySubstEmpty
\end{equation*}
%
We have $\isctx{\G}$ and $\isctx{\D}$ by induction hypothesis.
Using {\rlTyEmpty} we derive $\istype{\G}{\Empty}$ and $\istype{\D}{\Empty}$,
then using {\rlTySubst} we conclude $\istype{\G}{\subst{\Empty}{\sbs}}$.


\subsubsection*{Rule {\rlEqSubstExfalso}}

Consider a derivation ending with
%
\begin{equation*}
  \showEqSubstExfalso
\end{equation*}
%
By induction hypothesis on the left premise we get $\isctx{\G}$.
Using {\rlTySubst} we get $\istype{\G}{\subst{\A}{\sbs}}$.
Using {\rlTermExfalso} we derive $\isterm{\D}{\exfalso{\A} \uu}{\A}$
and then with {\rlTermSubst} we conclude
$\isterm{\G}{\subst{(\exfalso{\A} \uu)}{\sbs}}{\subst{\A}{\sbs}}$.
Using {\rlTermSubst} and then {\rlTermExfalso} yields
$\isterm{\G}{\exfalso{\subst{\A}{\sbs}} \subst{\uu}{\sbs}}{\subst{\A}{\sbs}}$.


\subsubsection*{Rule {\rlCongExfalso}}

Consider a derivation ending with
%
\begin{equation*}
  \showCongExfalso
\end{equation*}
%
By induction hypothesis on the left premise $\isctx{\G}$ and $\istype{\G}{\A}$
and $\istype{\G}{\B}$.
Induction hypothesis on the right premise yields $\isterm{\G}{\uu}{\Empty}$
and $\isterm{\G}{\vv}{\Empty}$.
Applications of {\rlTermExfalso} give $\isterm{\G}{\exfalso{\A} \uu}{\A}$
and $\isterm{\G}{\exfalso{\B} \vv}{\B}$.
We conclude $\isterm{\G}{\exfalso{\B} \vv}{\A}$ with {\rlEqTySym} and
{\rlTermTyConv}.

\subsubsection*{Rule {\rlEqTyExfalso}}

Consider a derivation ending with
%
\begin{equation*}
  \showEqTyExfalso
\end{equation*}
%
By induction hypothesis on the left premise we have $\isctx{\G}$ and $\istype{\G}{\A}$,
while induction on the right premise yields $\istype{\G}{\B}$.

\subsubsection*{Rule {\rlEqTermExfalso}}

Consider a derivation ending with
%
\begin{equation*}
  \showEqTermExfalso
\end{equation*}
%
By induction hypothesis on the left premise we have $\isctx{\G}$, and the rest is obtained
directly from the premises.

%%% Local Variables:
%%% mode: latex
%%% TeX-master: "main"
%%% End:
