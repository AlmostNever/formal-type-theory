% \documentclass[draft]{amsart}
\documentclass{amsart}

\usepackage[utf8]{inputenc}
\usepackage{times}
\usepackage{amsmath,amssymb}
\usepackage{amsthm}
\usepackage{mathpartir}
\usepackage{hyperref}

% Add some colors
\usepackage[usenames,dvipsnames,svgnames,table]{xcolor}

% Meta comment
\newcommand\meta[1]{\noindent\textcolor{blue}{\emph{#1}}}

% Include the macro file
% evergreens
\newcommand{\der}{\,\vdash}
\newcommand{\Der}{\,\Vdash}

%specific judgments
\newcommand{\dere}{\der_\mathbf{E}}

% semantic brackets
\def\lv{\mathopen{{[\kern-0.14em[}}}    % opening [[ value delimiter
\def\rv{\mathclose{{]\kern-0.14em]}}}   % closing ]] value delimiter
\newcommand{\den}[1]{\lv #1 \rv}
\newcommand{\Den}[3][]{\den{#2}^{#1}_{#3}}
\newcommand{\dent}[2]{\llparenthesis#1\rrparenthesis_{#2}}

% space-preserving paragraph headings
\newcommand{\subheading}[1]{\subparagraph{#1.}} %Alt: \subsection{#1}
% \newcommand{\paradot}[1]{\subparagraph{#1.}}
\newcommand{\paradot}[1]{\subsection*{#1.}}

% Inference rules
\newcommand{\rulename}[1]{\ensuremath{\mbox{\sc#1}}}
\newcommand{\ru}[2]{\dfrac{\begin{array}[b]{@{}c@{}} #1 \end{array}}{#2}}
\newcommand{\rux}[3]{\ru{#1}{#2}~#3}
\newcommand{\nru}[3]{#1\ \ru{#2}{#3}}
\newcommand{\nrux}[4]{#1\ \ru{#2}{#3}\ #4}
\newcommand{\dstack}[2]{\begin{array}[b]{c}#1\\#2\end{array}}
\newcommand{\dru}[3]{\ru{\dstack{#1}{#2}}{#3}}
\newcommand{\drux}[4]{\ru{\dstack{#1}{#2}}{#3}\ #4}
\newcommand{\tru}[4]{\dru{\dstack{#1}{#2}}{#3}{#4}}
\newcommand{\trux}[5]{\dru{\dstack{#1}{#2}}{#3}{#4}\ #5}
\newcommand{\qru}[5]{\tru{\dstack{#1}{#2}}{#3}{#4}{#5}}
\newcommand{\ndru}[4]{#1\ \ru{\dstack{#2}{#3}}{#4}}
\newcommand{\ndrux}[5]{#1\ \ru{\dstack{#2}{#3}}{#4}\ #5}

% Centered math environment
\newenvironment{mathc}{%
  \begin{center}%
  \(%
}{\)%
  \end{center}%
}

% proof by cases
\newenvironment{caselist}{%
  \begin{list}{{\it Case}}{}%
}{\end{list}%
}
\newenvironment{subcaselist}{%
  \begin{list}{{\it Subcase}}{}%
}{\end{list}%
}
\newenvironment{subsubcaselist}{%
  \begin{list}{{\it Subsubcase}}{}%
}{\end{list}%
}

\newcommand{\nextcase}{\item~}

% Symbols and names
\DeclareMathOperator{\J}{\mathbf{J}}
\DeclareMathOperator{\type}{\mathbf{type}}
\newcommand{\Id}[3]{\mathbf{Id}_{#1}\,#2\ #3}
\newcommand{\app}[3]{#2\ @^{#1}\ #3}
\newcommand\infers{\rightrightarrows}
\newcommand\checks{\leftleftarrows}
\DeclareMathOperator{\refl}{\mathbf{refl}}
\newcommand\barG{\overline{\Gamma}}
\newcommand\bart{\overline{t}}
\newcommand\barT{\overline{T}}
\newcommand\barA{\overline{A}}
\newcommand\barB{\overline{B}}
\newcommand\baru{\overline{u}}
\newcommand\barv{\overline{v}}
\newcommand\barp{\overline{p}}

% All the inference rules are defined here as macros.

\newcommand{\rulename}[1]{\text{\textsc{#1}}}
\newcommand{\referTo}[2]{\hyperref[#2]{$\rulename{#1}$}}

%%%% Contexts


\newcommand{\rlCtxEmpty}{\referTo{ctx-empty}{rul:ctx-empty}}
\newcommand{\showCtxEmpty}{%
  \infer[\rulename{ctx-empty}] % CtxEmpty
  { }
  {\isctx{\ctxempty}}
}


\newcommand{\rlCtxExtend}{\referTo{ctx-extend}{rul:ctx-extend}}
\newcommand{\showCtxExtend}{%
  \infer[\rulename{ctx-extend}] % CtxExtend
  {\isctx{\G} \\
   \istype{\G}{\A}
  }
  {\isctx{(\ctxextend{\G}{\A})}}
}


%%%% Substitutions

\newcommand{\rlSubstZero}{\referTo{subst-zero}{rul:subst-zero}}
\newcommand{\showSubstZero}{%
  \infer[\rulename{subst-zero}] % SubstZero
  {\isterm{\G}{\uu}{\A}}
  {\issubst{\sbzero{\G}{\A}{\uu}}{\G}{\ctxextend{\G}{\A}}}
}

\newcommand{\rlSubstWeak}{\referTo{subst-weak}{rul:subst-weak}}
\newcommand{\showSubstWeak}{%
  \infer[\rulename{subst-weak}] % SubstWeak
  {\istype{\G}{\A}}
  {\issubst
    {\sbweak{\G}{\A}}
    {\ctxextend{\G}{\A}}
    {\G}
  }
}

\newcommand{\rlSubstShift}{\referTo{subst-shift}{rul:subst-shift}}
\newcommand{\showSubstShift}{%
  \infer[\rulename{subst-shift}] % SubstShift
  {\issubst{\sbs}{\G}{\D} \\
   \istype{\D}{\A}
  }
  {\issubst
      {(\sbshift{\G}{\A}{\sbs})}
      {\ctxextend{\G}{\subst{\A}{\sbs}}}
      {\ctxextend{\D}{\A}}
  }
}

\newcommand{\rlSubstId}{\referTo{subst-id}{rul:subst-id}}
\newcommand{\showSubstId}{%
  \infer[\rulename{subst-id}] % SubstId
  {\isctx{\G}}
  {\issubst
    {\sbid{\G}}
    {\G}
    {\G}
  }
}

\newcommand{\rlSubstComp}{\referTo{subst-comp}{rul:subst-comp}}
\newcommand{\showSubstComp}{%
  \infer[\rulename{subst-comp}] % SubstComp
  {\issubst{\sbs}{\G}{\D} \\
   \issubst{\sbt}{\D}{\E}
  }
  {\issubst
    {\sbcomp{\sbs}{\sbt}}
    {\G}
    {\E}
  }
}

\newcommand{\rlSubstCtxConv}{\referTo{subst-ctx-conv}{rul:subst-ctx-conv}}
\newcommand{\showSubstCtxConv}{%
  \infer[\rulename{subst-ctx-conv}] % SubstCtxConv
  {\issubst{\sbs}{\G_1}{\D_1} \\
   \eqctx{\G_1}{\G_2} \\
   \eqctx{\D_1}{\D_2}
  }
  {\issubst
    {\sbs}
    {\G_2}
    {\D_2}
  }
}

%%%% Types


\newcommand{\rlTyCtxConv}{\referTo{ty-ctx-conv}{rul:ty-ctx-conv}}
\newcommand{\showTyCtxConv}{%
  \infer[\rulename{ty-ctx-conv}] % TyCtxConv
  {\istype{\G}{\A} \\
    \eqctx{\G}{\D}
  }
  {\istype{\D}{\A}}
}


\newcommand{\rlTySubst}{\referTo{ty-subst}{rul:ty-subst}}
\newcommand{\showTySubst}{%
  \infer[\rulename{ty-subst}] % TySubst
  {\issubst{\sbs}{\G}{\D} \\
    \istype{\D}{\A}
  }
  {\istype{\G}{\subst{\A}{\sbs}}}
}

% We want \istype{\G}{\A} because in the sanity theorem we
% need to prove that \G is a context (without inversion on the
% extended context).

\newcommand{\rlTyProd}{\referTo{ty-prod}{rul:ty-prod}}
\newcommand{\showTyProd}{%
  \infer[\rulename{ty-prod}] % TyProd
  {\istype{\G}{\A} \\
    \istype{\ctxextend{\G}{\A}}{\B}
  }
  {\istype{\G}{\Prod{\A}{\B}}}
}


\newcommand{\rlTyId}{\referTo{ty-id}{rul:ty-id}}
\newcommand{\showTyId}{%
  \infer[\rulename{ty-id}] % TyId
  {\istype{\G}{\A}\\
    \isterm{\G}{\uu}{\A}\\
    \isterm{\G}{\vv}{\A}
  }
  {\istype{\G}{\Id{\A}{\uu}{\vv}}}
}


\newcommand{\rlTyEmpty}{\referTo{ty-empty}{rul:ty-empty}}
\newcommand{\showTyEmpty}{%
  \infer[\rulename{ty-empty}] % TyEmpty
  {\isctx{\G}}
  {\istype{\G}{\Empty}}
}


\newcommand{\rlTyUnit}{\referTo{ty-unit}{rul:ty-unit}}
\newcommand{\showTyUnit}{%
  \infer[\rulename{ty-unit}] % TyUnit
  {\isctx{\G}}
  {\istype{\G}{\Unit}}
}


\newcommand{\rlTyBool}{\referTo{ty-bool}{rul:ty-bool}}
\newcommand{\showTyBool}{%
  \infer[\rulename{ty-bool}] % TyBool
  {\isctx{\G}}
  {\istype{\G}{\Bool}}
}


%%%% Terms


\newcommand{\rlTermTyConv}{\referTo{term-ty-conv}{rul:term-ty-conv}}
\newcommand{\showTermTyConv}{%
  \infer[\rulename{term-ty-conv}] % TermTyConv
  {\isterm{\G}{\uu}{\A} \\
    \eqtype{\G}{\A}{\B}
  }
  {\isterm{\G}{\uu}{\B}}
}


\newcommand{\rlTermCtxConv}{\referTo{term-ctx-conv}{rul:term-ctx-conv}}
\newcommand{\showTermCtxConv}{%
  \infer[\rulename{term-ctx-conv}] % TermCtxConv
  {\isterm{\G}{\uu}{\A} \\
    \eqctx{\G}{\D}
  }
  {\isterm{\D}{\uu}{\A}}
}


\newcommand{\rlTermSubst}{\referTo{term-subst}{rul:term-subst}}
\newcommand{\showTermSubst}{%
  \infer[\rulename{term-subst}] % TermSubst
  {\issubst{\sbs}{\G}{\D} \\
    \isterm{\D}{\uu}{\A}
  }
  {\isterm{\G}{\subst{\uu}{\sbs}}{\subst{\A}{\sbs}}}
}


\newcommand{\rlTermVarZero}{\referTo{term-var-zero}{rul:term-var-zero}}
\newcommand{\showTermVarZero}{%
  \infer[\rulename{term-var-zero}] % TermVarZero
  {
    \istype{\G}{\A}
  }
  {\isterm
    {\ctxextend{\G}{\A}}
    {\var{0}}
    {\subst{\A}{\sbweak{\G}{\A}}}
  }
}


\newcommand{\rlTermVarSucc}{\referTo{term-var-succ}{rul:term-var-succ}}
\newcommand{\showTermVarSucc}{%
  \infer[\rulename{term-var-succ}] % TermVarSucc
  {\isterm{\G}{\var{k}}{\A} \\
    \istype{\G}{\B}
  }
  {\isterm
    {\ctxextend{\G}{\B}}
    {\var{k+1}}
    {\subst{\A}{\sbweak{\G}{\B}}}
  }
}



% Remark: we want \istype{\G}{\A} as a premise because in order to form the
% product type we need it, and we do not want to resort to inversion on the
% extended context.

\newcommand{\rlTermAbs}{\referTo{term-abs}{rul:term-abs}}
\newcommand{\showTermAbs}{%
  \infer[\rulename{term-abs}] % TermAbs
  { \istype{\G}{\A} \\
    \isterm{\ctxextend{\G}{\A}}{\uu}{\B}
  }
  {\isterm{\G}{(\lam{\A}{\B}{\uu})}{\Prod{\A}{\B}}}
}


% Remark: we want \istype{\ctxextend{\G}{\A}}{\B} because we need to
% know that \B is a type in \ctxextend{\G}{\A} and we want to avoid
% inversion on ty-prod.

\newcommand{\rlTermApp}{\referTo{term-app}{rul:term-app}}
\newcommand{\showTermApp}{%
  \infer[\rulename{term-app}] % TermApp
  {\istype{\ctxextend{\G}{\A}}{\B} \\
    \isterm{\G}{\uu}{\Prod{\A} \B} \\
    \isterm{\G}{\vv}{\A}
  }
  {\isterm
    {\G}
    {\app{\uu}{\A}{\B}{\vv}}
    {\subst{\B}{\sbzero{\G}{\A}{\vv}}}
  }
}


\newcommand{\rlTermRefl}{\referTo{term-refl}{rul:term-refl}}
\newcommand{\showTermRefl}{%
  \infer[\rulename{term-refl}] % TermRefl
  {\isterm{\G}{\uu}{\A}}
  {\isterm{\G}{\refl{\A} \uu}{\Id{\A}{\uu}{\uu}}}
}


\newcommand{\rlTermJ}{\referTo{term-j}{rul:term-j}}
\newcommand{\showTermJ}{%
  \infer[\rulename{term-j}] % TermJ
  {\istype{\G}{\A} \\
   \isterm{\G}{\uu}{\A} \\
   \istype
    {\ctxextend
      {\ctxextend{\G}{\A}}
      {\Id
        {\subst{\A}{\sbweak{\G}{\A}}}
        {\subst{\uu}{\sbweak{\G}{\A}}}
        {\var{0}}}
    }
    {\C} \\
   \isterm
    {\G}
    {\ww}
    {\subst
      {\subst
        {\C}
        {\sbshift
          {\G}
          {\Id
            {\subst{\A}{\sbweak{\G}{\A}}}
            {\subst{\uu}{\sbweak{\G}{\A}}}
            {\var{0}}
          }
          {\sbzero{\G}{\A}{\uu}}
        }
      }
      {\sbzero{\G}{\Id{\A}{\uu}{\uu}}{\refl{\A} \uu}}
    } \\
   \isterm{\G}{\vv}{\A} \\
   \isterm{\G}{\p}{\Id{\A}{\uu}{\vv}}
  }
  {\isterm{\G}
    {\J{\A}{\uu}{\C}{\ww}{\vv}{\p}}
    {\subst
      {\subst
        {\C}
        {\sbshift
          {\G}
          {\Id
            {\subst{\A}{\sbweak{\G}{\A}}}
            {\subst{\uu}{\sbweak{\G}{\A}}}
            {\var{0}}
          }
          {\sbzero{\G}{\A}{\vv}}
        }
      }
      {\sbzero{\G}{\Id{\A}{\uu}{\vv}}{\p}}
    }
  }
}


\newcommand{\rlTermExfalso}{\referTo{term-exfalso}{rul:term-exfalso}}
\newcommand{\showTermExfalso}{%
  \infer[\rulename{term-exfalso}] % TermExfalso
  {\istype{\G}{\A} \\
   \isterm{\G}{\uu}{\Empty}
  }
  {\isterm{\G}{\exfalso{\A} \uu}{\A}}
}


\newcommand{\rlTermUnit}{\referTo{term-unit}{rul:term-unit}}
\newcommand{\showTermUnit}{%
  \infer[\rulename{term-unit}] % TermUnit
  {\isctx{\G}}
  {\isterm{\G}{\unit}{\Unit}}
}


\newcommand{\rlTermTrue}{\referTo{term-true}{rul:term-true}}
\newcommand{\showTermTrue}{%
  \infer[\rulename{term-true}] % TermTrue
  {\isctx{\G}}
  {\isterm{\G}{\true}{\Bool}}
}


\newcommand{\rlTermFalse}{\referTo{term-false}{rul:term-false}}
\newcommand{\showTermFalse}{%
  \infer[\rulename{term-false}] % TermFalse
  {\isctx{\G}}
  {\isterm{\G}{\false}{\Bool}}
}


\newcommand{\rlTermCond}{\referTo{term-cond}{rul:term-cond}}
\newcommand{\showTermCond}{%
  \infer[\rulename{term-cond}] % TermCond
  {\isterm{\G}{\uu}{\Bool} \\
   \istype{\ctxextend{\G}{\Bool}}{\C} \\
   \isterm{\G}{\vv}{\subst{\C}{\sbzero{\G}{\Bool}{\true}}} \\
   \isterm{\G}{\ww}{\subst{\C}{\sbzero{\G}{\Bool}{\false}}}
  }
  {\isterm{\G}{\cond{\C}{\uu}{\vv}{\ww}}{\subst{\C}{\sbzero{\G}{\Bool}{\uu}}}}
}


%%%% Context equality


\newcommand{\rlCtxRefl}{\referTo{ctx-refl}{rul:ctx-refl}}
\newcommand{\showCtxRefl}{%
  \infer[\rulename{ctx-refl}] % CtxRefl
  {\isctx{\G}}
  {\eqctx{\G}{\G}}
}


\newcommand{\rlCtxSym}{\referTo{ctx-sym}{rul:ctx-sym}}
\newcommand{\showCtxSym}{%
  \infer[\rulename{ctx-sym}] % CtxSym
  {\eqctx{\G}{\D}}
  {\eqctx{\D}{\G}}
}


\newcommand{\rlCtxTrans}{\referTo{ctx-trans}{rul:ctx-trans}}
\newcommand{\showCtxTrans}{%
  \infer[\rulename{ctx-trans}] % CtxTrans
  {\eqctx{\G}{\D} \\
   \eqctx{\D}{\E}
  }
  {\eqctx{\G}{\E}}
}


\newcommand{\rlEqCtxEmpty}{\referTo{eq-ctx-empty}{rul:eq-ctx-empty}}
\newcommand{\showEqCtxEmpty}{%
  \infer[\rulename{eq-ctx-empty}] % EqCtxEmpty
  { }
  {\eqctx{\ctxempty}{\ctxempty}}
}


\newcommand{\rlEqCtxExtend}{\referTo{eq-ctx-extend}{rul:eq-ctx-extend}}
\newcommand{\showEqCtxExtend}{%
  \infer[\rulename{eq-ctx-extend}] % EqCtxExtend
  {
    \eqctx{\G}{\D} \\
    \eqtype{\G}{\A}{\B}
  }
  {\eqctx{(\ctxextend{\G}{\A})}{(\ctxextend{\D}{\B})}}
}


%%% Substitution equality


\newcommand{\rlSubstRefl}{\referTo{subst-refl}{rul:subst-refl}}
\newcommand{\showSubstRefl}{%
  \infer[\rulename{subst-refl}] % SubstRefl
  {\issubst{\sbs}{\G}{\D}}
  {\eqsubst{\sbs}{\sbs}{\G}{\D}}
}


\newcommand{\rlSubstSym}{\referTo{subst-sym}{rul:subst-sym}}
\newcommand{\showSubstSym}{%
  \infer[\rulename{subst-sym}] % SubstSym
  {\eqsubst{\sbs}{\sbt}{\G}{\D}}
  {\eqsubst{\sbt}{\sbs}{\G}{\D}}
}


\newcommand{\rlSubstTrans}{\referTo{subst-trans}{rul:subst-trans}}
\newcommand{\showSubstTrans}{%
  \infer[\rulename{subst-trans}] % SubstTrans
  {\eqsubst{\sbs}{\sbt}{\G}{\D} \\
   \eqsubst{\sbt}{\sbr}{\G}{\D}
  }
  {\eqsubst{\sbs}{\sbr}{\G}{\D}}
}


\newcommand{\rlCongSubstZero}{\referTo{cong-subst-zero}{rul:cong-subst-zero}}
\newcommand{\showCongSubstZero}{%
  \infer[\rulename{cong-subst-zero}] % CongSubstZero
  {\eqctx{\G_1}{\G_2} \\
   \eqtype{\G_1}{\A_1}{\A_2} \\
   \eqterm{\G_1}{\uu_1}{\uu_2}{\A_1}
  }
  {\eqsubst
    {\sbzero{\G_1}{\A_1}{\uu_1}}
    {\sbzero{\G_2}{\A_2}{\uu_2}}
    {\G_1}
    {\ctxextend{\G_1}{\A_1}}
  }
}


\newcommand{\rlCongSubstWeak}{\referTo{cong-subst-weak}{rul:cong-subst-weak}}
\newcommand{\showCongSubstWeak}{%
  \infer[\rulename{cong-subst-weak}] % CongSubstWeak
  {\eqctx{\G_1}{\G_2} \\
   \eqtype{\G_1}{\A_1}{\A_2}
  }
  {\eqsubst
    {\sbweak{\G_1}{\A_1}}
    {\sbweak{\G_2}{\A_2}}
    {\ctxextend{\G_1}{\A_1}}
    {\G_1}
  }
}


\newcommand{\rlCongSubstShift}{\referTo{cong-subst-shift}{rul:cong-subst-shift}}
\newcommand{\showCongSubstShift}{%
  \infer[\rulename{cong-subst-shift}] % CongSubstShift
  {\eqctx{\G_1}{\G_2} \\
   \eqsubst{\sbs}{\sbt}{\G_1}{\D} \\
   \eqtype{\D}{\A_1}{\A_2}
  }
  {\eqsubst
    {\sbshift{\G_1}{\A_1}{\sbs}}
    {\sbshift{\G_2}{\A_2}{\sbt}}
    {\ctxextend{\G_1}{\subst{\A_1}{\sbs}}}
    {\ctxextend{\D}{\A_1}}
  }
}


\newcommand{\rlCongSubstComp}{\referTo{cong-subst-comp}{rul:cong-subst-comp}}
\newcommand{\showCongSubstComp}{%
  \infer[\rulename{cong-subst-comp}] % CongSubstComp
  {\eqsubst{\sbs_1}{\sbs_2}{\G}{\D} \\
   \eqsubst{\sbt_1}{\sbt_2}{\D}{\E}
  }
  {\eqsubst
    {\sbcomp{\sbs_1}{\sbt_1}}
    {\sbcomp{\sbs_2}{\sbt_2}}
    {\G}
    {\E}
  }
}


\newcommand{\rlEqSubstCtxConv}{\referTo{eq-subst-ctx-conv}{rul:eq-subst-ctx-conv}}
\newcommand{\showEqSubstCtxConv}{%
  \infer[\rulename{eq-subst-ctx-conv}] % EqSubstCtxConv
  {\eqsubst{\sbs}{\sbt}{\G_1}{\D_1} \\
   \eqctx{\G_1}{\G_2} \\
   \eqctx{\D_1}{\D_2}
  }
  {\eqsubst{\sbs}{\sbt}{\G_2}{\D_2}}
}


\newcommand{\rlCompAssoc}{\referTo{comp-assoc}{rul:comp-assoc}}
\newcommand{\showCompAssoc}{%
  \infer[\rulename{comp-assoc}] % CompAssoc
  {\issubst{\sbs}{\G}{\D} \\
   \issubst{\sbt}{\D}{\E} \\
   \issubst{\sbr}{\E}{\F}
  }
  {\eqsubst
    {\sbcomp{\sbcomp{\sbs}{\sbt}}{\sbr}}
    {\sbcomp{\sbs}{\sbcomp{\sbt}{\sbr}}}
    {\G}
    {\F}
  }
}


\newcommand{\rlWeakNat}{\referTo{weak-nat}{rul:weak-nat}}
\newcommand{\showWeakNat}{%
  \infer[\rulename{weak-nat}] % WeakNat
  {\issubst{\sbs}{\G}{\D} \\
   \istype{\D}{\A}
  }
  {\eqsubst
    {\sbcomp{\sbshift{\G}{\A}{\sbs}}
            {\sbweak{\D}{\A}}}
    {\sbcomp{\sbweak{\G}{\subst{\A}{\sbs}}}
            {\sbs}}
    {\ctxextend{\G}{\subst{\A}{\sbs}}}
    {\D}
  }
}


\newcommand{\rlWeakZero}{\referTo{weak-zero}{rul:weak-zero}}
\newcommand{\showWeakZero}{%
  \infer[\rulename{weak-zero}] % WeakZero
  {\isterm{\G}{\uu}{\A}}
  {\eqsubst
    {\sbcomp{\sbzero{\G}{\A}{\uu}}
            {\sbweak{\G}{\A}}}
    {\sbid{\G}}
    {\G}
    {\G}
  }
}


\newcommand{\rlShiftZero}{\referTo{shift-zero}{rul:shift-zero}}
\newcommand{\showShiftZero}{%
  \infer[\rulename{shift-zero}] % ShiftZero
  {\issubst{\sbs}{\G}{\D} \\
   \isterm{\D}{\uu}{\A}
  }
  {\eqsubst
    {\sbcomp{\sbzero{\G}{\subst{\A}{\sbs}}{\subst{\uu}{\sbs}}}
            {\sbshift{\G}{\A}{\sbs}}}
    {\sbcomp{\sbs}
            {\sbzero{\D}{\A}{\uu}}}
    {\G}
    {\ctxextend{\D}{\A}}
  }
}


\newcommand{\rlCompShift}{\referTo{comp-shift}{rul:comp-shift}}
\newcommand{\showCompShift}{%
  \infer[\rulename{comp-shift}] % CompShift
  {\issubst{\sbs}{\G}{\D} \\
   \issubst{\sbt}{\D}{\E} \\
   \istype{\E}{\A}
  }
  {\eqsubst
    {\sbcomp{\sbshift{\G}{\subst{\A}{\sbt}}{\sbs}}
            {\sbshift{\D}{\A}{\sbt}}}
    {\sbshift{\G}{\A}{\sbcomp{\sbs}{\sbt}}}
    {\ctxextend{\G}{\subst{\A}{\sbcomp{\sbs}{\sbt}}}}
    {\ctxextend{\E}{\A}}
  }
}


\newcommand{\rlCompIdRight}{\referTo{comp-id-right}{rul:comp-id-right}}
\newcommand{\showCompIdRight}{%
  \infer[\rulename{comp-id-right}] % CompIdRight
  {\issubst{\sbs}{\G}{\D}}
  {\eqsubst
    {\sbcomp{\sbid{\D}}{\sbs}}
    {\sbs}
    {\G}
    {\D}
  }
}


\newcommand{\rlCompIdLeft}{\referTo{comp-id-left}{rul:comp-id-left}}
\newcommand{\showCompIdLeft}{%
  \infer[\rulename{comp-id-left}] % CompIdLeft
  {\issubst{\sbs}{\G}{\D}}
  {\eqsubst
    {\sbcomp{\sbs}{\sbid{\G}}}
    {\sbs}
    {\G}
    {\D}
  }
}


%%%% Type equality


\newcommand{\rlEqTyCtxConv}{\referTo{eq-ty-ctx-conv}{rul:eq-ty-ctx-conv}}
\newcommand{\showEqTyCtxConv}{%
  \infer[\rulename{eq-ty-ctx-conv}] % EqTyCtxConv
  {\eqtype{\G}{\A}{\B}\\
    \eqctx{\G}{\D}}
  {\eqtype{\D}{\A}{\B}}
}


\newcommand{\rlEqTyRefl}{\referTo{eq-ty-refl}{rul:eq-ty-refl}}
\newcommand{\showEqTyRefl}{%
  \infer[\rulename{eq-ty-refl}] % EqTyRefl
  {\istype{\G}{\A}}
  {\eqtype{\G}{\A}{\A}}
}


\newcommand{\rlEqTySym}{\referTo{eq-ty-sym}{rul:eq-ty-sym}}
\newcommand{\showEqTySym}{%
  \infer[\rulename{eq-ty-sym}] % EqTySym
  {\eqtype{\G}{\A}{\B}}
  {\eqtype{\G}{\B}{\A}}
}


\newcommand{\rlEqTyTrans}{\referTo{eq-ty-trans}{rul:eq-ty-trans}}
\newcommand{\showEqTyTrans}{%
  \infer[\rulename{eq-ty-trans}] % EqTyTrans
  {\eqtype{\G}{\A}{\B}\\
    \eqtype{\G}{\B}{\C}}
  {\eqtype{\G}{\A}{\C}}
}


\newcommand{\rlEqTyIdSubst}{\referTo{eq-ty-id-subst}{rul:eq-ty-id-subst}}
\newcommand{\showEqTyIdSubst}{%
  \infer[\rulename{eq-ty-id-subst}] % EqTyIdSubst
  {\istype{\G}{\A}}
  {\eqtype
   {\G}
   {\subst{\A}{\sbid{\G}}}
   {\A}
  }
}


\newcommand{\rlEqTySubstComp}{\referTo{eq-ty-subst-comp}{rul:eq-ty-subst-comp}}
\newcommand{\showEqTySubstComp}{%
  \infer[\rulename{eq-ty-subst-comp}] % EqTySubstComp
  {\istype{\E}{\A} \\
   \issubst{\sbs}{\G}{\D} \\
   \issubst{\sbt}{\D}{\E}
  }
  {\eqtype{\G}
    {\subst{\subst{\A}{\sbt}}{\sbs}}
    {\subst{\A}{\sbcomp{\sbs}{\sbt}}}
  }
}


\newcommand{\rlEqTySubstProd}{\referTo{eq-ty-subst-prod}{rul:eq-ty-subst-prod}}
\newcommand{\showEqTySubstProd}{%
  \infer[\rulename{eq-ty-subst-prod}] % EqTySubstProd
  {\issubst{\sbs}{\G}{\D} \\
    \istype{\D}{\A} \\
    \istype{\ctxextend{\D}{\A}}{\B}
  }
  {\eqtype{\G}
    {\subst{(\Prod{\A}{\B})}{\sbs}}
    {\Prod
      {\subst{\A}{\sbs}}
      {\subst{\B}{\sbshift{\G}{\A}{\sbs}}}
    }
  }
}


\newcommand{\rlEqTySubstId}{\referTo{eq-ty-subst-id}{rul:eq-ty-subst-id}}
\newcommand{\showEqTySubstId}{%
  \infer[\rulename{eq-ty-subst-id}] % EqTySubstId
  {\issubst{\sbs}{\G}{\D} \\
    \istype{\D}{\A} \\
    \isterm{\D}{\uu}{\A} \\
    \isterm{\D}{\vv}{\A}
  }
  {\eqtype{\G}
    {\subst{(\Id{\A}{\uu}{\vv})}{\sbs}}
    {\Id{\subst{\A}{\sbs}}{\subst{\uu}{\sbs}}{\subst{\vv}{\sbs}}}
  }
}


\newcommand{\rlEqTySubstEmpty}{\referTo{eq-ty-subst-empty}{rul:eq-ty-subst-empty}}
\newcommand{\showEqTySubstEmpty}{%
  \infer[\rulename{eq-ty-subst-empty}] % EqTySubstEmpty
  {\issubst{\sbs}{\G}{\D}}
  {\eqtype{\G}{\subst{\Empty}{\sbs}}{\Empty}}
}


\newcommand{\rlEqTySubstUnit}{\referTo{eq-ty-subst-unit}{rul:eq-ty-subst-unit}}
\newcommand{\showEqTySubstUnit}{%
  \infer[\rulename{eq-ty-subst-unit}] % EqTySubstUnit
  {\issubst{\sbs}{\G}{\D}}
  {\eqtype{\G}{\subst{\Unit}{\sbs}}{\Unit}}
}


\newcommand{\rlEqTySubstBool}{\referTo{eq-ty-subst-bool}{rul:eq-ty-subst-bool}}
\newcommand{\showEqTySubstBool}{%
  \infer[\rulename{eq-ty-subst-bool}] % EqTySubstBool
  {\issubst{\sbs}{\G}{\D}}
  {\eqtype{\G}{\subst{\Bool}{\sbs}}{\Bool}}
}


\newcommand{\rlEqTyExfalso}{\referTo{eq-ty-exfalso}{rul:eq-ty-exfalso}}
\newcommand{\showEqTyExfalso}{%
  \infer[\rulename{eq-ty-exfalso}] % EqTyExfalso
  {\istype{\G}{\A} \\
    \istype{\G}{\B} \\
    \isterm{\G}{\uu}{\Empty}
  }
  {\eqtype{\G}{\A}{\B}
  }
}


\newcommand{\rlCongProd}{\referTo{cong-prod}{rul:cong-prod}}
\newcommand{\showCongProd}{%
  \infer[\rulename{cong-prod}] % CongProd
  {\eqtype{\G}{\A_1}{\B_1}\\
    \eqtype{\ctxextend{\G}{\A_1}}{\A_2}{\B_2}}
  {\eqtype{\G}{\Prod{\A_1}{\A_2}}{\Prod{\B_1}{\B_2}}}
}


\newcommand{\rlCongId}{\referTo{cong-id}{rul:cong-id}}
\newcommand{\showCongId}{%
  \infer[\rulename{cong-id}] % CongId
  {\eqtype{\G}{\A}{\B}\\
    \eqterm{\G}{\uu_1}{\vv_1}{\A}\\
    \eqterm{\G}{\uu_2}{\vv_2}{\A}
  }
  {\eqtype{\G}{\Id{\A}{\uu_1}{\uu_2}}
    {\Id{\B}{\vv_1}{\vv_2}}}
}


\newcommand{\rlCongTySubst}{\referTo{cong-ty-subst}{rul:cong-ty-subst}}
\newcommand{\showCongTySubst}{%
  \infer[\rulename{cong-ty-subst}] % CongTySubst
  {\eqsubst{\sbs}{\sbt}{\G}{\D} \\
    \eqtype{\D}{\A}{\B}
  }
  {\eqtype{\G}{\subst{\A}{\sbs}}{\subst{\B}{\sbt}}}
}


%%%% Term equality


\newcommand{\rlEqTyConv}{\referTo{eq-ty-conv}{rul:eq-ty-conv}}
\newcommand{\showEqTyConv}{%
  \infer[\rulename{eq-ty-conv}] % EqTyConv
  {\eqterm{\G}{\uu}{\vv}{\A}\\
    \eqtype{\G}{\A}{\B}}
  {\eqterm{\G}{\uu}{\vv}{\B}}
}


\newcommand{\rlEqCtxConv}{\referTo{eq-ctx-conv}{rul:eq-ctx-conv}}
\newcommand{\showEqCtxConv}{%
  \infer[\rulename{eq-ctx-conv}] % EqCtxConv
  {\eqterm{\G}{\uu}{\vv}{\A}\\
    \eqctx{\G}{\D}}
  {\eqterm{\D}{\uu}{\vv}{\A}}
}


\newcommand{\rlEqRefl}{\referTo{eq-refl}{rul:eq-refl}}
\newcommand{\showEqRefl}{%
  \infer[\rulename{eq-refl}] % EqRefl
  {\isterm{\G}{\uu}{\A}}
  {\eqterm{\G}{\uu}{\uu}{\A}}
}


\newcommand{\rlEqSym}{\referTo{eq-sym}{rul:eq-sym}}
\newcommand{\showEqSym}{%
  \infer[\rulename{eq-sym}] % EqSym
  {\eqterm{\G}{\vv}{\uu}{\A}}
  {\eqterm{\G}{\uu}{\vv}{\A}}
}


\newcommand{\rlEqTrans}{\referTo{eq-trans}{rul:eq-trans}}
\newcommand{\showEqTrans}{%
  \infer[\rulename{eq-trans}] % EqTrans
  {\eqterm{\G}{\uu}{\vv}{\A}\\
    \eqterm{\G}{\vv}{\ww}{\A}}
  {\eqterm{\G}{\uu}{\ww}{\A}}
}



\newcommand{\rlEqIdSubst}{\referTo{eq-id-subst}{rul:eq-id-subst}}
\newcommand{\showEqIdSubst}{%
  \infer[\rulename{eq-id-subst}] % EqIdSubst
  {\isterm{\G}{\uu}{\A}}
  {\eqterm{\G}
    {\subst{\uu}{\sbid{\G}}}
    {\uu}
    {\A}
  }
}


\newcommand{\rlEqSubstComp}{\referTo{eq-subst-comp}{rul:eq-subst-comp}}
\newcommand{\showEqSubstComp}{%
  \infer[\rulename{eq-subst-comp}] % EqSubstComp
  {\isterm{\E}{\uu}{\A} \\
   \issubst{\sbs}{\G}{\D} \\
   \issubst{\sbt}{\D}{\E}
  }
  {\eqterm{\G}
    {\subst{\subst{\uu}{\sbt}}{\sbs}}
    {\subst{\uu}{\sbcomp{\sbs}{\sbt}}}
    {\subst{\A}{\sbcomp{\sbs}{\sbt}}}
  }
}


\newcommand{\rlEqSubstWeak}{\referTo{eq-subst-weak}{rul:eq-subst-weak}}
\newcommand{\showEqSubstWeak}{%
  \infer[\rulename{eq-subst-weak}] % EqSubstWeak
  {\isterm{\G}{\var{k}}{\A} \\
    \istype{\G}{\B}
  }
  {\eqterm{\ctxextend{\G}{\B}}
    {\subst{\var{k}}{\sbweak{\G}{\B}}}
    {\var{k+1}}
    {\subst{\A}{\sbweak{\G}{\B}}}
  }
}


\newcommand{\rlEqSubstZeroZero}{\referTo{eq-subst-zero-zero}{rul:eq-subst-zero-zero}}
\newcommand{\showEqSubstZeroZero}{%
  \infer[\rulename{eq-subst-zero-zero}] % EqSubstZeroZero
  {
    \isterm{\G}{\uu}{\A}
  }
  {\eqterm{\G}
    {\subst{\var{0}}{\sbzero{\G}{\A}{\uu}}}
    {\uu}
    {\A}
  }
}

\newcommand{\rlEqSubstZeroSucc}{\referTo{eq-subst-zero-succ}{rul:eq-subst-zero-succ}}
\newcommand{\showEqSubstZeroSucc}{%
  \infer[\rulename{eq-subst-zero-succ}] % EqSubstZeroSucc
  {
    \isterm{\G}{\var{k}}{\A} \\
    \isterm{\G}{\uu}{\B}
  }
  {\eqterm{\G}
    {\subst
      {\var{k+1}}
      {\sbzero{\G}{\B}{\uu}}
    }
    {\var{k}}
    {\A}
  }
}

\newcommand{\rlEqSubstShiftZero}{\referTo{eq-subst-shift-zero}{rul:eq-subst-shift-zero}}
\newcommand{\showEqSubstShiftZero}{%
  \infer[\rulename{eq-subst-shift-zero}] % EqSubstShiftZero
  {\issubst{\sbs}{\G}{\D}\\
   \istype{\D}{\A}
  }
  {\eqterm
   {\ctxextend{\G}{\subst{\A}{\sbs}}}
   {\subst{\var{0}}{\sbshift{\G}{\A}{\sbs}}}
   {\var{0}}
   {\subst{(\subst{\A}{\sbs})}{\sbweak{\G}{\subst{\A}{\sbs}}}}
  }
}

\newcommand{\rlEqSubstShiftSucc}{\referTo{eq-subst-shift-succ}{rul:eq-subst-shift-succ}}
\newcommand{\showEqSubstShiftSucc}{%
  \infer[\rulename{eq-subst-shift-succ}] % EqSubstShiftSucc
  {\issubst{\sbs}{\G}{\D} \\
   \isterm{\D}{\var{k}}{\B} \\
   \istype{\D}{\A}
  }
  {\eqterm
   {\ctxextend{\G}{\subst{\A}{\sbs}}}
   {\subst{\var{k+1}}{\sbshift{\G}{\A}{\sbs}}}
   {\subst{(\subst{\var{k}}{\sbs})}{\sbweak{\G}{\subst{\A}{\sbs}}}}
   {\subst{\subst{\B}{\sbs}}{\sbweak{\G}{\subst{\A}{\sbs}}}}
  }
}



\newcommand{\rlEqSubstAbs}{\referTo{eq-subst-abs}{rul:eq-subst-abs}}
\newcommand{\showEqSubstAbs}{%
  \infer[\rulename{eq-subst-abs}] % EqSubstAbs
  {\issubst{\sbs}{\G}{\D} \\
   \istype{\D}{\A} \\
    \isterm{\ctxextend{\D}{\A}}{\uu}{\B}
  }
  {\eqterm{\G}
    {\subst{(\lam{\A}{\B} \uu)}{\sbs}}
    {(\lam
      {\subst{\A}{\sbs}}
      {\subst
        {\B}
        {\sbshift{\G}{\A}{\sbs}}
      }
      \subst{\uu}{\sbshift{\G}{\A}{\sbs}})
    }
    {\Prod
      {\subst{\A}{\sbs}}
      {\subst
        {\B}
        {\sbshift{\G}{\A}{\sbs}}
      }
    }
  }
}


\newcommand{\rlEqSubstApp}{\referTo{eq-subst-app}{rul:eq-subst-app}}
\newcommand{\showEqSubstApp}{%
  \infer[\rulename{eq-subst-app}] % EqSubstApp
  {\issubst{\sbs}{\G}{\D} \\
    \istype{\ctxextend{\D}{\A}}{\B} \\
    \isterm{\D}{\uu}{\Prod{\A}{\B}} \\
    \isterm{\D}{\vv}{\A}
  }
  {\eqterm{\G}
    {\subst{(\app{\uu}{\A}{\B}{\vv})}{\sbs}}
    {\app
      {\subst{\uu}{\sbs}}
      {\subst{\A}{\sbs}}
      {\subst
        {\B}
        {\sbshift{\G}{\A}{\sbs}}
      }
      {\subst{\vv}{\sbs}}}
    {\subst
      {(\subst{\B}{\sbzero{\D}{\A}{\vv}})}
      {\sbs}
    }
  }
}


\newcommand{\rlEqSubstRefl}{\referTo{eq-subst-refl}{rul:eq-subst-refl}}
\newcommand{\showEqSubstRefl}{%
  \infer[\rulename{eq-subst-refl}] % EqSubstRefl
  {\issubst{\sbs}{\G}{\D} \\
    \isterm{\D}{\uu}{\A}
  }
  {\eqterm{\G}
    {\subst{(\refl{\A}{\uu})}{\sbs}}
    {\refl{\subst{\A}{\sbs}}{\subst{\uu}{\sbs}}}
    {\Id{\subst{\A}{\sbs}}{\subst{\uu}{\sbs}}{\subst{\uu}{\sbs}}}
  }
}


\newcommand{\rlEqSubstJ}{\referTo{eq-subst-j}{rul:eq-subst-j}}
\newcommand{\showEqSubstJ}{%
  \infer[\rulename{eq-subst-j}] % EqSubstJ
  {\issubst{\sbs}{\G}{\D} \\
   \istype{\D}{\A} \\
   \isterm{\D}{\uu}{\A} \\
   \istype
    {\ctxextend
      {\ctxextend{\D}{\A}}
      {\Id
        {\subst{\A}{\sbweak{\D}{\A}}}
        {\subst{\uu}{\sbweak{\D}{\A}}}
        {\var{0}}}
    }
    {\C} \\
   \isterm
    {\D}
    {\ww}
    {\subst
      {\subst
        {\C}
        {\sbshift
          {\D}
          {\Id
            {\subst{\A}{\sbweak{\D}{\A}}}
            {\subst{\uu}{\sbweak{\D}{\A}}}
            {\var{0}}
          }
          {\sbzero{\D}{\A}{\uu}}
        }
      }
      {\sbzero{\D}{\Id{\A}{\uu}{\uu}}{\refl{\A} \uu}}
    } \\
   \isterm{\D}{\vv}{\A} \\
   \isterm{\D}{\p}{\Id{\A}{\uu}{\vv}}
  }
  {\eqterm{\G}
    {\subst
      {\J{\A}{\uu}{\C}{\ww}{\vv}{\p}}
      {\sbs}
    }
    {\J
      {\subst{\A}{\sbs}}
      {\subst{\uu}{\sbs}}
      {\subst{\C}{\sbt}}
      {\subst{\ww}{\sbs}}
      {\subst{\vv}{\sbs}}
      {\subst{\p}{\sbs}}
    }
    {\subst
      {\subst
        {\subst
          {\C}
          {\sbr}
        }
        {\sbzero{\G}{\Id{\A}{\uu}{\vv}}{\p}}
      }
      {\sbs}
    }
  }

  \text{Where } \sbt =
    \sbshift
      {\ctxextend{\G}{\subst{\A}{\sbs}}}
      {\Id
        {\subst{\A}{\sbweak{\D}{\A}}}
        {\subst{\uu}{\sbweak{\D}{\A}}}
        {\var{0}}
      }
      {(\sbshift{\G}{\A}{\sbs})}

  \text{and } \sbr =
    \sbshift
      {\D}
      {\Id
        {\subst{\A}{\sbweak{\D}{\A}}}
        {\subst{\uu}{\sbweak{\D}{\A}}}
        {\var{0}}
      }
      {\sbzero{\D}{\A}{\vv}}
}


\newcommand{\rlEqSubstExfalso}{\referTo{eq-subst-exfalso}{rul:eq-subst-exfalso}}
\newcommand{\showEqSubstExfalso}{%
  \infer[\rulename{eq-subst-exfalso}] % EqSubstExfalso
  {\issubst{\sbs}{\G}{\D} \\
   \istype{\D}{\A} \\
   \isterm{\D}{\uu}{\Empty}
  }
  {\eqterm{\G}
    {\subst{(\exfalso{\A} \uu)}{\sbs}}
    {\exfalso{\subst{\A}{\sbs}} \subst{\uu}{\sbs}}
    {\subst{\A}{\sbs}}
  }
}


\newcommand{\rlEqSubstUnit}{\referTo{eq-subst-unit}{rul:eq-subst-unit}}
\newcommand{\showEqSubstUnit}{%
  \infer[\rulename{eq-subst-unit}] % EqSubstUnit
  {\issubst{\sbs}{\G}{\D}}
  {\eqterm{\G}{\subst{\unit}{\sbs}}{\unit}{\Unit}}
}


\newcommand{\rlEqSubstTrue}{\referTo{eq-subst-true}{rul:eq-subst-true}}
\newcommand{\showEqSubstTrue}{%
  \infer[\rulename{eq-subst-true}] % EqSubstTrue
  {\issubst{\sbs}{\G}{\D}}
  {\eqterm{\G}{\subst{\true}{\sbs}}{\true}{\Bool}}
}


\newcommand{\rlEqSubstFalse}{\referTo{eq-subst-false}{rul:eq-subst-false}}
\newcommand{\showEqSubstFalse}{%
  \infer[\rulename{eq-subst-false}] % EqSubstFalse
  {\issubst{\sbs}{\G}{\D}}
  {\eqterm{\G}{\subst{\false}{\sbs}}{\false}{\Bool}}
}


\newcommand{\rlEqSubstCond}{\referTo{eq-subst-cond}{rul:eq-subst-cond}}
\newcommand{\showEqSubstCond}{%
  \infer[\rulename{eq-subst-cond}] % EqSubstCond
  {\issubst{\sbs}{\G}{\D} \\
   \isterm{\D}{\uu}{\Bool} \\
   \istype{\ctxextend{\D}{\Bool}}{\C} \\
   \isterm{\D}{\vv}{\subst{\C}{\sbzero{\D}{\Bool}{\true}}} \\
   \isterm{\D}{\ww}{\subst{\C}{\sbzero{\D}{\Bool}{\false}}}
  }
  {\eqterm{\G}
    {\subst{\cond{\C}{\uu}{\vv}{\ww}}{\sbs}}
    {\cond
      {\subst{\C}{\sbshift{\G}{\Bool}{\sbs}}}
      {\subst{\uu}{\sbs}}
      {\subst{\vv}{\sbs}}
      {\subst{\ww}{\sbs}}
    }
    {\subst{\subst{\C}{\sbzero{\D}{\Bool}{\uu}}}{\sbs}}
  }
}


\newcommand{\rlEqTermExfalso}{\referTo{eq-term-exfalso}{rul:eq-term-exfalso}}
\newcommand{\showEqTermExfalso}{%
  \infer[\rulename{eq-term-exfalso}] % EqTermExfalso
  {\istype{\G}{\A} \\
    \isterm{\G}{\uu}{\A} \\
    \isterm{\G}{\vv}{\A} \\
    \isterm{\G}{\ww}{\Empty}
  }
  {\eqterm{\G}{\uu}{\vv}{\A}
  }
}


\newcommand{\rlUnitEta}{\referTo{unit-eta}{rul:unit-eta}}
\newcommand{\showUnitEta}{%
  \infer[\rulename{unit-eta}] % UnitEta
  {\isterm{\G}{\uu}{\Unit} \\
   \isterm{\G}{\vv}{\Unit}
 }
 {\eqterm{\G}{\uu}{\vv}{\Unit}}
}


%


\newcommand{\rlEqReflection}{\referTo{eq-reflection}{rul:eq-reflection}}
\newcommand{\showEqReflection}{%
  \infer[\rulename{eq-reflection}] % EqReflection
  {\isterm{\G}{\ww_1}{\Id{\A}{\uu}{\vv}} \\
    \isterm{\G}{\ww_2}{\textsf{UIP}(\A)}
  }
  {\eqterm{\G}{\uu}{\vv}{\A}}
}


\newcommand{\rlProdBeta}{\referTo{prod-beta}{rul:prod-beta}}
\newcommand{\showProdBeta}{%
  \infer[\rulename{prod-beta}] % ProdBeta
  {\isterm{\ctxextend{\G}{\A}}{\uu}{\B}\\
    \isterm{\G}{\vv}{\A}}
  {\eqterm{\G}{\bigl(\app{(\lam{\A}{\B}{\uu})}{\A}{\B}{\vv}\bigr)}
    {\subst{\uu}{\sbzero{\G}{\A}{\vv}}}
    {\subst{\B}{\sbzero{\G}{\A}{\vv}}}}
}


\newcommand{\rlCondTrue}{\referTo{cond-true}{rul:cond-true}}
\newcommand{\showCondTrue}{%
  \infer[\rulename{cond-true}] % CondTrue
  {\istype{\ctxextend{\G}{\Bool}}{\C} \\
   \isterm{\G}{\vv}{\subst{\C}{\sbzero{\G}{\Bool}{\true}}} \\
   \isterm{\G}{\ww}{\subst{\C}{\sbzero{\G}{\Bool}{\false}}}
  }
  {\eqterm{\G}
    {\cond{\C}{\true}{\vv}{\ww}}
    {\vv}
    {\subst{\C}{\sbzero{\G}{\Bool}{\true}}}
  }
}

\newcommand{\rlCondFalse}{\referTo{cond-false}{rul:cond-false}}
\newcommand{\showCondFalse}{%
  \infer[\rulename{cond-false}] % CondFalse
  {\istype{\ctxextend{\G}{\Bool}}{\C} \\
   \isterm{\G}{\vv}{\subst{\C}{\sbzero{\G}{\Bool}{\true}}} \\
   \isterm{\G}{\ww}{\subst{\C}{\sbzero{\G}{\Bool}{\false}}}
  }
  {\eqterm{\G}
    {\cond{\C}{\false}{\vv}{\ww}}
    {\ww}
    {\subst{\C}{\sbzero{\G}{\Bool}{\false}}}
  }
}


\newcommand{\rlProdEta}{\referTo{prod-eta}{rul:prod-eta}}
\newcommand{\showProdEta}{%
  \infer[\rulename{prod-eta}] % ProdEta
  {\isterm{\G}{\uu}{\Prod{\A}{\B}}\\
    \isterm{\G}{\vv}{\Prod{\A}{\B}}\\\\
    \eqterm
    {\ctxextend{\G}{\A}}
    {(\app
      {\subst{\uu}{\sbweak{\G}{\A}}}
      {\subst{\A}{\sbweak{\G}{\A}}}
      {\subst{\B}{\sbt}}
      {\var{0}})
    }
    {(\app
      {\subst{\vv}{\sbweak{\G}{\A}}}
      {\subst{\A}{\sbweak{\G}{\A}}}
      {\subst{\B}{\sbt}}
      {\var{0}})
    }
    {\B}
  }
  {\eqterm{\G}{\uu}{\vv}{\Prod{\A}{\B}}}

  \text{Where } \sbt =
    \sbshift
      {\ctxextend{\G}{\A}}
      {\A}
      {\sbweak{\G}{\A}}
}


\newcommand{\rlJRefl}{\referTo{j-refl}{rul:j-refl}}
\newcommand{\showJRefl}{%
  \infer[\rulename{j-refl}] % JRefl
  {\istype{\G}{\A} \\
   \isterm{\G}{\uu}{\A} \\
   \istype
    {\ctxextend
      {\ctxextend{\G}{\A}}
      {\Id
        {\subst{\A}{\sbweak{\G}{\A}}}
        {\subst{\uu}{\sbweak{\G}{\A}}}
        {\var{0}}}
    }
    {\C} \\
   \isterm
    {\G}
    {\ww}
    {\subst
      {\subst
        {\C}
        {\sbt}
      }
      {\sbzero{\G}{\Id{\A}{\uu}{\uu}}{\refl{\A} \uu}}
    }
  }
  {\eqterm{\G}
    {\J{\A}{\uu}{\C}{\ww}{\uu}{\refl{\A} \uu}}
    {\ww}
    {\subst
      {\subst
        {\C}
        {\sbt}
      }
      {\sbzero{\G}{\Id{\A}{\uu}{\uu}}{\refl{\A} \uu}}
    }
  }

  \text{Where } \sbt =
    \sbshift
      {\G}
      {\Id
        {\subst{\A}{\sbweak{\G}{\A}}}
        {\subst{\uu}{\sbweak{\G}{\A}}}
        {\var{0}}
      }
      {\sbzero{\G}{\A}{\uu}}
}


\newcommand{\rlCongAbs}{\referTo{cong-abs}{rul:cong-abs}}
\newcommand{\showCongAbs}{%
  \infer[\rulename{cong-abs}] % CongAbs
  {\eqtype{\G}{\A_1}{\B_1}\\
    \eqtype{\ctxextend{\G}{\A_1}}{\A_2}{\B_2}\\
    \eqterm{\ctxextend{\G}{\A_1}}{\uu_1}{\uu_2}{\A_2}}
  {\eqterm{\G}{(\lam{\A_1}{\A_2}{\uu_1})}
    {(\lam{\B_1}{\B_2}{\uu_2})}
    {\Prod{\A_1}{\A_2}}}
}


\newcommand{\rlCongApp}{\referTo{cong-app}{rul:cong-app}}
\newcommand{\showCongApp}{%
  \infer[\rulename{cong-app}] % CongApp
  {\eqtype{\G}{\A_1}{\B_1}\\
    \eqtype{\ctxextend{\G}{\A_1}}{\A_2}{\B_2}\\\\
    \eqterm{\G}{\uu_1}{\vv_1}{\Prod{\A_1}{\A_2}}\\
    \eqterm{\G}{\uu_2}{\vv_2}{\A_1}}
  {\eqterm
    {\G}
    {(\app{\uu_1}{\A_1}{\A_2}{\uu_2})}
    {(\app{\vv_1}{\B_1}{\B_2}{\vv_2})}
    {\subst{\A_2}{\sbzero{\G}{\A_1}{\uu_2}}}
  }
}


\newcommand{\rlCongRefl}{\referTo{cong-refl}{rul:cong-refl}}
\newcommand{\showCongRefl}{%
  \infer[\rulename{cong-refl}] % CongRefl
  {\eqterm{\G}{\uu_1}{\uu_2}{\A_1}\\
    \eqtype{\G}{\A_1}{\A_2}}
  {\eqterm{\G}{\refl{\A_1} \uu_1}{\refl{\A_2} \uu_2}{\Id{\A_1}{\uu_1}{\uu_1}}}
}


\newcommand{\rlCongJ}{\referTo{cong-j}{rul:cong-j}}
\newcommand{\showCongJ}{%
  \infer[\rulename{cong-j}] % CongJ
  {\eqtype{\G}{\A_1}{\A_2} \\
   \eqterm{\G}{\uu_1}{\uu_2}{\A_1} \\
   \eqtype
    {\ctxextend
      {\ctxextend{\G}{\A_1}}
      {\Id
        {\subst{\A_1}{\sbweak{\G}{\A_1}}}
        {\subst{\uu}{\sbweak{\G}{\A_1}}}
        {\var{0}}}
    }
    {\C_1}
    {\C_2} \\
   \eqterm
    {\G}
    {\ww_1}
    {\ww_2}
    {\subst
      {\subst
        {\C_1}
        {\sbshift
          {\G}
          {\Id
            {\subst{\A_1}{\sbweak{\G}{\A_1}}}
            {\subst{\uu_1}{\sbweak{\G}{\A_1}}}
            {\var{0}}
          }
          {\sbzero{\G}{\A_1}{\uu_1}}
        }
      }
      {\sbzero{\G}{\Id{\A_1}{\uu_1}{\uu_1}}{\refl{\A_1} \uu_1}}
    } \\
   \eqterm{\G}{\vv_1}{\vv_2}{\A_1} \\
   \eqterm{\G}{\p_1}{\p_2}{\Id{\A_1}{\uu_1}{\vv_1}}
  }
  {\eqterm{\G}
    {\J{\A_1}{\uu_1}{\C_1}{\ww_1}{\vv_1}{\p_1}}
    {\J{\A_2}{\uu_2}{\C_2}{\ww_2}{\vv_2}{\p_2}}
    {\subst
      {\subst
        {\C_1}
        {\sbr}
      }
      {\sbzero{\G}{\Id{\A_1}{\uu_1}{\vv_1}}{\p_1}}
    }
  }

  \text{Where } \sbr =
    \sbshift
      {\G}
      {\Id
        {\subst{\A_1}{\sbweak{\G}{\A_1}}}
        {\subst{\uu_1}{\sbweak{\G}{\A_1}}}
        {\var{0}}
      }
      {\sbzero{\G}{\A_1}{\vv_1}}
}


\newcommand{\rlCongCond}{\referTo{cong-cond}{rul:cong-cond}}
\newcommand{\showCongCond}{%
  \infer[\rulename{cong-cond}] % CongCond
  {\eqterm{\G}{\uu_1}{\uu_2}{\Bool} \\
   \eqtype{\ctxextend{\G}{\Bool}}{\C_1}{\C_2} \\
   \eqterm{\G}{\vv_1}{\vv_2}{\subst{\C_1}{\sbzero{\G}{\Bool}{\true}}} \\
   \eqterm{\G}{\ww_1}{\ww_2}{\subst{\C_1}{\sbzero{\G}{\Bool}{\false}}}
  }
  {\eqterm{\G}
    {\cond{\C_1}{\uu_1}{\vv_1}{\ww_1}}
    {\cond{\C_2}{\uu_2}{\vv_2}{\ww_2}}
    {\subst{\C_1}{\sbzero{\G}{\Bool}{\uu_1}}}}
}


\newcommand{\rlCongTermSubst}{\referTo{cong-term-subst}{rul:cong-term-subst}}
\newcommand{\showCongTermSubst}{%
  \infer[\rulename{cong-term-subst}] % CongTermSubst
  {\eqsubst{\sbs}{\sbt}{\G}{\D} \\
    \eqterm{\D}{\uu_1}{\uu_2}{\A}
  }
  {\eqterm{\G}{\subst{\uu_1}{\sbs}}{\subst{\uu_2}{\sbt}}{\subst{\A}{\sbs}}}
}

%%% Local Variables:
%%% mode: latex
%%% TeX-master: "main"
%%% End:
 % inference rules as macros

\newtheorem{theorem}{Theorem}[section]
\newtheorem{problem}[theorem]{Problem}

\newenvironment{construction}{\begin{proof}[Construction]}{\end{proof}}

\begin{document}

\title{Elimination of equality reflection}

\author{Andrej Bauer}
\address{Andrej Bauer\\University of Ljubljana\\Slovenia}
\email{Andrej.Bauer@andrej.com}

\author{Philipp G.~Haselwarter}
\address{Philipp G.~Haselwarter\\University of Ljubljana\\Slovenia}
\email{philipp@haselwarter.org}

\author{Théo Winterhalter}
\address{Théo Winterhalter\\ENS Cachan, Université Paris-Saclay\\France}
\email{theo.winterhalter@ens-cachan.fr}

\begin{abstract}
  We give a translation of type theory with equality reflection to intensional type theory.
\end{abstract}

\maketitle

\section{Type theory}
\label{sec:type-theory}

In this section we give the formulation of type theory that we shall work with.

\subsection{Syntax}
\label{sec:syntax}

\begin{align*}
  \text{Context $\G$, $\D$}
    \bnf   {}& \ctxempty                && \text{empty context} \\
    \bnfor {}& \ctxextend{\G}{\A}       && \text{context $\G$ extended with $\A$} \\
  \\
  \text{Type $\A$, $\B$, $\C$}
    \bnf   {}& \Prod{\A} \B             && \text{product}\\
    \bnfor {}& \Id{\A}{\uu}{\vv}        && \text{identity type} \\
    \bnfor {}& \subst{\A}{\sbs}         && \text{substitution $\sbs$ applied to $\A$} \\
  \\
  \text{Term $\uu$, $\vv$, $\ww$}
    \bnf   {}& \var{k}                  && \text{the $k$-th variable index (a la de Bruijn)} \\
    \bnfor {}& \lam{\A}{\B} \uu         && \text{$\lambda$-abstraction} \\
    \bnfor {}& \app{\uu}{\A}{\B}{\vv}   && \text{application} \\
    \bnfor {}& \refl{\A} \uu            && \text{reflexivity} \\
    \bnfor {}& \subst{\uu}{\sbs}        && \text{substitution $\sbs$ applied to $\uu$} \\
  \\
  \text{Substitution $\sbs$, $\sbt$}
    \bnf   {}& \sbid{\G}                && \text{identity substitution $\G \to \G$} \\
    \bnfor {}& \sbcomp{\sbs}{\sbt}      && \text{composition of substitutions} \\
    \bnfor {}& \sbextend{\sbs}{\A}{\uu} && \text{substitution $\sbs$ extended with $\uu$ of type $\subst{\A}{\sbs}$} \\
    \bnfor {}& \sbweak{\G}{\A}          && \text{weakening substitution $\ctxextend{\G}{\A} \to \G$}
\end{align*}

\subsection{Judgments}
\label{sec:judgments}

\begin{align*}
& \isctx{\G}                    && \text{$\G$ is a context} \\
& \issubst{\sbs}{\G}{\D}        && \text{$\sbs$ is a substitution from context $\G$ to context $\D$} \\
& \istype{\G}{\A}               && \text{$\A$ is a type in context $\G$} \\
& \isterm{\G}{\uu}{\A}          && \text{$\uu$ is a term of type $\A$ in context $\G$} \\
& \eqctx{\G}{\D}                && \text{$\G$ and $\D$ are equal contexts} \\
& \eqtype{\G}{\A}{\B}           && \text{$\A$ and $\B$ are equal types in context $\G$} \\
& \eqterm{\G}{\uu}{\vv}{\A}     && \text{$\uu$ and $\vv$ are equal terms of type $\A$ in context $\G$}
\end{align*}

\subsection{Contexts \fbox{$\isctx{\G}$}}
\label{sec:contexts}

\begin{mathpar}
  \infer[\rl{ctx-empty}]
  { }
  {\isctx{\ctxempty}}

  \infer[\rl{ctx-extend}]
  {\istype{\G}{\A}
  }
  {\isctx{(\ctxextend{\G}{\A})}}
\end{mathpar}

\subsection{Substitutions \fbox{$\issubst{\sbs}{\G}{\D}$}}
\label{sec:subst}

\begin{mathpar}
  \infer[\rl{subst-id}]
  {\isctx{\G}}
  {\issubst{\sbid{\G}}{\G}{\G}}

  \infer[\rl{subst-compose}]
  {\issubst{\sbs}{\G}{\D} \\
   \issubst{\sbt}{\D}{\E}
  }
  {\issubst{\sbcomp{\sbt}{\sbs}}{\G}{\E}}

  \infer[\rl{subst-extend}]
  {\issubst{\sbs}{\G}{\D} \\
   \istype{\D}{\A} \\
   \isterm{\G}{\uu}{\subst{\A}{\sbs}}
  }
  {\issubst
     {(\sbextend{\sbs}{\A}{\uu})}
     {\G}
     {(\ctxextend{\D}{\A})}
  }

  \infer[\rl{subst-weak}]
  {\istype{\G}{\A}}
  {\issubst
     {\sbweak{\G}{\A}}
     {\ctxextend{\G}{\A}}
     {\G}
  }
\end{mathpar}

We let $\sbshift{\sbs}{\G}{\A}$ stand for
$\sbextend{\sbcomp{\sbs}{\sbweak{\G}{\A}}}{\A}{\var{0}}$. Then we have the derivable rule
%
\begin{equation*}
  \infer
  {\issubst{\sbs}{\G}{\D} \\
   \istype{\D}{\A}
  }
  {\issubst
      {(\sbshift{\sbs}{\G}{\A})}
      {\ctxextend{\G}{\subst{\A}{\sbs}}}
      {\ctxextend{\D}{\A}}
  }
\end{equation*}

\subsection{Types \fbox{$\istype{\G}{\A}$}}

\subsubsection*{General rules}

\begin{mathpar}
  \infer[\rl{ty-ctx-conv}]
  {\istype{\G}{\A} \\
    \eqctx{\G}{\D}
  }
  {\istype{\D}{\A}}

  \infer[\rl{ty-subst}]
  {\issubst{\sbs}{\G}{\D} \\
   \istype{\D}{\A}
  }
  {\istype{\G}{\subst{\A}{\sbs}}}
\end{mathpar}

\subsubsection*{Type formers}

\begin{mathpar}
  \infer[\rl{ty-prod}]
  {\istype{\G}{\A} \\
   \istype{\ctxextend{\G}{\A}}{\B}
  }
  {\istype{\G}{\Prod{\A}{\B}}}

  \infer[\rl{ty-id}]
  {\istype{\G}{\A}\\
   \isterm{\G}{\uu}{\A}\\
   \isterm{\G}{\vv}{\A}
  }
  {\istype{\G}{\Id{\A}{\uu}{\vv}}}
\end{mathpar}

\subsection{Terms \fbox{$\isterm{\G}{\uu}{\A}$}}

\subsubsection*{General rules}
\begin{mathpar}
  \infer[\rl{term-ty-conv}]
  {\isterm{\G}{\uu}{\A} \\
   \eqtype{\G}{\A}{\B}
  }
  {\isterm{\G}{\uu}{\B}}

  \infer[\rl{term-ctx-conv}]
  {\isterm{\G}{\uu}{\A} \\
   \eqctx{\G}{\D}
  }
  {\isterm{\D}{\uu}{\A}}

  \infer[\rl{term-subst}]
  {\issubst{\sbs}{\G}{\D} \\
   \isterm{\D}{\uu}{\A}
  }
  {\isterm{\G}{\subst{\uu}{\sbs}}{\subst{\A}{\sbs}}}
\end{mathpar}

\subsubsection*{Variables}

\begin{mathpar}
  \infer[\rl{term-var}]
  {\isctx{\G} \\
   \istype{\G}{\A}
  }
  {\isterm
     {\ctxextend{\G}{\A}}
     {\var{0}}
     {\subst{\A}{\sbweak{\G}{\A}}}
  }

  \infer[\rl{term-var-skip}]
  {\isterm{\G}{\var{k}}{\A} \\
   \istype{\G}{\B}
  }
  {\isterm
     {\ctxextend{\G}{\B}}
     {\var{k+1}}
     {\subst{\A}{\sbweak{\G}{\B}}}
  }
  \end{mathpar}

\subsubsection*{Abstraction and application}

\begin{mathpar}
  % Remark: we want \istype{\G}{\A} as a premise because in order to form the
  % product type we need it, and we do not want to resort to inversion on the
  % extended context.
  \infer[\rl{term-abs}]
  { \istype{\G}{\A} \\
    \isterm{\ctxextend{\G}{\A}}{\uu}{\B}
  }
  {\isterm{\G}{(\lam{\A}{\B}{\uu})}{\Prod{\A}{\B}}}

  % Remark: we want \istype{\ctxextend{\G}{\A}}{\B} because we need to
  % know that \B is a type in \ctxextend{\G}{\A} and we want to avoid
  % inversion on ty-prod.
  \infer[\rl{term-app}]
  {\istype{\ctxextend{\G}{\A}}{\B} \\
   \isterm{\G}{\uu}{\Prod{\A} \B} \\
   \isterm{\G}{\vv}{\A}
  }
  {\isterm
     {\G}
     {\app{\uu}{\A}{\B}{\vv}}
     {\subst{\B}{\sbextend{\sbid{\G}}{\A}{\vv}}}
  }
\end{mathpar}


\subsubsection*{Reflexivity}

\begin{mathpar}
  \infer[\rl{term-refl}]
  {\isterm{\G}{\uu}{\A}}
  {\isterm{\G}{\refl{\A} \uu}{\Id{\A}{\uu}{\uu}}}
\end{mathpar}

\subsection{Context equality \fbox{$\eqctx{\G}{\D}$}}
\label{sec:cont-equal}

\begin{mathpar}
  \infer[\rl{eq-ctx-empty}]
  { }
  {\eqctx{\ctxempty}{\ctxempty}}

  \infer[\rl{eq-ctx-extend}]
  {\eqctx{\G}{\D} \\
   \eqtype{\G}{\A}{\B}
  }
  {\eqctx{(\ctxextend{\G}{\A})}{(\ctxextend{\D}{\B})}}

\end{mathpar}

\goodbreak

\subsection{Type equality \fbox{$\eqtype{\G}{\A}{\B}$}}
\label{sec:type-equality}

\subsubsection*{General rules}

\begin{mathpar}
  \infer[\rl{eq-ty-conv}]
  {\eqtype{\G}{\A}{\B}\\
    \eqctx{\G}{\D}}
  {\eqtype{\D}{\A}{\B}}

  \infer[\rl{eq-ty-refl}]
  {\istype{\G}{\A}}
  {\eqtype{\G}{\A}{\A}}

  \infer[\rl{eq-ty-sym}]
  {\eqtype{\G}{\B}{\A}}
  {\eqtype{\G}{\A}{\B}}

  \infer[\rl{eq-ty-trans}]
  {\eqtype{\G}{\A}{\B}\\
   \eqtype{\G}{\B}{\C}}
  {\eqtype{\G}{\A}{\C}}
\end{mathpar}

\subsubsection*{Substitution}

\begin{mathpar}
  \infer[\rl{eq-ty-subst-id}]
  {\istype{\G}{\A}}
  {\eqtype{\G}
     {\subst{\A}{\sbid{\G}}}
     {\A}
  }

  \infer[\rl{eq-ty-subst-compose}]
  {\issubst{\sbs}{\G}{\D} \\
   \issubst{\sbt}{\D}{\E} \\
   \istype{\E}{\A}
  }
  {\eqtype{\G}
    {\subst{\A}{\sbcomp{\sbs}{\sbt}}}
    {\subst{(\subst{\A}{\sbs})}{\sbt}}
  }

  \infer[\rl{eq-ty-subst-prod}]
  {\issubst{\sbs}{\G}{\D} \\
   \istype{\D}{\A} \\
   \istype{\ctxextend{\D}{\A}}{\B}
  }
  {\eqtype{\G}
   {\subst{(\Prod{\A}{\B})}{\sbs}}
   {\Prod
     {\subst{\A}{\sbs}}
     {\subst{\B}{\sbshift{\sbs}{\G}{\A}}}
   }
  }

  \infer[\rl{eq-ty-subst-id}]
  {\issubst{\sbs}{\G}{\D} \\
   \istype{\D}{\A} \\
   \isterm{\D}{\uu}{\A} \\
   \isterm{\D}{\vv}{\A}
  }
  {\eqtype{\G}
   {\subst{(\Id{\A}{\uu}{\vv})}{\sbs}}
   {\Id{\subst{\A}{\sbs}}{\subst{\uu}{\sbs}}{\subst{\vv}{\sbs}}}
  }
\end{mathpar}

\subsubsection*{Congruence rules}

\begin{mathpar}
  \infer[\rl{cong-prod}]
  {\eqtype{\G}{\A_1}{\B_1}\\
   \eqtype{\ctxextend{\G}{\A_1}}{\A_2}{\B_2}}
  {\eqtype{\G}{\Prod{\A_1}{\A_2}}{\Prod{\B_1}{\B_2}}}

  \infer[\rl{cong-id}]
  {\eqtype{\G}{\A}{\B}\\
   \eqterm{\G}{\uu_1}{\vv_1}{\A}\\
   \eqterm{\G}{\uu_2}{\vv_2}{\A}
  }
  {\eqtype{\G}{\Id{\A}{\uu_1}{\uu_2}}
              {\Id{\B}{\vv_1}{\vv_2}}}

  \infer[\rl{cong-ty-subst}]
  {\issubst{\sbs}{\G}{\D} \\
   \eqtype{\D}{\A}{\B}
  }
  {\eqtype{\G}{\subst{\A}{\sbs}}{\subst{\B}{\sbs}}}
\end{mathpar}

\goodbreak

\subsection{Term equality \fbox{$\eqterm{\G}{\uu_1}{\uu_2}{\A}$}}

\subsubsection*{General rules}

\begin{mathpar}
  \infer[\rl{eq-ty-conv}]
  {\eqterm{\G}{\uu}{\vv}{\A}\\
    \eqtype{\G}{\A}{\B}}
  {\eqterm{\G}{\uu}{\vv}{\B}}

  \infer[\rl{eq-ctx-conv}]
  {\eqterm{\G}{\uu}{\vv}{\A}\\
    \eqctx{\G}{\D}}
  {\eqterm{\D}{\uu}{\vv}{\A}}

  \infer[\rl{eq-refl}]
  {\isterm{\G}{\uu}{\A}}
  {\eqterm{\G}{\uu}{\uu}{\A}}

  \infer[\rl{eq-sym}]
  {\eqterm{\G}{\vv}{\uu}{\A}}
  {\eqterm{\G}{\uu}{\vv}{\A}}

  \infer[\rl{eq-trans}]
  {\eqterm{\G}{\uu}{\vv}{\A}\\
   \eqterm{\G}{\vv}{\ww}{\A}}
  {\eqterm{\G}{\uu}{\ww}{\A}}
\end{mathpar}

\subsubsection*{Substitutions}

\begin{mathpar}
  \infer[\rl{eq-subst-id}]
  {\isterm{\G}{\uu}{\A}}
  {\eqterm{\G}
     {\subst{\uu}{\sbid{\G}}}
     {\uu}
     {\A}
  }

  \infer[\rl{eq-subst-compose}]
  {\issubst{\sbs}{\G}{\D} \\
   \issubst{\sbt}{\D}{\E} \\
   \isterm{\E}{\uu}{\A}
  }
  {\eqterm{\G}
    {\subst{\uu}{\sbcomp{\sbs}{\sbt}}}
    {\subst{(\subst{\uu}{\sbs})}{\sbt}}
    {\subst{\A}{\sbcomp{\sbs}{\sbt}}}
  }

  \infer[\rl{eq-subst-weak}]
  {\isterm{\G}{\var{k}}{\A} \\
   \istype{\G}{\B}   
  }
  {\eqterm{\ctxextend{\G}{\B}}
   {\subst{\var{k}}{\sbweak{\G}{\B}}}
   {\var{k+1}}
   {\subst{\A}{\sbweak{\G}{\B}}}
  }

  \infer[\rl{eq-subst-extend-zero}]
  {\issubst{\sbs}{\G}{\D} \\
   \istype{\D}{\A} \\
   \isterm{\G}{\uu}{\subst{\A}{\sbs}}
  }
  {\eqterm{\G}
     {\subst{\var{0}}{\sbextend{\sbs}{\A}{\uu}}}
     {\uu}
     {\subst{\A}{\sbs}}
  }

  \infer[\rl{eq-subst-extend-succ}]
  {\issubst{\sbs}{\G}{\D} \\
   \isterm{\D}{\var{k}}{\A} \\
   \isterm{\G}{\uu}{\subst{\B}{\sbs}}
  }
  {\eqterm{\G}
     {\subst
        {\var{k+1}}
        {\sbextend{\sbs}{\B}{\uu}}
     }
     {\subst{\var{k}}{\sbs}}
     {\subst{\A}{\sbs}}
  }

  \infer[\rl{eq-subst-abs}]
  {\issubst{\sbs}{\G}{\D} \\
   \isterm{\ctxextend{\D}{\A}}{\uu}{\B}
  }
  {\eqterm{\G}
    {\subst{(\lam{\A}{\B} \uu)}{\sbs}}
    {(\lam
      {\subst{\A}{\sbs}}
      {\subst
        {\B}
        {\sbshift{\sbs}{\G}{\subst{\A}{\sbs}}}
      }
      \subst{\uu}{\sbextend{\sbs}{\A}{\var{0}}})
    }
    {\Prod
      {\subst{\A}{\sbs}}
      {\subst
        {\B}
        {\sbshift{\sbs}{\G}{\subst{\A}{\sbs}}}
      }
    }
  }

  \infer[\rl{eq-subst-app}]
  {\issubst{\sbs}{\G}{\D} \\
   \istype{\ctxextend{\D}{\A}}{\B} \\
   \isterm{\D}{\uu}{\Prod{\A}{\B}} \\
   \isterm{\D}{\vv}{\A}
  }
  {\eqterm{\G}
   {\subst{(\app{\uu}{\A}{\B}{\vv})}{\sbs}}
   {\app
      {\subst{\uu}{\sbs}}
      {\subst{\A}{\sbs}}
      {\subst
        {\B}
        {\sbshift{\sbs}{\G}{\subst{\A}{\sbs}}}
      }
      {\subst{\vv}{\sbs}}}
   {\subst
     {(\subst{\B}{\sbextend{\sbid{\G}}{\A}{\vv}})}
     {\sbs}
   }
  }

  \infer[\rl{eq-subst-refl}]
  {\issubst{\sbs}{\G}{\D} \\
   \isterm{\D}{\uu}{\A}
  }
  {\eqterm{\G}
   {\subst{(\refl{\A}{\uu})}{\sbs}}
   {\refl{\subst{\A}{\sbs}}{\subst{\uu}{\sbs}}}
   {\Id{\subst{\A}{\sbs}}{\subst{\uu}{\sbs}}{\subst{\uu}{\sbs}}}
  }
\end{mathpar}

\subsubsection*{Equality reflection}
%
\begin{mathpar}
  \infer[\rl{eq-reflection}]
  {\isterm{\G}{\ww_1}{\Id{\A}{\uu}{\vv}} \\
   \isterm{\G}{\ww_2}{\textsf{UIP}(\A)}
  }
  {\eqterm{\G}{\uu}{\vv}{\A}}
\end{mathpar}
%
Here $\mathsf{UIP}(\A)$ is an abbreviation for what is written traditionally as
``all parallel paths in $\A$ are (propositionally) equal''.

\subsubsection*{Computation and Extensionality}

\begin{mathpar}
\infer[\rl{prod-beta}]
  {\isterm{\ctxextend{\G}{\A}}{\uu}{\B}\\
    \isterm{\G}{\vv}{\A}}
  {\eqterm{\G}{\bigl(\app{(\lam{\A}{\B}{\uu})}{\A}{\B}{\vv}\bigr)}
              {\subst{\uu}{\sbextend{\sbid{\G}}{\A}{\vv}}}
              {\subst{\B}{\sbextend{\sbid{\G}}{\A}{\vv}}}}

  % \infer[\rl{uip}]
  % {\isterm{\G}{\vv_1}{\Id{\A}{\uu}{\vv}} \\
  %   \isterm{\G}{\vv_2}{\Id{\A}{\uu}{\vv}}
  % }
  % {\eqterm{\G}{\vv_1}{e'_2}{\Id{\A}{\uu}{\vv}}}

  \infer[\rl{prod-eta}]
  {\isterm{\G}{\uu}{\Prod{\A}{\B}}\\
   \isterm{\G}{\vv}{\Prod{\A}{\B}}\\\\
   \eqterm
      {\ctxextend{\G}{\A}}
      {(\app
          {\subst{\uu}{\sbweak{\G}{\A}}}
          {\subst{\A}{\sbweak{\G}{\A}}}
          {\subst
            {\B}
            {\sbshift
              {\sbweak{\G}{\A}}
              {\ctxextend{\G}{\A}}
              {\A}
            }
          }
          {\var{0}})
      }
      {(\app
          {\subst{\vv}{\sbweak{\G}{\A}}}
          {\subst{\A}{\sbweak{\G}{\A}}}
          {\subst
            {\B}
            {\sbshift
              {\sbweak{\G}{\A}}
              {\ctxextend{\G}{\A}}
              {\A}
            }
          }
          {\var{0}})
      }
      {\B}
  }
  {\eqterm{\G}{\uu}{\vv}{\Prod{\A}{\B}}}
\end{mathpar}

The rule $\rl{prod-eta}$ is optional. We may prefer not to keep it around, as it amounts
to function extensionality.

\subsubsection*{Congruence rules}

\begin{mathpar}

  \infer[\rl{cong-abs}]
  {\eqtype{\G}{\A_1}{\B_1}\\
    \eqtype{\ctxextend{\G}{\A_1}}{\A_2}{\B_2}\\
    \eqterm{\ctxextend{\G}{\A_1}}{\uu_1}{\uu_2}{\A_2}}
  {\eqterm{\G}{(\lam{\A_1}{\A_2}{\uu_1})}
              {(\lam{\B_1}{\B_2}{\uu_2})}
              {\Prod{\A_1}{\A_2}}}

  \infer[\rl{cong-app}]
  {\eqtype{\G}{\A_1}{\B_1}\\
   \eqtype{\ctxextend{\G}{\A_1}}{\A_2}{\B_2}\\\\
   \eqterm{\G}{\uu_1}{\vv_1}{\Prod{\A_1}{\A_2}}\\
   \eqterm{\G}{\uu_2}{\vv_2}{\A_1}}
  {\eqterm
    {\G}
    {(\app{\uu_1}{\A_1}{\A_2}{\uu_2})}
    {(\app{\vv_1}{\B_1}{\B_2}{\vv_2})}
    {\subst{\A_2}{\sbextend{\sbid{\G}}{\A_1}{\uu_2}}}
  }

  \infer[\rl{cong-refl}]
  {\eqterm{\G}{\uu_1}{\uu_2}{\A_1}\\
    \eqtype{\G}{\A_1}{\A_2}}
  {\eqterm{\G}{\refl{\A_1} \uu_1}{\refl{\A_2} \uu_2}{\Id{\A_1}{\uu_1}{\uu_1}}}


  \infer[\rl{cong-term-subst}]
  {\issubst{\sbs}{\G}{\D} \\
   \eqterm{\D}{\uu_1}{\uu_2}{\A}
  }
  {\eqterm{\G}{\subst{\uu_1}{\sbs}}{\subst{\uu_2}{\sbs}}{\subst{\A}{\sbs}}}

\end{mathpar}


%%% Local Variables:
%%% mode: latex
%%% TeX-master: "main"
%%% End:

% \section{Type theory is well formed}
\label{sec:type-theory-well}

\begin{theorem}
  Type theory is well formed in the following sense:
  %
  \begin{enumerate}
  \item \label{sane-issubst} if $\issubst{\sbs}{\G}{\D}$ then $\isctx{\G}$ and $\isctx{\D}$,
  \item \label{sane-istype}  if $\istype{\G}{\A}$ then $\isctx{\G}$,
  \item \label{sane-isterm}  if $\isterm{\G}{\uu}{\A}$ then $\istype{\G}{\A}$,
  \item \label{sane-eqctx}   if $\eqctx{\G}{\D}$ then $\isctx{\G}$ and $\isctx{\D}$,
  \item \label{sane-eqtype}  if $\eqtype{\G}{\A}{\B}$ then $\istype{\G}{\A}$ and $\istype{\G}{\B}$,
  \item \label{sane-eqterm}  if $\eqterm{\G}{\uu}{\vv}{\A}$ then $\isterm{\G}{\uu}{\A}$ and $\isterm{\G}{\vv}{\A}$.
  \end{enumerate}
\end{theorem}

The rest of this section contains the proof, which proceeds by structural induction on the
derivation.

\subsection{Contexts \fbox{$\isctx{\G}$}}

The rules {\rlCtxEmpty} and {\rlCtxExtend} need not be considered.

\subsection{Substitutions \fbox{$\issubst{\sbs}{\G}{\D}$}}

This section proves clause \eqref{sane-issubst} of the theorem.

\subsubsection*{Rule {\rlSubstId}}

Consider a derivation ending with
%
\begin{equation*}
  \showSubstId
\end{equation*}
%
By the premise we have $\isctx{\G}$.

\subsubsection*{Rule {\rlSubstCompose}}

Consider a derivation ending with
%
\begin{equation*}
  \showSubstCompose
\end{equation*}
%
By induction hypothesis on the left premise $\isctx{\G}$ follows, and by induction
hypothesis on the right premise $\isctx{\E}$.

\subsubsection*{Rule {\rlSubstExtend}}

Consider a derivation ending with
%
\begin{equation*}
  \showSubstExtend
\end{equation*}
%
Then $\isctx{\G}$ by induction hypothesis on the left premise, while
$\isctx{(\ctxextend{\D}{\A})}$ follows by the induction hypothesis on the middle
premise and an application of {\rlCtxExtend}.

\subsubsection*{Rule {\rlSubstWeak}}

Consider a derivation ending with
%
\begin{equation*}
  \showSubstWeak
\end{equation*}
%
First, $\isctx{\G}$ follows by induction hypothesis on the premise, and then
$\isctx{\ctxextend{\G}{\A}}$ by an application of {\rlCtxExtend}.

\subsection{Types \fbox{$\istype{\G}{\A}$}}

In this section we prove clause \eqref{sane-istype}.

\subsubsection*{Rule {\rlTyCtxConv}}

Consider a derivation ending with
%
\begin{equation*}
  \showTyCtxConv
\end{equation*}
%
By induction hypothesis on the right premise we obtain $\isctx{\D}$.

\subsubsection*{Rule {\rlTySubst}}

Consider a derivation ending with
%
\begin{equation*}
  \showTySubst
\end{equation*}
%
By induction hypothesis on the left premise we obtain $\isctx{\G}$.

\subsubsection*{Rule {\rlTyProd}}

Consider a derivation ending with
%
\begin{equation*}
  \showTyProd
\end{equation*}
%
By induction hypothesis on the left premise we obtain $\isctx{\G}$.

\subsubsection*{Rule {\rlTyId}}

Consider a derivation ending with
%
\begin{equation*}
  \showTyId
\end{equation*}
%
By induction hypothesis on the left premise we obtain $\isctx{\G}$.

\subsection{Terms \fbox{$\isterm{\G}{\uu}{\A}$}}

In this section we prove clause \eqref{sane-isterm}.

\subsubsection*{Rule {\rlTermTyConv}}

Consider a derivation ending with
%
\begin{equation*}
  \showTermTyConv
\end{equation*}
%
By induction hypothesis on the right premise we obtain $\istype{\G}{\B}$.


\subsubsection*{Rule {\rlTermCtxConv}}

Consider a derivation ending with
%
\begin{equation*}
  \showTermCtxConv
\end{equation*}
%
By induction hypothesis on the left premise we obtain $\istype{\G}{\A}$, and then we apply
{\rlTyCtxConv} using the right premise.


\subsubsection*{Rule {\rlTermSubst}}

Consider a derivation ending with
%
\begin{equation*}
  \showTermSubst
\end{equation*}
%
By induction hypothesis on the right premise we obtain $\istype{\D}{\A}$, and then we
apply {\rlTySubst} using the left premise.

\subsubsection*{Rule {\rlTermVar}}

Consider a derivation ending with
%
\begin{equation*}
  \showTermVar
\end{equation*}
%
We apply {\rlSubstWeak} to the premise to obtain
$\issubst{\sbweak{\G}{\A}}{\ctxextend{\G}{\A}}{\G}$. Then we apply {\rlTySubst} to it
and to the premise.

\subsubsection*{Rule {\rlTermVarSkip}}

Consider a derivation ending with
%
\begin{equation*}
  \showTermVarSkip
\end{equation*}
%
By induction hypothesis on the left premise we obtain $\istype{\G}{\A}$. We
apply {\rlTySubst} to it and to {\rlSubstWeak} applied to the
right premise.


\subsubsection*{Rule {\rlTermAbs}}

Consider a derivation ending with
%
\begin{equation*}
  \showTermAbs
\end{equation*}
%
By induction hypothesis on the right premise we obtain $\istype{\ctxextend{\G}{\A}}{\B}$,
after which we apply {\rlTyProd} to it and to the left premise.


\subsubsection*{Rule {\rlTermApp}}

Consider a derivation ending with
%
\begin{equation*}
  \showTermApp
\end{equation*}
%
We first argue that we may apply {\rlSubstExtend} as follows:
%
\begin{equation*}
  \inferrule*
  {
   \inferrule*{\isctx{\G}}{\issubst{\sbid{\G}}{\G}{\G}}
   \\
   \istype{\G}{\A} \\
   \inferrule*
      {\isterm{\G}{\vv}{\A} \\
       \eqtype{\G}{\A}{\subst{\A}{\sbid{\G}}}
      }
      {\isterm{\G}{\vv}{\subst{\A}{\sbid{\G}}}}
  }
  {\issubst
   {(\sbextend{\sbid{\G}}{\A}{\vv})}
   {\G}
   {\ctxextend{\G}{\A}}
  }
\end{equation*}
%
The left premise of this derivation follows by \meta{BROKEN:CtxId}, which in turn is
justified by \meta{\{something missing: several solutions like strengthening
the theorem/program, adding $\istype{\G}{\A}$ as a premise to {\rlTermApp} or
devise a way to call the induction hypothesis on its result. \}}

\subsubsection*{Rule {\rlTermRefl}}

Consider a derivation ending with
%
\begin{equation*}
  \showTermRefl
\end{equation*}
%
By induction hypothesis on the premise we obtain $\istype{\G}{\A}$
and then by using {\rlTyId} on it and on the premise itself twice,
we conclude $\istype{\G}{\Id{\A}{\uu}{\uu}}$.

\subsection{Context equality \fbox{$\eqctx{\G}{\D}$}}

In this section we prove clause~\eqref{sane-eqctx}.

\subsubsection*{Rule {\rlEqCtxEmpty}}

Consider a derivation ending with
%
\begin{equation*}
  \showEqCtxEmpty
\end{equation*}
%
By {\rlCtxEmpty}, $\isctx{\ctxempty}$ holds.

\subsubsection*{Rule {\rlEqCtxExtend}}

Consider a derivation ending with
%
\begin{equation*}
  \showEqCtxExtend
\end{equation*}
%
We apply {\rlCtxExtend} twice to the induction hypotheses to conclude
$\isctx{\ctxextend{\G}{\A}}$ and $\isctx{\ctxextend{\D}{\B}}$.


\subsection{Type equality \fbox{$\eqtype{\G}{\A}{\B}$}}

In this section we prove clause~\eqref{sane-eqtype}.

\subsubsection*{Rule {\rlEqTyConv}}

Consider a derivation ending with
%
\begin{equation*}
  \showEqTyConv
\end{equation*}
%
By induction hypothesis on the left premise we obtain $\istype{\G}{\A}$
and $\istype{\G}{\B}$.
Using {\rlTyCtxConv} on both combined with the right premise we obtain
$\istype{\D}{\A}$ and $\istype{\D}{\B}$.

\subsubsection*{Rule {\rlEqTyRefl}}

Consider a derivation ending with
%
\begin{equation*}
  \showEqTyRefl
\end{equation*}
%
We have $\istype{\G}{\A}$ as premise.

\subsubsection*{Rule {\rlEqTySym}}

Consider a derivation ending with
%
\begin{equation*}
  \showEqTySym
\end{equation*}
%
By induction hypothesis on the premise, we conclude.

\subsubsection*{Rule {\rlEqTyTrans}}

Consider a derivation ending with
%
\begin{equation*}
  \showEqTyTrans
\end{equation*}
%
By induction hypothesis on the premises, we conclude.

\subsubsection*{Rule {\rlEqTySubstId}}

Consider a derivation ending with
%
\begin{equation*}
  \showEqTySubstId
\end{equation*}
%
We have $\istype{\G}{\A}$ as premise. We derive $\issubst{\sbid{\G}}{\G}{\G}$ from the
induction hypothesis by {\rlSubstId}, and use it in {\rlTySubst} to conclude
$\istype{\G}{\subst{\A}{\sbid{\G}}}$.

\subsubsection*{Rule {\rlEqTySubstCompose}}

Consider a derivation ending with
%
\begin{equation*}
  \showEqTySubstCompose
\end{equation*}
%
We use the first two premises and {\rlSubstCompose} to derive
$\issubst{\sbcomp{\sbs}{\sbt}}{\G}{\E}$, which we then use in {\rlTySubst} to obtain
$\istype{\G}{\subst{\A}{\sbcomp{\sbs}{\sbt}}}$. Similarly, we apply {\rlTySubst} twice
on the premises to conclude $\istype{\G}{\subst{(\subst{\A}{\sbs})}{\sbt}}$.

\subsubsection*{Rule {\rlEqTySubstProd}}

Consider a derivation ending with
%
\begin{equation*}
  \showEqTySubstProd
\end{equation*}
%
Using {\rlTyProd} and {\rlTySubst} we get
$\istype{\G}{\subst{(\Prod{\A}{\B})}{\sbs}}$.
An application of {\rlTySubst} yields $\istype{\G}{\subst{\A}{\sbs}}$
and a slightly more complicated one
$\istype{\ctxextend{\G}{\subst{\A}{\sbs}}}{\subst{\B}{\sbshift{\sbs}{\G}{\A}}}$.
We now apply {\rlTyProd} to conclude
$\istype
  {\G}
  {\Prod
    {\subst{\A}{\sbs}}
    {\subst{\B}{\sbshift{\sbs}{\G}{\A}}}}$.

\subsubsection*{Rule {\rlEqTySubstId}}

Consider a derivation ending with
%
\begin{equation*}
  \showEqTySubstId
\end{equation*}
%
We apply {\rlTyId} and {\rlTySubst} to the premises to get
$\istype{\G}{\subst{(\Id{\A}{\uu}{\vv})}{\sbs}}$. Then we apply {\rlTySubst} and
{\rlTermSubst} on the last three premises and put the results together using
{\rlTyId} to get
$\istype{\G}{\Id{\subst{\A}{\sbs}}{\subst{\uu}{\sbs}}{\subst{\vv}{\sbs}}}$.

\subsubsection*{Rule {\rlCongProd}}

Consider a derivation ending with
%
\begin{equation*}
  \showCongProd
\end{equation*}
%
The induction hypotheses yield $\istype{\G}{\A_1}$ and
$\istype{\ctxextend{\G}{\A_1}}{\A_2}$, as well as $\istype{\G}{\B_1}$ and
$\istype{\ctxextend{\G}{\A_1}}{\B_2}$. We may conclude $\istype{\G}{\Prod{A_1} A_2}$ and
$\istype{\G}{\Prod{B_1} B_2}$ by {\rlTyProd}, provided we derive
$\istype{\ctxextend{\G}{\B_1}}{\B_2}$. To do so, we first derive
$\eqctx{\ctxextend{\G}{\A_1}}{\ctxextend{\G}{\B_1}}$ by {\rlEqCtxExtend} and then
apply {\rlTyCtxConv} to $\istype{\ctxextend{\G}{\A_1}}{\B_2}$.

\subsubsection*{Rule {\rlCongId}}

Consider a derivation ending with
%
\begin{equation*}
  \showCongId
\end{equation*}
%
We apply {\rlTyId} to the induction hypotheses to get
$\istype{\G}{\Id{\A}{\uu_1}{\uu_2}}$. We obtain $\istype{\G}{\Id{\B}{\vv_1}{\vv_2}}$ in
the same way, but use {\rlTermTyConv} on the last two induction hypotheses beforehand.

\subsubsection*{Rule {\rlCongTySubst}}

Consider a derivation ending with
%
\begin{equation*}
  \showCongTySubst
\end{equation*}
%
We apply {\rlTySubst} on the left premise and the induction hypothesis
applied to the right premise to conclude.

\goodbreak

\subsection{Term equality \fbox{$\eqterm{\G}{\uu_1}{\uu_2}{\A}$}}

This section proves clause \eqref{sane-eqterm}.



\subsubsection*{Rule {\rlEqTyConv}}

Consider a derivation ending with
%
\begin{equation*}
  \showEqTyConv
\end{equation*}
%
By induction hypothesis on the left premise we obtain $\isterm{\G}{\uu}{\A}$
and $\isterm{\G}{\vv}{\A}$. Then using {\rlTermTyConv} on both and on the
right premise, we conclude $\isterm{\G}{\uu}{\B}$ and $\isterm{\G}{\vv}{\B}$.

\subsubsection*{Rule {\rlEqCtxConv}}

Consider a derivation ending with
%
\begin{equation*}
  \showEqCtxConv
\end{equation*}
%
By induction hypothesis on the left premise we obtain $\isterm{\G}{\uu}{\A}$
and $\isterm{\G}{\vv}{\A}$. Then using {\rlTermCtxConv} on both and on the
right premise, we conclude $\isterm{\D}{\uu}{\A}$ and $\isterm{\D}{\vv}{\A}$.

\subsubsection*{Rule {\rlEqRefl}}

Consider a derivation ending with
%
\begin{equation*}
  \showEqRefl
\end{equation*}
%
We have $\isterm{\G}{\uu}{\A}$ as a premise.

\subsubsection*{Rule {\rlEqSym}}

Consider a derivation ending with
%
\begin{equation*}
  \showEqSym
\end{equation*}
%
We conclude immediately using the induction hypothesis.

\subsubsection*{Rule {\rlEqTrans}}

Consider a derivation ending with
%
\begin{equation*}
  \showEqTrans
\end{equation*}
%
We conclude immediately using the induction hypotheses.

\subsubsection*{Rule {\rlEqSubstId}}

Consider a derivation ending with
%
\begin{equation*}
  \showEqSubstId
\end{equation*}
%
We have $\isterm{\G}{\uu}{\A}$ as a premise.
Moreover, \meta{\{how do we get it?\}}, $\isctx{\G}$ and so
$\issubst{\sbid{\G}}{\G}{\G}$. Then, applying {\rlTermSubst} on it and
the premise, we get
$\isterm{\G}{\subst{\uu}{\sbid{\G}}}{\subst{\A}{\sbid{\G}}}$.
Similarly, using {\rlEqTySubstId} on the induction hypothesis yielding
$\istype{\G}{\A}$, we derive $\eqtype{\G}{\subst{\A}{\sbid{\G}}}{\A}$.
With {\rlTermTyConv} we conclude
$\isterm{\G}{\subst{\uu}{\sbid{\G}}}{\A}$.

\subsubsection*{Rule {\rlEqSubstCompose}}

Consider a derivation ending with
%
\begin{equation*}
  \showEqSubstCompose
\end{equation*}
%
Using {\rlSubstCompose} to the first two premises we get
$\issubst{\sbcomp{\sbs}{\sbt}}{\G}{\E}$. Then applying {\rlTermSubst}
to it and the right premise yields
$\isterm{\G}{\subst{\uu}{\sbcomp{\sbs}{\sbt}}}{\subst{\A}{\sbcomp{\sbs}{\sbt}}}$.
%
Applying {\rlTermSubst} twice on the premises yields
$\isterm{\G}{\subst{(\subst{\uu}{\sbs})}{\sbt}}{\subst{(\subst{\A}{\sbs})}{\sbt}}$. It
remains to apply {\rlTermTyConv}, for which we need
$\eqtype{\G}{\subst{(\subst{\A}{\sbs})}{\sbt}}{\subst{\A}{\sbcomp{\sbs}{\sbt}}}$. By
induction hypothesis on the right premise we get $\istype{\E}{\A}$, to which we apply
{\rlEqTySubstCompose} and finally {\rlEqTySym}.



\subsubsection*{Rule {\rlEqSubstWeak}}

Consider a derivation ending with
%
\begin{equation*}
  \showEqSubstWeak
\end{equation*}

\subsubsection*{Rule {\rlEqSubstExtendZero}}

Consider a derivation ending with
%
\begin{equation*}
  \showEqSubstExtendZero
\end{equation*}

\subsubsection*{Rule {\rlEqSubstExtendSucc}}

Consider a derivation ending with
%
\begin{equation*}
  \showEqSubstExtendSucc
\end{equation*}

\subsubsection*{Rule {\rlEqSubstAbs}}

Consider a derivation ending with
%
\begin{equation*}
  \showEqSubstAbs
\end{equation*}

\subsubsection*{Rule {\rlEqSubstApp}}

Consider a derivation ending with
%
\begin{equation*}
  \showEqSubstApp
\end{equation*}

\subsubsection*{Rule {\rlEqSubstRefl}}

Consider a derivation ending with
%
\begin{equation*}
  \showEqSubstRefl
\end{equation*}

\subsubsection*{Rule {\rlEqReflection}}

Consider a derivation ending with
%
\begin{equation*}
  \showEqReflection
\end{equation*}


\subsubsection*{Rule {\rlProdBeta}}

Consider a derivation ending with
%
\begin{equation*}
  \showProdBeta
\end{equation*}

\subsubsection*{Rule {\rlProdEta}}

Consider a derivation ending with
%
\begin{equation*}
  \showProdEta
\end{equation*}

\subsubsection*{Rule {\rlCongAbs}}

Consider a derivation ending with
%
\begin{equation*}
  \showCongAbs
\end{equation*}

\subsubsection*{Rule {\rlCongApp}}

Consider a derivation ending with
%
\begin{equation*}
  \showCongApp
\end{equation*}

\subsubsection*{Rule {\rlCongRefl}}

Consider a derivation ending with
%
\begin{equation*}
  \showCongRefl
\end{equation*}

\subsubsection*{Rule {\rlCongTermSubst}}

Consider a derivation ending with
%
\begin{equation*}
  \showCongTermSubst
\end{equation*}

%%% Local Variables:
%%% mode: latex
%%% TeX-master: "main"
%%% End:

%
\section{Translation}
\label{sec:translation}

We need the following translations:
%
\begin{align}
  \label{lbl:ctx-ctr}
  \isctx{\G}
  &\quad\leadsto\quad
  \isctx{\G'}
  \\
  \label{lbl:ty-ctr}
  \G', (\istype{\G}{\A})
  &\quad\leadsto\quad
  \istype{\G'}{\A'}
  \\
  \label{lbl:ty-path}
  % the following is needed if we ever need to construct an equivalence given A',B' and A \equiv B
  (\istype{\G'}{\A'}), (\istype{\G}{\A})
  &\quad\leadsto\quad
  (\istype{\G'}{\A''}), (\G' \vdash \A' \simeq \A'')
  \\
  \label{lbl:term-ctr}
  (\G', A'), (\isterm{\G}{\uu}{\A})
  &\quad\leadsto\quad
  \isterm{\G'}{\uu'}{\A'}
  \\
  % the following is needed if we ever need to construct an equivalence given u', v' and u \equiv v
  % (which happens if we need to construct an equivalence given A',B' and A \equiv B)
  \label{lbl:term-path}
  (\isterm{\G'}{\uu'}{\A'}), (\isterm{\G}{\uu}{\A})
  &\quad\leadsto\quad
  (\isterm{\G'}{\uu''}{\A'}), (\G' \vdash \uu' \simeq \uu'' : \A')
  \\
  \G', (\issubst{\sbs}{\G}{\D})
  &\quad\leadsto\quad
  \issubst{\sbs'}{\G'}{\D'}
  \\
  (\issubst{\sbs'}{\G'}{\D'}), (\issubst{\sbs}{\G}{\D})
  &\quad\leadsto\quad
  (\issubst{\sbs''}{\G'}{\D''}), (\sbs' \simeq \sbs'')
  \\
  \label{lbl:eq-ctx-path}
  \G', (\eqctx{\G}{\D})
  &\quad\leadsto\quad
  (\isctx{\D'}), (\G' \simeq \D')
  \\
  \label{lbl:eq-ty-path}
  \G', \B', (\eqtype{\G}{\A}{\B})
  &\quad\leadsto\quad
  (\istype{\G'}{\A'}), (\G' \vdash \A' \simeq \B')
  \\
  \label{lbl:eq-term-path}
  \G', \A', \vv', (\eqterm{\G}{\uu}{\vv}{\A})
  &\quad\leadsto\quad
  (\isterm{\G'}{\A'}{\uu'}), (\G' \vdash \uu' \simeq \vv' : \A')
\end{align}
%
Remarks:
%
\begin{itemize}
\item \eqref{lbl:ty-path} must give the same result as~\eqref{lbl:eq-ty-path} applied to
  the reflexivity case. This is so that we can use the result of~\eqref{lbl:eq-ty-path} as
  if it were obtained through conversion.
\item \eqref{lbl:term-path} must give the same result as~\eqref{lbl:eq-term-path} applied
  to the reflexivity case, for similar reasons.
\item We will need to know that all the equivalences generated from the derivations of a
  given equation are equal (propositionally pointwise). This is where UIP will come into
  play, since such equivalences are compositions of structural acrobatics and transports
  along paths used in $\rl{eq-reflection}$.
\item We also need to know that translations of contexts have the same shape
  as the original context, and the same goes for translation of types up
  to context isomorphisms.
\item In the rules producing isomorphims, we actually assume that the center
  generated is the same as the one that would be generated by the corresponding
  ``simple'' rule.
\end{itemize}


\subsection{Proof of the translation}
\label{sec:proof-tran}

\Case{ctx-empty}
%
The derivation
%
\begin{equation*}
  \infer{ }{\isctx{\ctxempty}}
\end{equation*}
%
is translated into
%
\begin{equation*}
  \infer{ }{\isctx{\ctxempty}}
\end{equation*}


\Case{ctx-extend}

Consider the derivation
%
%
\begin{equation*}
  \infer{
    \inferrule*{\DD}{\isctx{\G}} \\
    \inferrule*{\EE}{\istype{\G}{\A}} \\
    x \not\in \ctxdom{\G}
  }
  {\isctx{(\ctxextend{\G}{\x}{\A})}}
\end{equation*}
%
By~\eqref{lbl:ctx-ctr} on $\derives{\DD}{\isctx{\G}}$, we get
$\derives{\DD'}{\isctx{\G'}}$.
By~\eqref{lbl:ty-ctr} on $\derives{\EE}{\istype{\G}{\A}}$ and $\G'$, we get
$\derives{\EE'}{\istype{\G'}{\A'}}$.
Since $\ctxdom{\G} = \ctxdom{\G'}$, we have $x \notin \ctxdom{\G'}$ and thus
%
\begin{equation*}
  \infer{
    \inferrule*{\DD'}{\isctx{\G'}} \\
    \inferrule*{\EE'}{\istype{\G'}{\A'}} \\
    x \not\in \ctxdom{\G'}
  }
  {\isctx{(\ctxextend{\G'}{\x}{\A'})}}
\end{equation*}


\Case{ty-ctx-conv}

Consider the derivation
%
%
\begin{equation*}
  \infer{
    \inferrule*{\DD}{\istype{\G}{\A}} \\
    \inferrule*{\EE}{\eqctx{\G}{\D}}
  }
  {\istype{\D}{\A}}
\end{equation*}
%
We need to prove~\eqref{lbl:ty-ctr} and~\eqref{lbl:ty-path}.
First assume we have $\D'$ a translation of $\D$.
From $\derives{\EE}{\eqctx{\G}{\D}}$ and $\D'$, we
apply~\eqref{lbl:eq-ctx-path} (\meta{in the symmetric version}) and we deduce
$\G'$ a translation of $\G$ such that $\G' \simeq \D'$ (meaning that in
particular we have some morphism $f$ from $\G'$ to $\D'$).
From $\G'$ and $\derives{\DD}{\istype{\G}{\A}}$, we apply~\eqref{lbl:ty-ctr}
to get $\derives{\DD'}{\istype{\G'}{\A'}}$.
We can finally build the translation:
%
\begin{equation*}
  \infer{
    \inferrule*{\DD'}{\istype{\G'}{\A'}} \\
    \issubst{\f}{\D'}{\G'}
  }
  {\istype{\D'}{\subst{\A'}{f}}}
\end{equation*}
%
\meta{We need to specify who is $\f$ (and its derivation) and how it applies.}

Now, let's extend our proof to match~\eqref{lbl:ty-path}.
Assume we are given $\istype{\D'}{\A''}$, let's build an equivalence
$\D' \vdash \subst{\A'}{\f} \simeq \A''$.
Since $\G' \simeq \D'$, we have an inverse to $\f$, say $\g$.
For this we deduce $\istype{\G'}{\subst{\A''}{\g}}$.
Using this, together with $\istype{\G'}{\A'}$, we can apply~\eqref{lbl:ty-path}
to get some isomorphism $\G' \vdash \A' \simeq \subst{\A''}{\g}$.
From this we can build an isomorphism
$\D' \vdash \subst{\A'}{\f} \simeq \subst{\subst{\A''}{\g}}{\f}$.
But as we said, $\g$ and $\f$ are inverses, so
$\D' \vdash \subst{\subst{\A''}{\g}}{\f} \simeq \A''$, hence
$\D' \vdash \subst{\A'}{\f} \simeq \A''$.

\Case{ty-subst}
\meta{TODO}

\Case{ty-prod}

Consider
%
\begin{equation*}
  \infer{
    \inferrule*{\DD}{\istype{\G}{\A}} \\
    \inferrule*{\EE}{\istype{\ctxextend{\G}{\x}{\A}}{\B}}
  }
  {\istype{\G}{\Prod{\x}{\A}{\B}}}
\end{equation*}
%
We prove~\eqref{lbl:ty-ctr} and then~\eqref{lbl:ty-path}.
Assume $\G'$, from~\eqref{lbl:ty-ctr}, we get $\istype{\G'}{\A'}$ and then
using $\ctxextend{\G'}{\x}{\A'}$
(well-formed since $x \notin \ctxdom{\G} = \ctxdom{\G'}$)
we get $\istype{\ctxextend{\G'}{\x}{\A'}}{\B'}$.
Finally we reconstruct $\istype{\G'}{\Prod{\x}{\A'}{\B'}}$.

Now, to prove~\eqref{lbl:ty-path}, assume $\istype{\G'}{\C''}$
is some translation of $\istype{\G}{\Prod{\x}{\A}{\B}}$.
If we invert on the translation, $\C''$ is actually an arbitrary number
(possibly zero) of context morphisms applied to a $\Pi$-type translation
$\istype{\G''}{\Prod{\x}{\A''}{\B''}}$.
To simplify matters, we assume we only have one such context morphism
$\G' \simeq \G''$ with $\issubst{\f}{\G''}{\G'}$ and $\issubst{\g}{\G'}{\G''}$
which is the composition of all the substitutions (in case we have none, this
would result in the identity morphism). It is indeed admissible as we can show
an equivalence between the two types (the one with one, and the one with an
arbitrary number of substitutions in its head).
By inversion, $\istype{\G''}{\A''}$, and thus $\istype{\G'}{\subst{\A''}{\g}}$
so by~\eqref{lbl:ty-path}, we have $\G' \vdash \subst{\A''}{\g} \simeq \A'$,
in particular, we can extend $\f$ and $\g$ into $\f'$ and $\g'$ to get an
isomorphism $\ctxextend{\G'}{\x}{\A'} \simeq \ctxextend{\G''}{\x}{\A''}$.
By inversion, we also have $\istype{\ctxextend{\G''}{\x}{\A''}}{\B''}$,
and thus $\istype{\ctxextend{\G'}{\x}{\A'}}{\subst{\B''}{\g'}}$,
so by~\eqref{lbl:ty-path}, we have
$\ctxextend{\G'}{\x}{\A'} \vdash \subst{\B''}{\g'} \simeq \B'$.

We can derive from that
$\G' \vdash \Prod{\x}{\subst{\A''}{\g}}{\subst{\B''}{\g'}}
\simeq \Prod{\x}{\A'}{\B'}$.
However, \meta{magically},
$\G' \vdash \Prod{\x}{\subst{\A''}{\g}}{\subst{\B''}{\g'}}
\simeq \subst{(\Prod{\x}{\A''}{\B''})}{\g}$,
which gives the expected conclusion by transitivity.
(\meta{We should write it in a better way.})

\Case{ty-id}

Consider
%
\begin{equation*}
  \infer{
    \istype{\G}{\A} \\
    \isterm{\G}{\uu}{\A} \\
    \isterm{\G}{\vv}{\A}
  }
  {\istype{\G}{\Id{\A}{\uu}{\vv}}}
\end{equation*}
%
First, let us handle~\eqref{lbl:ty-ctr}. Assume $\G'$ a translation of $\G$.
From~\eqref{lbl:ty-ctr}, we get $\istype{\G'}{\A'}$ and
from~\eqref{lbl:term-ctr} using $\G'$ and $\A'$, we get
$\isterm{\G'}{\uu'}{\A'}$ and $\isterm{\G'}{\vv'}{\A'}$.
We finally have $\istype{\G'}{\Id{\A'}{\uu'}{\vv'}}$.

\sloppy
Now, let us prove~\eqref{lbl:ty-path} by assuming $\istype{\G'}{\C''}$ another
translation of $\istype{\G}{\Id{\A}{\uu}{\vv}}$.
Using the same trick as before, \eqref{lbl:ty-path} and~\eqref{lbl:term-path},
and the \meta{magic} fact that
%
\begin{equation*}
  \G' \vdash \Id{\subst{\A''}{\g}}{\subst{\uu''}{\g}}{\subst{\vv''}{\g}}
  \simeq \subst{(\Id{\A''}{\uu''}{\vv''})}{\g},
\end{equation*}
%
we can conclude.

\Case{term-ty-conv}

Consider
%
\begin{equation*}
  \infer{
    \isterm{\G}{\uu}{\A} \\
    \eqtype{\G}{\A}{\B}
  }
  {\isterm{\G}{\uu}{\B}}
\end{equation*}
%
This time we have to prove~\eqref{lbl:term-ctr} and~\eqref{lbl:term-path}.
First, for~\eqref{lbl:term-ctr}, assume $\G'$ and $\B'$ respective translations
of $\G$ and $\B$. From that and~\eqref{lbl:eq-ty-path} we get
$\istype{\G'}{\A'}$ and some equivalence $\G' \vdash \A' \simeq \B'$.
From $\G'$, $\A'$ and~\eqref{lbl:term-ctr}, we get $\isterm{\G'}{\uu'}{\A'}$.
We take the direct map from the equivalence $\f$ and finally
conclude $\isterm{\G'}{\app{\f}{\_}{\A'}{\B'}{\uu'}}{\B'}$.

Now, for~\eqref{lbl:term-path}, assume another translation
$\isterm{\G'}{\uu''}{\B'}$.
Using $\g$, the inverse of $\f$, we get
$\isterm{\G'}{\app{\g}{\_}{\B'}{\A'}{\uu''}}{\A'}$ which is a valid translation
thanks to the remarks. Thus, using~\eqref{lbl:term-path}, we get an equivalence
$\G' \vdash \app{\g}{\_}{\B'}{\A'}{\uu''} \simeq \uu' : \A'$.
From it we deduce the equivalence
$\G' \vdash \app{\f}{\_}{\A'}{\B'}{(\app{\g}{\_}{\B'}{\A'}{\uu''})}
\simeq \app{\f}{\_}{\A'}{\B'}{\uu'} : \A'$.
Since $\f$ and $\g$ are inverses, we have
$\G' \vdash \app{\f}{\_}{\A'}{\B'}{(\app{\g}{\_}{\B'}{\A'}{\uu''})}
\simeq \uu'' : \A'$ and so
$\G' \vdash \uu'' \simeq \app{\f}{\_}{\A'}{\B'}{\uu'} : \A'$.

\Case{term-ctx-conv}

%
\begin{equation*}
  \infer{
    \isterm{\G}{\uu}{\A} \\
    \eqctx{\G}{\D}
  }
  {\isterm{\D}{\uu}{\A}}
\end{equation*}

\eqref{lbl:term-ctr}: Assume $\D'$ and $\A'$.
By~\eqref{lbl:eq-ctx-path}, we have $\G'$ and $\G' \simeq \D'$.
Then\footnote{Assuming $\subst{\A'}{\g}$ is a suitable translation!},
by~\eqref{lbl:term-ctr}, $\isterm{\G'}{\uu'}{\subst{\A'}{\g}}$ and thus
$\isterm{\D'}{\subst{\uu'}{\f}}{\A'}$ (where $\f$ and $\g$ are the equivalence
$\G' \simeq \D'$). \meta{(How do we get rid of $\subst{\subst{\A'}{\g}}{\f}$
in favor of $\A'$? Probably by conversion in the target.)}

\eqref{lbl:term-path}: Moreover assume $\isterm{\D'}{\uu''}{\A'}$.
Thus, $\isterm{\G'}{\subst{\uu''}{\g}}{\subst{\A'}{\g}}$.
We deduce form~\eqref{lbl:term-path} that
$\G' \vdash \subst{\uu''}{\g} \simeq \uu' : \subst{\A'}{\g}$.
And finally
$\G' \vdash \uu'' \simeq \subst{\uu'}{\f} : \A'$.

\Case{term-subst}

\meta{TODO}

\Case{term-var}

%
\begin{equation*}
  \infer{
    \isctx{\G} \\
    \x \not\in \ctxdom{\G} \\
    \istype{\G}{\A} \\
  }{
    \isterm
      {\ctxextend{\G} {\x}{\A}}
      {\x}
      {\A}
  }
\end{equation*}

\eqref{lbl:term-ctr}: Assume $\ctxextend{\G'}{\x}{\A'}$ a translation of
$\ctxextend{\G}{\x}{\A}$ and $\A''$ a translation of $\A$.
By definition of a translation of contexts, $\G'$ is a translation of
$\G$ and $\A'$ a translation of $\A$.
\meta{Besides}, we have $\isctx{\G'}$ and $\istype{\G'}{\A''}$.
Since, for $\A'$ is also a translation of $\A$, we have $\istype{\G'}{\A'}$
\meta{(we need some strengthening by saying that $\x$ should not appear in
$\A'$)}, we can apply~\eqref{lbl:ty-path} twice to get an equivalence
$\G' \vdash \A' \simeq \A''$.
Thanks to the rule \rl{term-var}, we have
$\isterm{\ctxextend{\G'}{\x}{\A'}}{\x}{\A'}$
and then
$\isterm{\ctxextend{\G'}{\x}{\A'}}{\app{\f}{\_}{\A'}{\A''}{\x}}{\A''}$
where $\f$ is one of the underlying maps of the equivalence.

\eqref{lbl:term-path}: Assume, moreover, that
$\isterm{\ctxextend{\G'}{\x}{\A'}}{\uu'}{\A''}$
is another translation. Let's build an equivalence between it and the one we
produced.
\meta{This is where things start to be complicated. We can indeed say that
any translation of $\x$ has to be $\x$ transported along isomorphisms and
context isomorphisms. We apply all of their inverses in the right order to
discover $\x$ in an equivalent context, and at an equivalent type.
This is probably one of the times where we require the produced equivalences
to be equal.}


\Case{term-var-skip}

\begin{equation*}
  \infer{
    \isterm{\G}{\x}{\A} \\
    \y \not\in \ctxdom{\G} \\
    \istype{\G}{\B}
  }{
    \isterm
      {\ctxextend{\G}{\y}{\B}}
      {\x}
      {\A}
  }
\end{equation*}

\eqref{lbl:term-ctr}: Assume $\ctxextend{\G'}{\y}{\B'}$ and $\A'$
respective translations of $\ctxextend{\G}{\y}{\B}$ and $\A$.
By definition of a context translation, $\G'$ is a translation of $\G$.
Thus, by~\eqref{lbl:term-ctr} on $\isterm{\G}{\x}{\A}$, we have
$\isterm{\G'}{\uu'}{\A'}$ a translation.
We still have $\y \notin \ctxdom{\G'}$.
And since $\B'$ is a translation of $\B$ in translated context $\G'$,
we have $\istype{\G'}{\B'}$, from that we conclude
$\isterm{\ctxextend{\G'}{\y}{\B'}}{\uu'}{\A'}$.

\eqref{lbl:term-path}: Assume, moreover, that
$\isterm{\ctxextend{\G'}{\y}{\B'}}{\uu''}{\A'}$
is another translation, let's build a path.
\meta{Here it is unclear how to remove $\y$ from the context to use the
induction hypothesis...}

%%% Local Variables:
%%% mode: latex
%%% TeX-master: "main"
%%% End:



\end{document}
