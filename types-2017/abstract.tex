\documentclass{easychair}

\usepackage{doc}
\usepackage[utf8]{inputenc}
\usepackage{amsmath}

\newcommand{\Id}[3]{\mathsf{Id}_{#1}(#2,#3)}
\newcommand{\refl}[2]{\mathsf{refl}_{#1}\,#2}

\newcommand{\ctx}{\Gamma}
\newcommand{\isctx}{\ctx\;\mathsf{context}}
\newcommand{\istype}[1]{\ctx \vdash #1 \;\mathsf{type}}
\newcommand{\isterm}[2]{\ctx \vdash #1 : #2}

%% Front Matter
%%
% Regular title as in the article class.
%
\title{A modular formalization of type theory in Coq}

% Authors are joined by \and. Their affiliations are given by \inst, which indexes
% into the list defined using \institute
%
\author{
Andrej Bauer\inst{1}\thanks{
The author acknowledges the financial support from the Slovenian Research Agency (research core funding No.~P1-0294) at Institute of Mathematics, Physics and Mechanics, Ljubljana, Slovenia.}
\and
Philipp G.~Haselwarter\inst{2}\thanks{This material is based upon work supported by the Air Force Office of Scientific Research, Air Force Materiel Command, USAF under Award No.~FA9550-14-1-0096.}
\and
Théo Winterhalter\inst{3}
}

% Institutes for affiliations are also joined by \and,
\institute{
  University of Ljubljana, Slovenia\\
  \email{Andrej.Bauer@andrej.com}
\and
  University of Ljubljana, Slovenia\\
  \email{philipp@haselwarter.org}
\and
   École Normale Supérieure Paris-Saclay, France\\
   \email{theo.winterhalter@ens-cachan.fr}
}

%  \authorrunning{} has to be set for the shorter version of the authors' names;
% otherwise a warning will be rendered in the running heads. When processed by
% EasyChair, this command is mandatory: a document without \authorrunning
% will be rejected by EasyChair

\authorrunning{Bauer, Haselwarter and Winterhalter}

% \titlerunning{} has to be set to either the main title or its shorter
% version for the running heads. When processed by
% EasyChair, this command is mandatory: a document without \titlerunning
% will be rejected by EasyChair
\titlerunning{A modular formalization of type theory in Coq}


\begin{document}

\maketitle

\begin{abstract}
  We present a complete formalization of type theory in the Coq proof assistant. We use
  the type class mechanism to break down the formalization into fragments that can then be
  combined to several variants of type theory. We prove basic meta-theorems guaranteeing
  that our formulation is complete and sensible, as well as theorems that facilitate
  applications. To test the library, we formalize a translation of type theory into type
  theory by Boulier et al.\ which shows that function extensionality is not derivable.
\end{abstract}

\paragraph*{Introduction.}
\label{sec:introduction}

As anyone who has ever proved meta-theorems about dependent type theory will attest, the
excitement of discovering a new fact is dulled by the drudgery of checking large boring
proofs. In fact, just \emph{stating} precisely and completely all the rules of type theory
is considered inhumane, and therefore omitted in practically all presentations of type
theory. The received wisdom says that such informal rigor causes no harm and that
everything is known to work out. This may well be the case---although how can we really
know?---but we have learned the hard way that incomplete formulations of meta-theorems
with sloppy proofs lure us into a morass of confusion and half-truths that hinders
progress. In other words, we shall not trust ourselves again to have stated a meta-theorem
about dependent type theory, and much less proved it, unless it is fully formalized. Our
formalization of type theory is implemented in the Coq proof assistant~\cite{coq}, and is
freely available at~\cite{reflections}.

\paragraph*{The basic setup.}
\label{sec:basic-setup}

For our purposes (which have not come to fruition yet), we needed a formalization of
several variants of type theory, which all shared common basic structure but differed in
the inclusion or exclusion of certain principles and constructors. We intended to
manipulate all parts of the syntax and the derivations, including substitutions and
judgmental equalities. We thus opted for a ``deep'' formalization of the syntax and the
rules as straightforward inductive types, thereby having complete control over the
object-level theory, albeit at the price of having to perform manually various syntactic
manipulations that a ``shallower'' embedding would simply pass on to the ambient type
theory of the proof assistant. After several trials and make-overs we opted for a nameless
representation of variables, explicit substitutions, and terms fully annotated with types.

At present the formalization includes product and identity types, base types (the empty
type, the unit type, and booleans), as well as simple products. The modular design ought
to make it easy to add other constructions.

% - syntax, judgments, inference rules
% - explicit substitutions

\paragraph*{Paranoid and economic inference rules.}

% - paranoid vs. economic

There is an amount of freedom in the formulation of inference rules. For example, the
introduction rule for the identity type can be stated
%
\begin{equation*}
  \text{as}\qquad
  \frac{\isctx \qquad
  \istype{A} \qquad
  \isterm{u}{A}
  }{\isterm{\refl{A}{u}}{\Id{A}{u}{u}}}
  %
  \qquad\text{or as}\qquad
  %
  \frac{\isterm{u}{A}
  }{\isterm{\refl{A}{u}}{\Id{A}{u}{u}}}.
\end{equation*}
%
We call the left one \emph{paranoid} and the right one \emph{economic}. When proving things about type theory, it is generally
preferable to have the paranoid version as an assumption and the economic one as the goal.
We formulated type theory so that the paranoid or the economic variant of the rules can be
chosen easily through a type class instance, and we proved meta-theorems showing that they
derive the same judgments. Thus, we can easily switch between them and use whichever is
more convenient in a given situation.

\paragraph*{Configurations and variants of type theory.}
\label{sec:config-variant}

% - configurability through type classes

The inductive types for the syntax and the judgment forms are parameterized by type classes
that either enable or disable various features of the formalization. By providing
instances of the type classes, the user may combine the features to their liking. It is
also possible to leave a feature unconfigured, and thus have a development which is
agnostic with respect to it. At present we support the following features: paranoid
vs.~economic rules, extensional vs.~intensional identity types, presence of simple product
types, and presence of the $\eta$-rule for functions.

\paragraph*{Sanity theorems and other meta-theorem.}
\label{sec:sanity-theorems}

In our experience it is easy to forget an inference rule, or just formulate it badly. We
proved several meta-theorems which instill some trust in our formulation, and are
generally useful to have:
%
\begin{itemize}
\item \emph{sanity} theorems stating, e.g., that if $\isterm{u}{A}$ then $\istype{A}$ and $\isctx$,
\item the paranoid and the economic variants derive the same judgments,
\item uniqueness of typing: if $\isterm{u}{A}$ and $\isterm{u}{B}$ then $A$ and $B$ are
  judgmentally equal.
\end{itemize}
%
Another desirable theorem (which we intend to prove) is elimination of explicit
substitutions: every term is judgmentally equal to one without substitutions. A proof of
this fact amounts to computation of substitutions at the meta-level.

% meta-theorems:
% - equivalence of paranoid and economic
% - sanity theorems
% - uniqueness of typing

% what it takes to add another feature (say, simple products)

\paragraph*{Conclusion.}

To test the viability of our development we used it to state a translation of type theory
into type theory which invalidates function extensionality, following the work of
S.~Boulier, P.-M.~Pédrot, and N.~Tabareau~\cite{boulier17:_next_syntac_model_type_theor}.
Our original motivation was formalization of elimination of equality reflection, but we
leave that for future work.

There are several ways in which the library can be improved: we can provide better tactics
for constructing terms and derivations, improve the syntax and notations, and include more
features, such as inductive types and universes.

% application: funext is not derivable

% todo:
% - better magic
% - better syntactic support
% - N, universe, inductive types
%  elimination of substitutions

\bibliographystyle{plain}
\bibliography{references}



\end{document}
