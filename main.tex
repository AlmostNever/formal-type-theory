\documentclass{amsart}

\usepackage[utf8]{inputenc}
\usepackage{times}
\usepackage{amsmath,amssymb}
\usepackage{amsthm}
\usepackage{mathpartir}
\usepackage{hyperref}

% Add some colors
\usepackage[usenames,dvipsnames,svgnames,table]{xcolor}
\hypersetup{
 linktocpage,
 colorlinks,
 citecolor=BlueViolet,
 filecolor=red,
 linkcolor=Blue,
 urlcolor=BrickRed
}

% Meta comment
\newcommand\meta[1]{\noindent\textcolor{blue}{\emph{#1}}}

% Include the macro file
% evergreens
\newcommand{\der}{\,\vdash}
\newcommand{\Der}{\,\Vdash}

%specific judgments
\newcommand{\dere}{\der_\mathbf{E}}

% semantic brackets
\def\lv{\mathopen{{[\kern-0.14em[}}}    % opening [[ value delimiter
\def\rv{\mathclose{{]\kern-0.14em]}}}   % closing ]] value delimiter
\newcommand{\den}[1]{\lv #1 \rv}
\newcommand{\Den}[3][]{\den{#2}^{#1}_{#3}}
\newcommand{\dent}[2]{\llparenthesis#1\rrparenthesis_{#2}}

% space-preserving paragraph headings
\newcommand{\subheading}[1]{\subparagraph{#1.}} %Alt: \subsection{#1}
% \newcommand{\paradot}[1]{\subparagraph{#1.}}
\newcommand{\paradot}[1]{\subsection*{#1.}}

% Inference rules
\newcommand{\rulename}[1]{\ensuremath{\mbox{\sc#1}}}
\newcommand{\ru}[2]{\dfrac{\begin{array}[b]{@{}c@{}} #1 \end{array}}{#2}}
\newcommand{\rux}[3]{\ru{#1}{#2}~#3}
\newcommand{\nru}[3]{#1\ \ru{#2}{#3}}
\newcommand{\nrux}[4]{#1\ \ru{#2}{#3}\ #4}
\newcommand{\dstack}[2]{\begin{array}[b]{c}#1\\#2\end{array}}
\newcommand{\dru}[3]{\ru{\dstack{#1}{#2}}{#3}}
\newcommand{\drux}[4]{\ru{\dstack{#1}{#2}}{#3}\ #4}
\newcommand{\tru}[4]{\dru{\dstack{#1}{#2}}{#3}{#4}}
\newcommand{\trux}[5]{\dru{\dstack{#1}{#2}}{#3}{#4}\ #5}
\newcommand{\qru}[5]{\tru{\dstack{#1}{#2}}{#3}{#4}{#5}}
\newcommand{\ndru}[4]{#1\ \ru{\dstack{#2}{#3}}{#4}}
\newcommand{\ndrux}[5]{#1\ \ru{\dstack{#2}{#3}}{#4}\ #5}

% Centered math environment
\newenvironment{mathc}{%
  \begin{center}%
  \(%
}{\)%
  \end{center}%
}

% proof by cases
\newenvironment{caselist}{%
  \begin{list}{{\it Case}}{}%
}{\end{list}%
}
\newenvironment{subcaselist}{%
  \begin{list}{{\it Subcase}}{}%
}{\end{list}%
}
\newenvironment{subsubcaselist}{%
  \begin{list}{{\it Subsubcase}}{}%
}{\end{list}%
}

\newcommand{\nextcase}{\item~}

% Symbols and names
\DeclareMathOperator{\J}{\mathbf{J}}
\DeclareMathOperator{\type}{\mathbf{type}}
\newcommand{\Id}[3]{\mathbf{Id}_{#1}\,#2\ #3}
\newcommand{\app}[3]{#2\ @^{#1}\ #3}
\newcommand\infers{\rightrightarrows}
\newcommand\checks{\leftleftarrows}
\DeclareMathOperator{\refl}{\mathbf{refl}}
\newcommand\barG{\overline{\Gamma}}
\newcommand\bart{\overline{t}}
\newcommand\barT{\overline{T}}
\newcommand\barA{\overline{A}}
\newcommand\barB{\overline{B}}
\newcommand\baru{\overline{u}}
\newcommand\barv{\overline{v}}
\newcommand\barp{\overline{p}}


\newtheorem{theorem}{Theorem}[section]

\begin{document}

\title{Elimination of equality reflection}

\author{Andrej Bauer}
\address{Andrej Bauer\\University of Ljubljana\\Slovenia}
\email{Andrej.Bauer@andrej.com}

\author{Théo Winterhalter}
\address{Théo Winterhalter\\ENS Cachan, Université Paris-Saclay\\France}
\email{theo.winterhalter@gmail.com}

\begin{abstract}
  Equality reflection for h-sets is conservative over intensional type theory.
\end{abstract}

\maketitle


\section{Type theory}
\label{sec:type-theory}

In this section we give the formulation of type theory that we shall work with.

\subsection{Syntax}
\label{sec:syntax}

\begin{align*}
  \text{Context $\G$, $\D$}
    \bnf   {}& \ctxempty                && \text{empty context} \\
    \bnfor {}& \ctxextend{\G}{\A}       && \text{context $\G$ extended with $\A$} \\
  \\
  \text{Type $\A$, $\B$, $\C$}
    \bnf   {}& \Prod{\A} \B             && \text{product}\\
    \bnfor {}& \Id{\A}{\uu}{\vv}        && \text{identity type} \\
    \bnfor {}& \subst{\A}{\sbs}         && \text{substitution $\sbs$ applied to $\A$} \\
  \\
  \text{Term $\uu$, $\vv$, $\ww$}
    \bnf   {}& \var{k}                  && \text{the $k$-th variable index (a la de Bruijn)} \\
    \bnfor {}& \lam{\A}{\B} \uu         && \text{$\lambda$-abstraction} \\
    \bnfor {}& \app{\uu}{\A}{\B}{\vv}   && \text{application} \\
    \bnfor {}& \refl{\A} \uu            && \text{reflexivity} \\
    \bnfor {}& \subst{\uu}{\sbs}        && \text{substitution $\sbs$ applied to $\uu$} \\
  \\
  \text{Substitution $\sbs$, $\sbt$}
    \bnf   {}& \sbid{\G}                && \text{identity substitution $\G \to \G$} \\
    \bnfor {}& \sbcomp{\sbs}{\sbt}      && \text{composition of substitutions} \\
    \bnfor {}& \sbextend{\sbs}{\A}{\uu} && \text{substitution $\sbs$ extended with $\uu$ of type $\subst{\A}{\sbs}$} \\
    \bnfor {}& \sbweak{\G}{\A}          && \text{weakening substitution $\ctxextend{\G}{\A} \to \G$}
\end{align*}

\subsection{Judgments}
\label{sec:judgments}

\begin{align*}
& \isctx{\G}                    && \text{$\G$ is a context} \\
& \issubst{\sbs}{\G}{\D}        && \text{$\sbs$ is a substitution from context $\G$ to context $\D$} \\
& \istype{\G}{\A}               && \text{$\A$ is a type in context $\G$} \\
& \isterm{\G}{\uu}{\A}          && \text{$\uu$ is a term of type $\A$ in context $\G$} \\
& \eqctx{\G}{\D}                && \text{$\G$ and $\D$ are equal contexts} \\
& \eqtype{\G}{\A}{\B}           && \text{$\A$ and $\B$ are equal types in context $\G$} \\
& \eqterm{\G}{\uu}{\vv}{\A}     && \text{$\uu$ and $\vv$ are equal terms of type $\A$ in context $\G$}
\end{align*}

\subsection{Contexts \fbox{$\isctx{\G}$}}
\label{sec:contexts}

\begin{mathpar}
  \infer[\rl{ctx-empty}]
  { }
  {\isctx{\ctxempty}}

  \infer[\rl{ctx-extend}]
  {\istype{\G}{\A}
  }
  {\isctx{(\ctxextend{\G}{\A})}}
\end{mathpar}

\subsection{Substitutions \fbox{$\issubst{\sbs}{\G}{\D}$}}
\label{sec:subst}

\begin{mathpar}
  \infer[\rl{subst-id}]
  {\isctx{\G}}
  {\issubst{\sbid{\G}}{\G}{\G}}

  \infer[\rl{subst-compose}]
  {\issubst{\sbs}{\G}{\D} \\
   \issubst{\sbt}{\D}{\E}
  }
  {\issubst{\sbcomp{\sbt}{\sbs}}{\G}{\E}}

  \infer[\rl{subst-extend}]
  {\issubst{\sbs}{\G}{\D} \\
   \istype{\D}{\A} \\
   \isterm{\G}{\uu}{\subst{\A}{\sbs}}
  }
  {\issubst
     {(\sbextend{\sbs}{\A}{\uu})}
     {\G}
     {(\ctxextend{\D}{\A})}
  }

  \infer[\rl{subst-weak}]
  {\istype{\G}{\A}}
  {\issubst
     {\sbweak{\G}{\A}}
     {\ctxextend{\G}{\A}}
     {\G}
  }
\end{mathpar}

We let $\sbshift{\sbs}{\G}{\A}$ stand for
$\sbextend{\sbcomp{\sbs}{\sbweak{\G}{\A}}}{\A}{\var{0}}$. Then we have the derivable rule
%
\begin{equation*}
  \infer
  {\issubst{\sbs}{\G}{\D} \\
   \istype{\D}{\A}
  }
  {\issubst
      {(\sbshift{\sbs}{\G}{\A})}
      {\ctxextend{\G}{\subst{\A}{\sbs}}}
      {\ctxextend{\D}{\A}}
  }
\end{equation*}

\subsection{Types \fbox{$\istype{\G}{\A}$}}

\subsubsection*{General rules}

\begin{mathpar}
  \infer[\rl{ty-ctx-conv}]
  {\istype{\G}{\A} \\
    \eqctx{\G}{\D}
  }
  {\istype{\D}{\A}}

  \infer[\rl{ty-subst}]
  {\issubst{\sbs}{\G}{\D} \\
   \istype{\D}{\A}
  }
  {\istype{\G}{\subst{\A}{\sbs}}}
\end{mathpar}

\subsubsection*{Type formers}

\begin{mathpar}
  \infer[\rl{ty-prod}]
  {\istype{\G}{\A} \\
   \istype{\ctxextend{\G}{\A}}{\B}
  }
  {\istype{\G}{\Prod{\A}{\B}}}

  \infer[\rl{ty-id}]
  {\istype{\G}{\A}\\
   \isterm{\G}{\uu}{\A}\\
   \isterm{\G}{\vv}{\A}
  }
  {\istype{\G}{\Id{\A}{\uu}{\vv}}}
\end{mathpar}

\subsection{Terms \fbox{$\isterm{\G}{\uu}{\A}$}}

\subsubsection*{General rules}
\begin{mathpar}
  \infer[\rl{term-ty-conv}]
  {\isterm{\G}{\uu}{\A} \\
   \eqtype{\G}{\A}{\B}
  }
  {\isterm{\G}{\uu}{\B}}

  \infer[\rl{term-ctx-conv}]
  {\isterm{\G}{\uu}{\A} \\
   \eqctx{\G}{\D}
  }
  {\isterm{\D}{\uu}{\A}}

  \infer[\rl{term-subst}]
  {\issubst{\sbs}{\G}{\D} \\
   \isterm{\D}{\uu}{\A}
  }
  {\isterm{\G}{\subst{\uu}{\sbs}}{\subst{\A}{\sbs}}}
\end{mathpar}

\subsubsection*{Variables}

\begin{mathpar}
  \infer[\rl{term-var}]
  {\isctx{\G} \\
   \istype{\G}{\A}
  }
  {\isterm
     {\ctxextend{\G}{\A}}
     {\var{0}}
     {\subst{\A}{\sbweak{\G}{\A}}}
  }

  \infer[\rl{term-var-skip}]
  {\isterm{\G}{\var{k}}{\A} \\
   \istype{\G}{\B}
  }
  {\isterm
     {\ctxextend{\G}{\B}}
     {\var{k+1}}
     {\subst{\A}{\sbweak{\G}{\B}}}
  }
  \end{mathpar}

\subsubsection*{Abstraction and application}

\begin{mathpar}
  % Remark: we want \istype{\G}{\A} as a premise because in order to form the
  % product type we need it, and we do not want to resort to inversion on the
  % extended context.
  \infer[\rl{term-abs}]
  { \istype{\G}{\A} \\
    \isterm{\ctxextend{\G}{\A}}{\uu}{\B}
  }
  {\isterm{\G}{(\lam{\A}{\B}{\uu})}{\Prod{\A}{\B}}}

  % Remark: we want \istype{\ctxextend{\G}{\A}}{\B} because we need to
  % know that \B is a type in \ctxextend{\G}{\A} and we want to avoid
  % inversion on ty-prod.
  \infer[\rl{term-app}]
  {\istype{\ctxextend{\G}{\A}}{\B} \\
   \isterm{\G}{\uu}{\Prod{\A} \B} \\
   \isterm{\G}{\vv}{\A}
  }
  {\isterm
     {\G}
     {\app{\uu}{\A}{\B}{\vv}}
     {\subst{\B}{\sbextend{\sbid{\G}}{\A}{\vv}}}
  }
\end{mathpar}


\subsubsection*{Reflexivity}

\begin{mathpar}
  \infer[\rl{term-refl}]
  {\isterm{\G}{\uu}{\A}}
  {\isterm{\G}{\refl{\A} \uu}{\Id{\A}{\uu}{\uu}}}
\end{mathpar}

\subsection{Context equality \fbox{$\eqctx{\G}{\D}$}}
\label{sec:cont-equal}

\begin{mathpar}
  \infer[\rl{eq-ctx-empty}]
  { }
  {\eqctx{\ctxempty}{\ctxempty}}

  \infer[\rl{eq-ctx-extend}]
  {\eqctx{\G}{\D} \\
   \eqtype{\G}{\A}{\B}
  }
  {\eqctx{(\ctxextend{\G}{\A})}{(\ctxextend{\D}{\B})}}

\end{mathpar}

\goodbreak

\subsection{Type equality \fbox{$\eqtype{\G}{\A}{\B}$}}
\label{sec:type-equality}

\subsubsection*{General rules}

\begin{mathpar}
  \infer[\rl{eq-ty-conv}]
  {\eqtype{\G}{\A}{\B}\\
    \eqctx{\G}{\D}}
  {\eqtype{\D}{\A}{\B}}

  \infer[\rl{eq-ty-refl}]
  {\istype{\G}{\A}}
  {\eqtype{\G}{\A}{\A}}

  \infer[\rl{eq-ty-sym}]
  {\eqtype{\G}{\B}{\A}}
  {\eqtype{\G}{\A}{\B}}

  \infer[\rl{eq-ty-trans}]
  {\eqtype{\G}{\A}{\B}\\
   \eqtype{\G}{\B}{\C}}
  {\eqtype{\G}{\A}{\C}}
\end{mathpar}

\subsubsection*{Substitution}

\begin{mathpar}
  \infer[\rl{eq-ty-subst-id}]
  {\istype{\G}{\A}}
  {\eqtype{\G}
     {\subst{\A}{\sbid{\G}}}
     {\A}
  }

  \infer[\rl{eq-ty-subst-compose}]
  {\issubst{\sbs}{\G}{\D} \\
   \issubst{\sbt}{\D}{\E} \\
   \istype{\E}{\A}
  }
  {\eqtype{\G}
    {\subst{\A}{\sbcomp{\sbs}{\sbt}}}
    {\subst{(\subst{\A}{\sbs})}{\sbt}}
  }

  \infer[\rl{eq-ty-subst-prod}]
  {\issubst{\sbs}{\G}{\D} \\
   \istype{\D}{\A} \\
   \istype{\ctxextend{\D}{\A}}{\B}
  }
  {\eqtype{\G}
   {\subst{(\Prod{\A}{\B})}{\sbs}}
   {\Prod
     {\subst{\A}{\sbs}}
     {\subst{\B}{\sbshift{\sbs}{\G}{\A}}}
   }
  }

  \infer[\rl{eq-ty-subst-id}]
  {\issubst{\sbs}{\G}{\D} \\
   \istype{\D}{\A} \\
   \isterm{\D}{\uu}{\A} \\
   \isterm{\D}{\vv}{\A}
  }
  {\eqtype{\G}
   {\subst{(\Id{\A}{\uu}{\vv})}{\sbs}}
   {\Id{\subst{\A}{\sbs}}{\subst{\uu}{\sbs}}{\subst{\vv}{\sbs}}}
  }
\end{mathpar}

\subsubsection*{Congruence rules}

\begin{mathpar}
  \infer[\rl{cong-prod}]
  {\eqtype{\G}{\A_1}{\B_1}\\
   \eqtype{\ctxextend{\G}{\A_1}}{\A_2}{\B_2}}
  {\eqtype{\G}{\Prod{\A_1}{\A_2}}{\Prod{\B_1}{\B_2}}}

  \infer[\rl{cong-id}]
  {\eqtype{\G}{\A}{\B}\\
   \eqterm{\G}{\uu_1}{\vv_1}{\A}\\
   \eqterm{\G}{\uu_2}{\vv_2}{\A}
  }
  {\eqtype{\G}{\Id{\A}{\uu_1}{\uu_2}}
              {\Id{\B}{\vv_1}{\vv_2}}}

  \infer[\rl{cong-ty-subst}]
  {\issubst{\sbs}{\G}{\D} \\
   \eqtype{\D}{\A}{\B}
  }
  {\eqtype{\G}{\subst{\A}{\sbs}}{\subst{\B}{\sbs}}}
\end{mathpar}

\goodbreak

\subsection{Term equality \fbox{$\eqterm{\G}{\uu_1}{\uu_2}{\A}$}}

\subsubsection*{General rules}

\begin{mathpar}
  \infer[\rl{eq-ty-conv}]
  {\eqterm{\G}{\uu}{\vv}{\A}\\
    \eqtype{\G}{\A}{\B}}
  {\eqterm{\G}{\uu}{\vv}{\B}}

  \infer[\rl{eq-ctx-conv}]
  {\eqterm{\G}{\uu}{\vv}{\A}\\
    \eqctx{\G}{\D}}
  {\eqterm{\D}{\uu}{\vv}{\A}}

  \infer[\rl{eq-refl}]
  {\isterm{\G}{\uu}{\A}}
  {\eqterm{\G}{\uu}{\uu}{\A}}

  \infer[\rl{eq-sym}]
  {\eqterm{\G}{\vv}{\uu}{\A}}
  {\eqterm{\G}{\uu}{\vv}{\A}}

  \infer[\rl{eq-trans}]
  {\eqterm{\G}{\uu}{\vv}{\A}\\
   \eqterm{\G}{\vv}{\ww}{\A}}
  {\eqterm{\G}{\uu}{\ww}{\A}}
\end{mathpar}

\subsubsection*{Substitutions}

\begin{mathpar}
  \infer[\rl{eq-subst-id}]
  {\isterm{\G}{\uu}{\A}}
  {\eqterm{\G}
     {\subst{\uu}{\sbid{\G}}}
     {\uu}
     {\A}
  }

  \infer[\rl{eq-subst-compose}]
  {\issubst{\sbs}{\G}{\D} \\
   \issubst{\sbt}{\D}{\E} \\
   \isterm{\E}{\uu}{\A}
  }
  {\eqterm{\G}
    {\subst{\uu}{\sbcomp{\sbs}{\sbt}}}
    {\subst{(\subst{\uu}{\sbs})}{\sbt}}
    {\subst{\A}{\sbcomp{\sbs}{\sbt}}}
  }

  \infer[\rl{eq-subst-weak}]
  {\isterm{\G}{\var{k}}{\A} \\
   \istype{\G}{\B}   
  }
  {\eqterm{\ctxextend{\G}{\B}}
   {\subst{\var{k}}{\sbweak{\G}{\B}}}
   {\var{k+1}}
   {\subst{\A}{\sbweak{\G}{\B}}}
  }

  \infer[\rl{eq-subst-extend-zero}]
  {\issubst{\sbs}{\G}{\D} \\
   \istype{\D}{\A} \\
   \isterm{\G}{\uu}{\subst{\A}{\sbs}}
  }
  {\eqterm{\G}
     {\subst{\var{0}}{\sbextend{\sbs}{\A}{\uu}}}
     {\uu}
     {\subst{\A}{\sbs}}
  }

  \infer[\rl{eq-subst-extend-succ}]
  {\issubst{\sbs}{\G}{\D} \\
   \isterm{\D}{\var{k}}{\A} \\
   \isterm{\G}{\uu}{\subst{\B}{\sbs}}
  }
  {\eqterm{\G}
     {\subst
        {\var{k+1}}
        {\sbextend{\sbs}{\B}{\uu}}
     }
     {\subst{\var{k}}{\sbs}}
     {\subst{\A}{\sbs}}
  }

  \infer[\rl{eq-subst-abs}]
  {\issubst{\sbs}{\G}{\D} \\
   \isterm{\ctxextend{\D}{\A}}{\uu}{\B}
  }
  {\eqterm{\G}
    {\subst{(\lam{\A}{\B} \uu)}{\sbs}}
    {(\lam
      {\subst{\A}{\sbs}}
      {\subst
        {\B}
        {\sbshift{\sbs}{\G}{\subst{\A}{\sbs}}}
      }
      \subst{\uu}{\sbextend{\sbs}{\A}{\var{0}}})
    }
    {\Prod
      {\subst{\A}{\sbs}}
      {\subst
        {\B}
        {\sbshift{\sbs}{\G}{\subst{\A}{\sbs}}}
      }
    }
  }

  \infer[\rl{eq-subst-app}]
  {\issubst{\sbs}{\G}{\D} \\
   \istype{\ctxextend{\D}{\A}}{\B} \\
   \isterm{\D}{\uu}{\Prod{\A}{\B}} \\
   \isterm{\D}{\vv}{\A}
  }
  {\eqterm{\G}
   {\subst{(\app{\uu}{\A}{\B}{\vv})}{\sbs}}
   {\app
      {\subst{\uu}{\sbs}}
      {\subst{\A}{\sbs}}
      {\subst
        {\B}
        {\sbshift{\sbs}{\G}{\subst{\A}{\sbs}}}
      }
      {\subst{\vv}{\sbs}}}
   {\subst
     {(\subst{\B}{\sbextend{\sbid{\G}}{\A}{\vv}})}
     {\sbs}
   }
  }

  \infer[\rl{eq-subst-refl}]
  {\issubst{\sbs}{\G}{\D} \\
   \isterm{\D}{\uu}{\A}
  }
  {\eqterm{\G}
   {\subst{(\refl{\A}{\uu})}{\sbs}}
   {\refl{\subst{\A}{\sbs}}{\subst{\uu}{\sbs}}}
   {\Id{\subst{\A}{\sbs}}{\subst{\uu}{\sbs}}{\subst{\uu}{\sbs}}}
  }
\end{mathpar}

\subsubsection*{Equality reflection}
%
\begin{mathpar}
  \infer[\rl{eq-reflection}]
  {\isterm{\G}{\ww_1}{\Id{\A}{\uu}{\vv}} \\
   \isterm{\G}{\ww_2}{\textsf{UIP}(\A)}
  }
  {\eqterm{\G}{\uu}{\vv}{\A}}
\end{mathpar}
%
Here $\mathsf{UIP}(\A)$ is an abbreviation for what is written traditionally as
``all parallel paths in $\A$ are (propositionally) equal''.

\subsubsection*{Computation and Extensionality}

\begin{mathpar}
\infer[\rl{prod-beta}]
  {\isterm{\ctxextend{\G}{\A}}{\uu}{\B}\\
    \isterm{\G}{\vv}{\A}}
  {\eqterm{\G}{\bigl(\app{(\lam{\A}{\B}{\uu})}{\A}{\B}{\vv}\bigr)}
              {\subst{\uu}{\sbextend{\sbid{\G}}{\A}{\vv}}}
              {\subst{\B}{\sbextend{\sbid{\G}}{\A}{\vv}}}}

  % \infer[\rl{uip}]
  % {\isterm{\G}{\vv_1}{\Id{\A}{\uu}{\vv}} \\
  %   \isterm{\G}{\vv_2}{\Id{\A}{\uu}{\vv}}
  % }
  % {\eqterm{\G}{\vv_1}{e'_2}{\Id{\A}{\uu}{\vv}}}

  \infer[\rl{prod-eta}]
  {\isterm{\G}{\uu}{\Prod{\A}{\B}}\\
   \isterm{\G}{\vv}{\Prod{\A}{\B}}\\\\
   \eqterm
      {\ctxextend{\G}{\A}}
      {(\app
          {\subst{\uu}{\sbweak{\G}{\A}}}
          {\subst{\A}{\sbweak{\G}{\A}}}
          {\subst
            {\B}
            {\sbshift
              {\sbweak{\G}{\A}}
              {\ctxextend{\G}{\A}}
              {\A}
            }
          }
          {\var{0}})
      }
      {(\app
          {\subst{\vv}{\sbweak{\G}{\A}}}
          {\subst{\A}{\sbweak{\G}{\A}}}
          {\subst
            {\B}
            {\sbshift
              {\sbweak{\G}{\A}}
              {\ctxextend{\G}{\A}}
              {\A}
            }
          }
          {\var{0}})
      }
      {\B}
  }
  {\eqterm{\G}{\uu}{\vv}{\Prod{\A}{\B}}}
\end{mathpar}

The rule $\rl{prod-eta}$ is optional. We may prefer not to keep it around, as it amounts
to function extensionality.

\subsubsection*{Congruence rules}

\begin{mathpar}

  \infer[\rl{cong-abs}]
  {\eqtype{\G}{\A_1}{\B_1}\\
    \eqtype{\ctxextend{\G}{\A_1}}{\A_2}{\B_2}\\
    \eqterm{\ctxextend{\G}{\A_1}}{\uu_1}{\uu_2}{\A_2}}
  {\eqterm{\G}{(\lam{\A_1}{\A_2}{\uu_1})}
              {(\lam{\B_1}{\B_2}{\uu_2})}
              {\Prod{\A_1}{\A_2}}}

  \infer[\rl{cong-app}]
  {\eqtype{\G}{\A_1}{\B_1}\\
   \eqtype{\ctxextend{\G}{\A_1}}{\A_2}{\B_2}\\\\
   \eqterm{\G}{\uu_1}{\vv_1}{\Prod{\A_1}{\A_2}}\\
   \eqterm{\G}{\uu_2}{\vv_2}{\A_1}}
  {\eqterm
    {\G}
    {(\app{\uu_1}{\A_1}{\A_2}{\uu_2})}
    {(\app{\vv_1}{\B_1}{\B_2}{\vv_2})}
    {\subst{\A_2}{\sbextend{\sbid{\G}}{\A_1}{\uu_2}}}
  }

  \infer[\rl{cong-refl}]
  {\eqterm{\G}{\uu_1}{\uu_2}{\A_1}\\
    \eqtype{\G}{\A_1}{\A_2}}
  {\eqterm{\G}{\refl{\A_1} \uu_1}{\refl{\A_2} \uu_2}{\Id{\A_1}{\uu_1}{\uu_1}}}


  \infer[\rl{cong-term-subst}]
  {\issubst{\sbs}{\G}{\D} \\
   \eqterm{\D}{\uu_1}{\uu_2}{\A}
  }
  {\eqterm{\G}{\subst{\uu_1}{\sbs}}{\subst{\uu_2}{\sbs}}{\subst{\A}{\sbs}}}

\end{mathpar}


%%% Local Variables:
%%% mode: latex
%%% TeX-master: "main"
%%% End:





\section{The Theorem we Want to Prove}

The theorem could now be like.

\begin{theorem}
  \leavevmode
  \begin{enumerate}
    \item If $\Gamma \dere A\ \type$ then $\Gamma$ has a translation $\barG$,
    and for any such $\barG$, $A$ has a translation $\barA$ such that
    $\barG \der \barA\ \type$. Furthermore, if $\barA'$ is provided as another
    translation of $A$, we also get an isomporphism between $\barA$ and
    $\barA'$.
    \item If $\Gamma \dere t \infers T$ then $\Gamma$ has a translation $\barG$,
    and for any such $\barG$, $t$ and $T$ have translations $\bart$ and $\barT$
    such that $\barG \der \bart : \barT$. Besides, if $\barT'$ is another
    translation of $T$, we get an isomorphism between $\barT$ and $\barT'$.
    \item If $\Gamma \dere t \checks T$ then $\Gamma$ has a translation $\barG$,
    and for any such $\barG$ and a translation $\barT$ of $T$, we have a
    translation $\bart$ of $t$ such that $\barG \der \bart : \barT$.
    \item If $\Gamma \dere A \equiv B$ then $\Gamma$ has a translation $\barG$
    and for any such $\barG$ and $\barA$ translation of $A$ then $B$ as a
    translation $\barB$ such that $\barA$ and $\barB$ have an isomorphism in
    $\barG$.
    \item If $\Gamma \dere A \equiv B$ then $\Gamma$ has a translation $\barG$
    and for any such $\barG$ and $\barA$, $\barB$ translations of $A$ and $B$
    then $\barA$ and $\barB$ have an isomorphism in $\barG$.
    \item  If $\Gamma \dere u \equiv v : A$ then $\Gamma$ has a translation
    $\barG$ and for any such $\barG$ and $\barA$ translation of $A$ then we
    have $\barp$, $\baru$ and $\barv$ such that
    $\barG \der \barp : \Id{\barA}{\baru}{\barv}$.
  \end{enumerate}
\end{theorem}

\meta{Furthermore, the notion of \emph{translation} preserves the shape of
types.}

\begin{proof}
  By induction on the derivation.
  \meta{We will focus on the (hopefully) hard cases.}
  \begin{caselist}
    \nextcase
    \begin{mathpar}
      \ru{\Gamma \dere A\ \type \qquad
          \Gamma, x:A \dere B\ \type
        }{\Gamma \dere \Pi x:A.B\ \type}
    \end{mathpar}
    By first induction hypothesis, we get a $\barG$ translation of $\Gamma$,
    now assume we take any such $\barG$.
    By first induction hypothesis again, we get $\barA$ such that
    $\barG \der \barA\ \type$.
    Now, $\barG, x:\barA$ is a valid translation of $\Gamma, x:A$ so,
    by second induction hypothesis, we get $\barB$ such that
    $\barG, x:\barA \der \barB\ \type$.
    From that we deduce $\barG \der \Pi x:\barA.\barB\ \type$ which is a valid
    translation of our goal.

    Finally, assume we are given a translation of $\Pi x:A.B$, by definition
    it has to be a $\Pi$-type as well. It is thus some $\Pi x:\barA'.\barB'$
    where $\barA'$ and $\barB'$ are respective translations of $A$ and $B$.
    By first induction hypothesis, we get $f$ an isomorphism between $\barA$
    and $\barA'$. By the second, we get $g$ an iso between $\barB$ and $\barB'$.
    As such,
    \begin{equation*}
    \lambda (e:(\Pi x:\barA.\barB).\Pi x:\barA'.\barB'). g \circ e \circ f^{-1}
    \end{equation*}
    is what we are looking for.

    \nextcase
    \begin{mathpar}
      \ru{\Gamma \der T\ \type \qquad
          \Gamma \der t,t' \checks T
        }{\Gamma \der \Id{T}{t}{t'}\ \type}
    \end{mathpar}
    By IH1, we can translate the context and take such a $\barG$.
    By IH1, we also get $\barT$ a translation of $T$ such that
    $\barG \der \barT\ \type$.
    We can thus use IH2 and IH3 to get $\bart$ and $\bart'$ such that
    $\barG \der \bart, \bart' : \barT$.
    This way, we get $\barG \der \Id{\barT}{\bart}{\bart'}\ \type$.

    Now assume we are given a translation of $\Id{T}{t}{t'}$, this means that
    we are given some $\Id{\barT'}{\baru}{\barv}$.
    By IH1, we get an isomporphism between $\barT$ and $\barT'$.
    \meta{Now we need to know what we can deduce from checking derivations...}

    \nextcase
    \begin{mathpar}
      \ru{\Gamma \dere t \checks \Pi x:A.B \qquad
          \Gamma \dere u \checks A
        }{\Gamma \dere \app{x:A.B}{t}{u} \infers B[u/x]}
    \end{mathpar}
    By first induction hypothesis we translate the context, and for any such
    translation $\barG$, \meta{we need to change the rule so that the premises
    can produce a translation of the types.}

    \nextcase
    \begin{mathpar}
      \ru{\Gamma \dere t \infers A \qquad
          \Gamma \dere B\ \type \qquad
          \Gamma \dere A \equiv B
        }{\Gamma \dere t \checks B}
    \end{mathpar}
    By first induction hypothesis, we get $\barG$ as a translation of $\Gamma$
    and now assume any such $\barG$. Furthermore, we assume $\barB$ as a
    translation of $B$, we want a translation $\bart$ of $t$ such that
    $\barG \der \bart : \barB$.

    By first IH, we have $\bart$ and $\barA$ such that
    $\barG \der \bart : \barA$. Since $\Gamma \dere A \equiv B$, we have
    an isomorphism $f$ in $\barG$ between $\barA$ and $\barB$.
    Thus $\barG \der \app{\barA \to \barB}{f}{\bart} : \barB$ which allows us
    to conclude.

    \nextcase
    \begin{mathpar}
      \ru{\Gamma \dere p : \Id{T}{t}{t'}
        }{\Gamma \dere t \equiv t' : T}
    \end{mathpar}
    \meta{Shouldn't we add as a premise that $T$ is an hSet?}
    By IH we deduce a translation of $\Gamma$ and now assume we have some such
    $\barG$. Now assume we aslo have $\barT$ a translation of $T$.
    \meta{How to proceed without translations of $t$ and $t'$?}

    \nextcase
    \begin{mathpar}
      \ru{\Gamma, x:U \der t:V \qquad
          \Gamma \der u : U
        }{\Gamma \der
          \app{x:U.V}{(\lambda (x:U.V).t)}{u} \equiv t[u/x] : V[u/x]}
    \end{mathpar}
    \meta{TODO}
  \end{caselist}
\end{proof}

\newpage
\hrulefill
OLD STUFF BELOW

Should the theorem be like:

\begin{theorem}
  If $\Gamma \dere t : T$, then there exists
  $\der \barG \in \den{\dere \Gamma}$ and for any such $\barG$ there exists
  $\barT$ such that $\barG \der \barT\ \type \in \den{\Gamma \dere T}$
  and for any such $\barT$ there exists $\bart$ such that
  $\barG \der \bart : \barT \in \den{\Gamma \dere t : T}$.
\end{theorem}
%
Where we have yet to define the translations sets $\den{\_}$.
We also have to tell explicitely how the theorem works on other judgments.
\meta{Also, we could use the bidirectionality for something...}

\begin{proof}
  By induction on the derivation.
  \begin{caselist}
    \nextcase
    \begin{mathpar}
      \ru{\Gamma \dere A\ \type \qquad
          \Gamma, x:A \dere B\ \type
        }{\Gamma \dere \Pi x:A.B\ \type}
    \end{mathpar}
    \meta{Assuming} context extension and translation commute, it works.

    \nextcase
    \begin{mathpar}
      \ru{\Gamma \dere T\ \type \qquad
          \Gamma \dere t,t' \checks T
        }{\Gamma \dere \Id{T}{t}{t'}\ \type}
    \end{mathpar}
    Same here.

    \nextcase
    \begin{mathpar}
      \ru{\Gamma \dere \Pi x:A.B\ \type \qquad
          \Gamma, x:A \dere t \checks B
        }{\Gamma \dere \lambda (x:A.B).t \infers \Pi x:A.B}
    \end{mathpar}
    By first induction hypothesis, we have $\der \barG \in \den{\dere \Gamma}$.
    Now given such $\barG$, using the same induction hypothesis we have
    $\barG \der \Pi x:\barA.\barB\ \type
    \in \den{\Gamma \dere \Pi x:A.B\ \type}$
    (\meta{assuming} translation of types preserves shapes).
    Now given such $\Pi x:\barA.\barB$, we in particular have
    $\barG \der \barA\ \type \in \den{\Gamma \dere A\ \type}$
    (\meta{assuming} we have some kind of inversion linked with the previous
    assumption). The same goes for $\barB$.

    Now, we have $\der \barG, x:\barA \in \den{\dere \Gamma, x:A}$.
    Thus by second induction hypothesis with our choice of $\barA$ and
    $\barB$, we have $\bart$ such that
    $\barG, x:\barA \der \bart \checks \barB \in
    \den{\Gamma, x:A \dere t \checks B}$.
    Thus $\barG \der \lambda (x:\barA.\barB).\bart \infers \Pi x:\barA.\barB
    \in \den{\Gamma \dere \lambda (x:A.B).t \infers \Pi x:A.B}$.

    \nextcase
    \begin{mathpar}
      \ru{\Gamma \dere t \checks \Pi x:A.B \qquad
          \Gamma \dere u \checks A
        }{\Gamma \dere \app{x:A.B}{t}{u} \infers B[u/x]}
    \end{mathpar}
    \meta{What liberty do we have in providing a translation of $B[u/x]$? The
    ideal would that it has to be of the form $\barB[\baru/x]$.}

    \nextcase
    \begin{mathpar}
      \ru{\Gamma \dere t \checks T
        }{\Gamma \dere \refl{T}{t} \infers \Id{T}{t}{t}}
    \end{mathpar}
    It seems to work under the same \meta{assumptions} as before (commutation
    between context extension and translation ; inversion of translation of
    types).

    \nextcase
    \begin{mathpar}
      \ru{\Gamma \dere t \infers A \qquad
          \Gamma \dere B\ \type \qquad
          \Gamma \dere A \equiv B
        }{\Gamma \dere t \checks B}
    \end{mathpar}
    \meta{We need to know how to translate equalities to conclude...}
    \meta{Assuming it gives us an equivalence (provided the context and the two
    involved types) we can conclude.}

    \nextcase
    \begin{mathpar}
      \ru{\Gamma, x:U \dere t:V \qquad
          \Gamma \dere u : U
        }{\Gamma \dere \app{x:U.V}{(\lambda (x:U.V).t)}{u} \equiv
          t[u/x] : V[u/x]}
    \end{mathpar}
    \meta{The same problem with translations of $V[u/x]$.}

    \nextcase
    \begin{mathpar}
      \ru{\Gamma \dere t : \Pi x:U.V
        }{\Gamma \dere t \equiv \lambda (x:U.V).\app{x:U.V}{t}{x} : \Pi x:U.V}
    \end{mathpar}
    \meta{TODO}

    \nextcase
    \begin{mathpar}
      \ru{\Gamma \dere p : \Id{T}{t}{t'}
        }{\Gamma \dere t \equiv t' : T}
    \end{mathpar}
    We get exactly what we want from the induction hypothesis (\meta{assuming}
    that for terms we require a proof term of equality).
  \end{caselist}
\end{proof}

\meta{Just an idea like that: Why not translating to OTT instead of ITT? This
way, it might reveal clearer what is an irrelevant equality since they would
compute more...}

\end{document}
