% Theo's old stuff

% Old macros
%%% Other macros by Theo

% evergreens
\newcommand{\der}{\,\vdash}
\newcommand{\Der}{\,\Vdash}

%specific judgments
\newcommand{\dere}{\der_\mathbf{E}}

% semantic brackets
\def\lv{\mathopen{{[\kern-0.14em[}}}    % opening [[ value delimiter
\def\rv{\mathclose{{]\kern-0.14em]}}}   % closing ]] value delimiter
\newcommand{\den}[1]{\lv #1 \rv}
\newcommand{\Den}[3][]{\den{#2}^{#1}_{#3}}
\newcommand{\dent}[2]{\llparenthesis#1\rrparenthesis_{#2}}

% space-preserving paragraph headings
\newcommand{\subheading}[1]{\subparagraph{#1.}} %Alt: \subsection{#1}
% \newcommand{\paradot}[1]{\subparagraph{#1.}}
\newcommand{\paradot}[1]{\subsection*{#1.}}

% Inference rules
%\newcommand{\rulename}[1]{\ensuremath{\mbox{\sc#1}}}
\newcommand{\ru}[2]{\dfrac{\begin{array}[b]{@{}c@{}} #1 \end{array}}{#2}}
\newcommand{\rux}[3]{\ru{#1}{#2}~#3}
\newcommand{\nru}[3]{#1\ \ru{#2}{#3}}
\newcommand{\nrux}[4]{#1\ \ru{#2}{#3}\ #4}
\newcommand{\dstack}[2]{\begin{array}[b]{c}#1\\#2\end{array}}
\newcommand{\dru}[3]{\ru{\dstack{#1}{#2}}{#3}}
\newcommand{\drux}[4]{\ru{\dstack{#1}{#2}}{#3}\ #4}
\newcommand{\tru}[4]{\dru{\dstack{#1}{#2}}{#3}{#4}}
\newcommand{\trux}[5]{\dru{\dstack{#1}{#2}}{#3}{#4}\ #5}
\newcommand{\qru}[5]{\tru{\dstack{#1}{#2}}{#3}{#4}{#5}}
\newcommand{\ndru}[4]{#1\ \ru{\dstack{#2}{#3}}{#4}}
\newcommand{\ndrux}[5]{#1\ \ru{\dstack{#2}{#3}}{#4}\ #5}

% proof by cases
\newenvironment{caselist}{%
  \begin{list}{{\it Case}}{}%
}{\end{list}%
}
\newenvironment{subcaselist}{%
  \begin{list}{{\it Subcase}}{}%
}{\end{list}%
}
\newenvironment{subsubcaselist}{%
  \begin{list}{{\it Subsubcase}}{}%
}{\end{list}%
}

\newcommand{\nextcase}{\item~}

% Symbols and names
\DeclareMathOperator{\J}{\mathbf{J}}
\DeclareMathOperator{\type}{\mathbf{type}}
%\newcommand{\Id}[3]{\mathbf{Id}_{#1}\,#2\ #3}
%\newcommand{\app}[3]{#2\ @^{#1}\ #3}
\newcommand\infers{\rightrightarrows}
\newcommand\checks{\leftleftarrows}
%\DeclareMathOperator{\refl}{\mathbf{refl}}
\newcommand\barG{\overline{\Gamma}}
\newcommand\bart{\overline{t}}
\newcommand\barT{\overline{T}}
\newcommand\barA{\overline{A}}
\newcommand\barB{\overline{B}}
\newcommand\baru{\overline{u}}
\newcommand\barv{\overline{v}}
\newcommand\barp{\overline{p}}



\section{The Theorem we Want to Prove}

The theorem could now be like.

\begin{theorem}
  \leavevmode
  \begin{enumerate}
    \item If $\Gamma \dere A\ \type$ then $\Gamma$ has a translation $\barG$,
    and for any such $\barG$, $A$ has a translation $\barA$ such that
    $\barG \der \barA\ \type$. Furthermore, if $\barA'$ is provided as another
    translation of $A$, we also get an isomporphism between $\barA$ and
    $\barA'$.
    \item If $\Gamma \dere t \infers T$ then $\Gamma$ has a translation $\barG$,
    and for any such $\barG$, $t$ and $T$ have translations $\bart$ and $\barT$
    such that $\barG \der \bart : \barT$. Besides, if $\barT'$ is another
    translation of $T$, we get an isomorphism between $\barT$ and $\barT'$.
    \item If $\Gamma \dere t \checks T$ then $\Gamma$ has a translation $\barG$,
    and for any such $\barG$ and a translation $\barT$ of $T$, we have a
    translation $\bart$ of $t$ such that $\barG \der \bart : \barT$.
    \item If $\Gamma \dere A \equiv B$ then $\Gamma$ has a translation $\barG$
    and for any such $\barG$ and $\barA$ translation of $A$ then $B$ as a
    translation $\barB$ such that $\barA$ and $\barB$ have an isomorphism in
    $\barG$.
    \item If $\Gamma \dere A \equiv B$ then $\Gamma$ has a translation $\barG$
    and for any such $\barG$ and $\barA$, $\barB$ translations of $A$ and $B$
    then $\barA$ and $\barB$ have an isomorphism in $\barG$.
    \item  If $\Gamma \dere u \equiv v : A$ then $\Gamma$ has a translation
    $\barG$ and for any such $\barG$ and $\barA$ translation of $A$ then we
    have $\barp$, $\baru$ and $\barv$ such that
    $\barG \der \barp : \Id{\barA}{\baru}{\barv}$.
  \end{enumerate}
\end{theorem}

\meta{Furthermore, the notion of \emph{translation} preserves the shape of
types.}

\begin{proof}
  By induction on the derivation.
  \meta{We will focus on the (hopefully) hard cases.}
  \begin{caselist}
    \nextcase
    \begin{mathpar}
      \ru{\Gamma \dere A\ \type \qquad
          \Gamma, x:A \dere B\ \type
        }{\Gamma \dere \Pi x:A.B\ \type}
    \end{mathpar}
    By first induction hypothesis, we get a $\barG$ translation of $\Gamma$,
    now assume we take any such $\barG$.
    By first induction hypothesis again, we get $\barA$ such that
    $\barG \der \barA\ \type$.
    Now, $\barG, x:\barA$ is a valid translation of $\Gamma, x:A$ so,
    by second induction hypothesis, we get $\barB$ such that
    $\barG, x:\barA \der \barB\ \type$.
    From that we deduce $\barG \der \Pi x:\barA.\barB\ \type$ which is a valid
    translation of our goal.

    Finally, assume we are given a translation of $\Pi x:A.B$, by definition
    it has to be a $\Pi$-type as well. It is thus some $\Pi x:\barA'.\barB'$
    where $\barA'$ and $\barB'$ are respective translations of $A$ and $B$.
    By first induction hypothesis, we get $f$ an isomorphism between $\barA$
    and $\barA'$. By the second, we get $g$ an iso between $\barB$ and $\barB'$.
    As such,
    \begin{equation*}
    \lambda (e:(\Pi x:\barA.\barB).\Pi x:\barA'.\barB'). g \circ e \circ f^{-1}
    \end{equation*}
    is what we are looking for.

    \nextcase
    \begin{mathpar}
      \ru{\Gamma \der T\ \type \qquad
          \Gamma \der t,t' \checks T
        }{\Gamma \der \Id{T}{t}{t'}\ \type}
    \end{mathpar}
    By IH1, we can translate the context and take such a $\barG$.
    By IH1, we also get $\barT$ a translation of $T$ such that
    $\barG \der \barT\ \type$.
    We can thus use IH2 and IH3 to get $\bart$ and $\bart'$ such that
    $\barG \der \bart, \bart' : \barT$.
    This way, we get $\barG \der \Id{\barT}{\bart}{\bart'}\ \type$.

    Now assume we are given a translation of $\Id{T}{t}{t'}$, this means that
    we are given some $\Id{\barT'}{\baru}{\barv}$.
    By IH1, we get an isomporphism between $\barT$ and $\barT'$.
    \meta{Now we need to know what we can deduce from checking derivations...}

    \nextcase
    \begin{mathpar}
      \ru{\Gamma \dere t \checks \Pi x:A.B \qquad
          \Gamma \dere u \checks A
        }{\Gamma \dere \app{x:A.B}{t}{u} \infers B[u/x]}
    \end{mathpar}
    By first induction hypothesis we translate the context, and for any such
    translation $\barG$, \meta{we need to change the rule so that the premises
    can produce a translation of the types.}

    \nextcase
    \begin{mathpar}
      \ru{\Gamma \dere t \infers A \qquad
          \Gamma \dere B\ \type \qquad
          \Gamma \dere A \equiv B
        }{\Gamma \dere t \checks B}
    \end{mathpar}
    By first induction hypothesis, we get $\barG$ as a translation of $\Gamma$
    and now assume any such $\barG$. Furthermore, we assume $\barB$ as a
    translation of $B$, we want a translation $\bart$ of $t$ such that
    $\barG \der \bart : \barB$.

    By first IH, we have $\bart$ and $\barA$ such that
    $\barG \der \bart : \barA$. Since $\Gamma \dere A \equiv B$, we have
    an isomorphism $f$ in $\barG$ between $\barA$ and $\barB$.
    Thus $\barG \der \app{\barA \to \barB}{f}{\bart} : \barB$ which allows us
    to conclude.

    \nextcase
    \begin{mathpar}
      \ru{\Gamma \dere p : \Id{T}{t}{t'}
        }{\Gamma \dere t \equiv t' : T}
    \end{mathpar}
    \meta{Shouldn't we add as a premise that $T$ is an hSet?}
    By IH we deduce a translation of $\Gamma$ and now assume we have some such
    $\barG$. Now assume we aslo have $\barT$ a translation of $T$.
    \meta{How to proceed without translations of $t$ and $t'$?}

    \nextcase
    \begin{mathpar}
      \ru{\Gamma, x:U \der t:V \qquad
          \Gamma \der u : U
        }{\Gamma \der
          \app{x:U.V}{(\lambda (x:U.V).t)}{u} \equiv t[u/x] : V[u/x]}
    \end{mathpar}
    \meta{TODO}
  \end{caselist}
\end{proof}

\newpage
\hrulefill
OLD STUFF BELOW

Should the theorem be like:

\begin{theorem}
  If $\Gamma \dere t : T$, then there exists
  $\der \barG \in \den{\dere \Gamma}$ and for any such $\barG$ there exists
  $\barT$ such that $\barG \der \barT\ \type \in \den{\Gamma \dere T}$
  and for any such $\barT$ there exists $\bart$ such that
  $\barG \der \bart : \barT \in \den{\Gamma \dere t : T}$.
\end{theorem}
%
Where we have yet to define the translations sets $\den{\_}$.
We also have to tell explicitely how the theorem works on other judgments.
\meta{Also, we could use the bidirectionality for something...}

\begin{proof}
  By induction on the derivation.
  \begin{caselist}
    \nextcase
    \begin{mathpar}
      \ru{\Gamma \dere A\ \type \qquad
          \Gamma, x:A \dere B\ \type
        }{\Gamma \dere \Pi x:A.B\ \type}
    \end{mathpar}
    \meta{Assuming} context extension and translation commute, it works.

    \nextcase
    \begin{mathpar}
      \ru{\Gamma \dere T\ \type \qquad
          \Gamma \dere t,t' \checks T
        }{\Gamma \dere \Id{T}{t}{t'}\ \type}
    \end{mathpar}
    Same here.

    \nextcase
    \begin{mathpar}
      \ru{\Gamma \dere \Pi x:A.B\ \type \qquad
          \Gamma, x:A \dere t \checks B
        }{\Gamma \dere \lambda (x:A.B).t \infers \Pi x:A.B}
    \end{mathpar}
    By first induction hypothesis, we have $\der \barG \in \den{\dere \Gamma}$.
    Now given such $\barG$, using the same induction hypothesis we have
    $\barG \der \Pi x:\barA.\barB\ \type
    \in \den{\Gamma \dere \Pi x:A.B\ \type}$
    (\meta{assuming} translation of types preserves shapes).
    Now given such $\Pi x:\barA.\barB$, we in particular have
    $\barG \der \barA\ \type \in \den{\Gamma \dere A\ \type}$
    (\meta{assuming} we have some kind of inversion linked with the previous
    assumption). The same goes for $\barB$.

    Now, we have $\der \barG, x:\barA \in \den{\dere \Gamma, x:A}$.
    Thus by second induction hypothesis with our choice of $\barA$ and
    $\barB$, we have $\bart$ such that
    $\barG, x:\barA \der \bart \checks \barB \in
    \den{\Gamma, x:A \dere t \checks B}$.
    Thus $\barG \der \lambda (x:\barA.\barB).\bart \infers \Pi x:\barA.\barB
    \in \den{\Gamma \dere \lambda (x:A.B).t \infers \Pi x:A.B}$.

    \nextcase
    \begin{mathpar}
      \ru{\Gamma \dere t \checks \Pi x:A.B \qquad
          \Gamma \dere u \checks A
        }{\Gamma \dere \app{x:A.B}{t}{u} \infers B[u/x]}
    \end{mathpar}
    \meta{What liberty do we have in providing a translation of $B[u/x]$? The
    ideal would that it has to be of the form $\barB[\baru/x]$.}

    \nextcase
    \begin{mathpar}
      \ru{\Gamma \dere t \checks T
        }{\Gamma \dere \refl{T}{t} \infers \Id{T}{t}{t}}
    \end{mathpar}
    It seems to work under the same \meta{assumptions} as before (commutation
    between context extension and translation ; inversion of translation of
    types).

    \nextcase
    \begin{mathpar}
      \ru{\Gamma \dere t \infers A \qquad
          \Gamma \dere B\ \type \qquad
          \Gamma \dere A \equiv B
        }{\Gamma \dere t \checks B}
    \end{mathpar}
    \meta{We need to know how to translate equalities to conclude...}
    \meta{Assuming it gives us an equivalence (provided the context and the two
    involved types) we can conclude.}

    \nextcase
    \begin{mathpar}
      \ru{\Gamma, x:U \dere t:V \qquad
          \Gamma \dere u : U
        }{\Gamma \dere \app{x:U.V}{(\lambda (x:U.V).t)}{u} \equiv
          t[u/x] : V[u/x]}
    \end{mathpar}
    \meta{The same problem with translations of $V[u/x]$.}

    \nextcase
    \begin{mathpar}
      \ru{\Gamma \dere t : \Pi x:U.V
        }{\Gamma \dere t \equiv \lambda (x:U.V).\app{x:U.V}{t}{x} : \Pi x:U.V}
    \end{mathpar}
    \meta{TODO}

    \nextcase
    \begin{mathpar}
      \ru{\Gamma \dere p : \Id{T}{t}{t'}
        }{\Gamma \dere t \equiv t' : T}
    \end{mathpar}
    We get exactly what we want from the induction hypothesis (\meta{assuming}
    that for terms we require a proof term of equality).
  \end{caselist}
\end{proof}

\meta{Just an idea like that: Why not translating to OTT instead of ITT? This
way, it might reveal clearer what is an irrelevant equality since they would
compute more...}
