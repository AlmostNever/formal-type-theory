%%% Type theory

% Macros for referring to rules.
\newcommand{\ruleFont}[1]{\textsc{#1}}
\newcommand{\ruleRef}[2]{\hyperref[#2]{$\ruleFont{#1}$}}
\newcommand{\configParameters}[1]{{\footnotesize #1}}

\newcommand{\defequiv}{\ \mathbin{{:}{\equiv}}\ }

% Judgment forms
\newcommand{\isctx}[1]{#1\ \mathsf{ctx}} % well formed context
\newcommand{\istype}[2]{#1 \vdash #2 \ \mathsf{type}} % well formed type
\newcommand{\isterm}[3]{#1 \vdash #2 : #3} % well formed term
\newcommand{\issubst}[3]{#1 : #2 \to #3} % well formed substitution
\newcommand{\eqctx}[2]{#1 \equiv #2} % equal contexts
\newcommand{\eqsubst}[4]{#1 \equiv #2 : #3 \to #4} % equal substitutions
\newcommand{\eqtype}[3]{#1 \vdash #2 \equiv #3} % equal types
\newcommand{\eqterm}[4]{#1 \vdash #2 \equiv #3 : #4} % equal terms

%% Variables
\newcommand{\var}[1]{\mathsf{x}_{#1}} % de Bruijn index
\newcommand{\suc}[1]{#1 + 1} % successor of a de Bruijn index

%% Substitutions
\newcommand{\Subst}[2]{#1[#2]}           % substitution operation on types
\newcommand{\subst}[2]{#1[#2]}           % substitution operation on terms
\newcommand{\sbzero}[2]{\{#2\}_{#1}}     % substitition of the zeroth index
\newcommand{\sbweak}[1]{\mathsf{w}_{#1}} % weakening substitution
\newcommand{\sbshift}[2]{(#2 {\mid} #1)}   % shifted substitution
\newcommand{\sbid}{\mathsf{id}}          % identity substitution
\newcommand{\sbcomp}[2]{#1 \circ #2}     % composition of substitutions

%% Contexts
\newcommand{\ctxempty}{\bullet} % empty context
\newcommand{\ctxextend}[2]{#1, #2} % extended context

%% Products
\newcommand{\Prod}[1]{\mathop{\textstyle\prod_{#1}}} % dependent product
\newcommand{\lam}[2]{\lambda #1 . #2 .}           % $\lambda$-abstraction
\newcommand{\app}[4]{#1\mathbin{@^{#2.#3}} #4}    % application

%% Identity types
\newcommand{\Id}[3]{\mathsf{Id}_{#1}(#2,#3)}         % identity type
\newcommand{\refl}[2]{{\mathsf{refl}_{#1}}\,#2}     % reflexivity
\newcommand{\J}[6]{\mathsf{J}(#1,#2,#3,#4,#5,#6)} % the J eliminator

%% Simple products
\newcommand{\SimProd}[2]{#1 \times #2}
\newcommand{\pair}[4]{\langle #3,#4 \rangle_{#1,#2}}
\newcommand{\projOne}[3]{\pi_1^{#1,#2} \, #3}
\newcommand{\projTwo}[3]{\pi_2^{#1,#2} \, #3}

%% Empty
\newcommand{\Empty}{\mathsf{Empty}} % empty type
\newcommand{\exfalso}[2]{\mathsf{exfalso}_{#1}\,#2} % ex falso quodlibet

%% Unit
\newcommand{\Unit}{\mathsf{Unit}} % the unit type
\newcommand{\unit}{\star} % the unit inhabitant

%% Bool
\newcommand{\Bool}{\mathsf{Bool}} % bool type
\newcommand{\true}{\mathsf{true}} % the true boolean
\newcommand{\false}{\mathsf{false}} % the false boolean
\newcommand{\cond}[4]{\mathsf{cond}_{#1}(#2,#3,#4)} % condtional term

% Universes
% We always put universe levels in supscripts
\newcommand{\El}[2]{\mathsf{El}^{#1}(#2)}
\newcommand{\Univ}[1]{\mathsf{U}^{#1}} % universe
\newcommand{\propLevel}{\mathsf{prop}} % the level of propositions
\newcommand{\univ}[1]{\lceil #1 \rceil} % universe level
\newcommand{\uniProd}[4]{\mathsf{prod}^{#1,#2}(#3,#4)}
\newcommand{\uniId}[4]{\mathsf{id}^{#1}(#2,#3,#4)}
\newcommand{\uniEmpty}[1]{\mathsf{empty}^{#1}}
\newcommand{\uniUnit}[1]{\mathsf{unit}^{#1}}
\newcommand{\uniBool}[1]{\mathsf{bool}^{#1}}
\newcommand{\uniSimProd}[4]{\mathsf{times}^{#1,#2}(#3,#4)}
\newcommand{\uniUni}[1]{\mathsf{u}^{#1}} % probably won't be confused with "u" due to supscript

%%% Local Variables:
%%% mode: latex
%%% TeX-master: "main"
%%% End:
